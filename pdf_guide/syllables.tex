\section*{Consonant and Vowel Inventory}
\begin{itemize}
    \item \textbf{Consonants}: \textipa{/b/, /d/, f, 
    % /\textphi/,
    /g/, /h/, /k/, /\textbeltl/, /l/, /m/, /\ng/, /n/, /p/, /r/, /s/, /\textsubbridge{t}/, /v/, /w/, /j/, /\textesh/, /\texttheta/}
    \item \textbf{Vowels}: \textipa{/a/, /\textsubumlaut{a}/, /e/, /i/, /o/, /u/}
    \item \textbf{Dipthongs}: \textipa{/ai/, /uj/}
\end{itemize} 


% \subsection*{Rules for /f/ and /\textphi/}

% \textbf{/f/ (Voiceless Labiodental Fricative)}:

% At the beginning of words:
% \begin{itemize}
% \item \textipa{/falor/} (leader)
% \item \textipa{/fethr/} (speaker)
% \end{itemize}
% Between vowels:
% \begin{itemize}
% \item /afa/ (...)
% \item /ujfae/ (...)
% \end{itemize}

% When adjacent to voiced consonants:
% \begin{itemize}
% \item ...
% \end{itemize}
% \textbf{/\textphi/ (Voiceless Bilabial Fricative)}:

% At the end of words or plurals:
% \begin{itemize}
% \item \textipa{/te\textphi/} (fortress)
% \item \textipa{/te\textphi e/} (fortresses)
% \end{itemize}


% In newer or borrowed words to indicate foreign or technical terms:
% \begin{itemize}
% \item ... (machine, from gnomish)
% \end{itemize}

% \subsection*{Rules for /l/ and /\textbeltl/}

% ...

\section*{Syllable Structure}
\begin{itemize}
    \item \textbf{Basic Structure}: Each syllable should follow the general structure of (C)(C)V(C)(C), where:
    \begin{itemize}
        \item \textbf{C} = Consonant
        \item \textbf{V} = Vowel
    \end{itemize}
    \item \textbf{Consonant Clusters}:
    \begin{itemize}
        \item At the beginning of a syllable, allow a maximum of two consonants.
        \item At the end of a syllable, allow a single consonant or specific consonant clusters.
    \end{itemize}
\end{itemize}

\section*{Syllable Patterns}
\redcolor{TODO: CONFIRM WITH DICTIONARY}
\begin{enumerate}
    \item \textbf{V (Vowel Only)}: Rare but can occur in specific cases, especially with prefix and suffix modifications.
    \begin{itemize}
        \item \textipa{/a/, /e/, /i/}
    \end{itemize}
    \item \textbf{CV (Consonant + Vowel)}: The most common syllable pattern.
    \begin{itemize}
        \item Example: \textipa{/la/, /me/, /ti/}
    \end{itemize}
    \item \textbf{CVC (Consonant + Vowel + Consonant)}: Also common, adding a consonant to the end.
    \begin{itemize}
        \item Example: \textipa{/lan/, /mujl/, /tir/}
    \end{itemize}
    \item \textbf{CCV (Consonant Cluster + Vowel)}: Allowed at the beginning of words.
    \begin{itemize}
        \item Example: \textipa{/tla/, /fre/, /tru/}
    \end{itemize}
     \item \textbf{CCVC (Consonant Cluster + Vowel + Consonant)}: Complex but permissible.
    \begin{itemize}
        \item Example: \textipa{/tl\textsubumlaut{a}n/, /frel/, /trus/}
    \end{itemize}
    \item \textbf{CVCC (Consonant + Vowel + Consonant Cluster)}: Possible only at the end of words or as whole words, and only including /\texttheta/ or /\textesh/.
    \begin{itemize}
        \item Example: \textipa{/fe\texttheta r/, /pa\texttheta r/, /lor\textesh/}
    \end{itemize}
\end{enumerate}

\section*{Stress Patterns}
\begin{enumerate}
    \item \textbf{Primary Stress}: On the penultimate syllable.
    \begin{itemize}
        \item Example: /la-\textbf{dri}-el/, /lo-\textbf{pe}-lor/
    \end{itemize}
    \item \textbf{Secondary Stress}: On the third to last syllable if the word has more than three.
    \begin{itemize}
        \item Example: /ti-\textbf{la}-dri-el/, /ti-\textbf{lo}-pe-lor/, /mel-or-\textbf{e}-ra-e/
    \end{itemize}
\end{enumerate}

\section*{Diphthongs and Vowel Clusters}
Clearly enunciate most vowel clusters to maintain the phonetic distinctiveness and breathy quality.
\begin{itemize}
\item \textbf{Diphthongs}: Diphthongs are combinations of two vowel sounds within the same syllable, gliding smoothly from one vowel to the other. The primary diphthongs in are:
\begin{itemize}
\item \textipa{/ai/}: Pronounced as in "eye."
\item \textipa{/uj/}: Pronounced as in "buoy."
\end{itemize}
\item \textbf{Vowel Clusters}: When two vowels appear together but do not form a diphthong, they must be clearly enunciated to distinguish them as separate sounds.
\begin{itemize}
\item Example: \textipa{/ea/} similar to "idea", where both vowels are pronounced distinctly.
\end{itemize}
\end{itemize}

\section*{Examples}
\begin{description}
        \item /la-\textbf{dri}-el ya-\textbf{lo}-pe-lor \textbf{lo}-re \textbf{em}-ril \textbf{si}-ne\textesh/ 
        \hanging The person quickly sees the city.
\end{description}




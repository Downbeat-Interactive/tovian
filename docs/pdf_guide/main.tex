\documentclass[letterpaper,twocolumn,openany,nodeprecatedcode,hidelinks]{dndbook}
\usepackage[english]{babel}
\usepackage[utf8]{inputenc}
\usepackage{pdfpages}
\usepackage{graphicx}
\usepackage{wrapfig}
\usepackage{color}
\usepackage{tikz}
\usepackage{lipsum}
\usepackage{hyperref}
\usepackage{titling}
\usepackage{pagecolor}
\usepackage{afterpage}
\usepackage{titletoc}
\usepackage{everyshi}
\usepackage{changepage}
\usepackage{geometry}
\usepackage{xstring}
\usepackage{afterpage}
\usepackage{multicol}

\usepackage{tabularray}
\usepackage{amsmath}
\usepackage{amsfonts}
\usepackage{enumitem}
\usepackage{tipa}
\usepackage{tablefootnote}
\usepackage{linguex}
\usepackage{expex}

\usepackage{makecell}

\usepackage{float}

\renewcommand\glt{} % removes extra vertical space from translation line
\newcommand{\redcolor}[1]{\color{red}#1\normalcolor}

% \catcode`_=\active
% \def_#1{\ifmmode\sb{#1}\else$\sb{#1}$\fi}


\usetikzlibrary{calc}

% \usepackage[unicodeparmaite]{tengwarscript}
% \usepackage{tengwarscript}
% \usepackage[annataritalic]{tengwarscript}
% \pdfmapfile{=tengwarscript.map}


\usepackage{fontspec}
% \newfontface{\tovian}{Ancient-Tovian.ttf}
\newfontface{\tovian}{Tovian.ttf}
\newfontface{\tengwar}{tngani.ttf}
\newcommand*{\red}[1]{\color{red}#1\normalcolor}

\def\vc#1{$\vcenter{\vspace{-30pt}\hbox{\vspace*{-50pt}#1}}$}

\newenvironment{nopagebreak}
  {\par\nobreak\vfil\penalty0\vfilneg
   \vtop\bgroup}
  {\par\xdef\tpd{\the\prevdepth}\egroup
   \prevdepth=\tpd}

\begin{document}
% \setmainfont{Ancient-Tovian.ttf}

% \tengwarannataritalic[1.5]



% IDEAS:
% Speak vs stand means performance versus factual
% to tree: stay in once place long time (good thing)


% Fethrdriel
\title{
    % \Huge\Tformen\Tthuule\TTdotbelow\Troomen\Tanna\Tando\TTdot\Ttelco\TTdot\Tlambe\TTdotbelow\\
    % {\fontsize{60pt}{30pt}\tovian ««ishmahluorniara»»}\\ % ˈiʃmahɫuorɳiara
    {\fontsize{80pt}{30pt}\tovian «««fethrydriel»»»}\\ % ˈapɸeθjdrieɫ
    % {\Huge\tengwar fethrydriel}\\
    \vspace{1in}
    \large \textit{Beginner's Guide to Tovian} \\
    % \Tando\Tsilme\TTacute\Ttinco\TTdoubler\TTthreedots\Tyanta
     % \Tquesse\TTacute\Ttinco\TTdoubler\TTthreedots\\
     % \Tquesse\TTacute\Ttinco\TTdoubler\TTthreedots\\
    % \huge{Tovis}\\
    % \begin{tikzpicture}[remember picture]\\%
        % \node (main) {\includegraphics[width=7in]{img/tovis2.png}};\\%
    % \end{tikzpicture}
    % \vspace{-1in}
    }

% \author{\large Jackson Dean}
\date{}

\newcommand{\continue}[0]{
% \textit{continued on the next page...}
\newpage
}

\newcommand\blfootnote[1]{%
  \begingroup
  \renewcommand\thefootnote{}\footnotetext{#1}%
  %\addtocounter{footnote}{-1}% not needed
  \endgroup
}
\titlespacing*{\subsubsection}
{0pt}{0pt}{0pt}

\newcommand{\ra}[0]{*}
\newcommand{\linkplaceholder}[0]{\subsubsubsection{}}

\newcounter{tablectr}

\newcommand*{\itemrow}[1]{
    \stepcounter{tablectr}%
    \StrSubstitute{#1}{ }{}[\temp] % Replace spaces in the name with nothing
    \StrSubstitute{\temp}{'}{}[\templabel] % Replace apostrophe
    \StrSubstitute{\templabel}{,}{}[\templabel] % Replace apostrophe
    \label{\templabel} 
    \arabic{tablectr}\subsubsubsection{}\stepcounter{tablectr}-\arabic{tablectr} & \textbf{#1}
}

\newcommand{\rarity}[3]{%
    \small{\textit{
    \newline #1, #2 %
    \ifbool{#3}{(*)}%
    }}
}

\newcolumntype{B}{X}
\newcolumntype{s}{>{\hsize=0.20\hsize}X}





\newcommand{\multicolinterrupt}[2]{% Stuff to span both rows
\setcounter{tempcolnum}{\col@number}
\end{multicols}
#1%
\begin{multicols}{#2}
}

% \newcommand{\todo}[1]{\color{red}#1\normalcolor}
\newcommand{\todo}[1]{\textit{#1}}

\renewcommand{\chaptermark}[1]{\markboth{#1}{}}

\newcommand{\hanging}[0]{\hspace{0.5in}}


\setcounter{page}{1}
\maketitle



\newcommand{\CLan}[0]{la} % animate class
\newcommand{\CLin}[0]{lo} % inanimate class
\newcommand{\CLab}[0]{le} % abstract class

% cases
\newcommand{\CAnom}[0]{i} % nominative case (default, often omitted)
\newcommand{\CAacc}[0]{yu} % accusative case
\newcommand{\CAgen}[0]{hi} % genitive case
\newcommand{\CAdat}[0]{mi} % dative case
\newcommand{\CAloc}[0]{ti} % locative case (place)
\newcommand{\CAtem}[0]{tu} % temporal case (time)
\newcommand{\CAins}[0]{si} % instrumental case
\newcommand{\CApur}[0]{lhu} % purpose case
\newcommand{\CApri}[0]{ni} % privitive case
\newcommand{\CAcom}[0]{yi} % comlative case
\newcommand{\CAexc}[0]{fi} % excessive case
\newcommand{\CAcomp}[0]{asl} % comparative case
\newcommand{\CAade}[0]{asl} % adessive case
\newcommand{\CAall}[0]{su} % allative case
\newcommand{\CAess}[0]{pu} % allative case


\newcommand{\CAabl}[0]{di} 

\newcommand{\casestable}[0]{
\subsection*{Noun Classes}
\begin{itemize}
    \item \textbf{Animate (living beings, includes plants)}
    \begin{itemize}
        \item \textbf{Prefix:} \textit{\CLan} 
        \item \textbf{Example:} \textit{\CLan-driel} (person)
    \end{itemize}
    \item \textbf{Inanimate (objects and places)}
    \begin{itemize}
        \item \textbf{Prefix:} \textit{\CLin-}
        \item \textbf{Example:} \textit{\CLin-pelor} (city)
    \end{itemize}
    \item \textbf{Abstract (concepts and ideas)}
    \begin{itemize}
        \item \textbf{Prefix:} \textit{\CLab-}
        \item \textbf{Example:} \textit{\CLab-ara} (wisdom)
    \end{itemize}
\end{itemize}

\subsection*{Noun Cases}
\begin{itemize}
    \item \textbf{Nominative (subject)}
    \begin{itemize}
        \item \textbf{Prefix:} \textit{\CAnom-} (default case, often omitted)
        \item \textbf{Example:} \textit{\CAnom-\CLan-driel} (the person as the subject)
    \end{itemize}
    \item \textbf{Accusative (direct object)}
    \begin{itemize}
        \item \textbf{Prefix:} \textit{\CAacc-}
        \item \textbf{Example:} \textit{\CAacc-\CLan-driel} (the person as the direct object)
        \item \textbf{Sentence}: \textit{\CAacc-\CLin-tel fa mrila \CAnom-\CLan-driel} (the person sees the lake)
    \end{itemize}
    \item \textbf{Genitive (possession)}
    \begin{itemize}
        \item \textbf{Prefix:} \textit{\CAgen-}
        \item \textbf{Example:} \textit{\CAgen-\CLan-driel} (of the person)
    \end{itemize}
    \item \textbf{Dative (indirect object)}
    \begin{itemize}
        \item \textbf{Prefix:} \textit{\CAdat-}
        \item \textbf{Example:} \textit{\CAdat-\CLan-driel} (to/for the person)
    \end{itemize}
    \item \textbf{Locative (location)}
    \begin{itemize}
        \item \textbf{Prefix:} \textit{\CAloc-}
        \item \textbf{Example:} \textit{\CAloc-\CLin-pelor} (in the city)
    \end{itemize}
        \item \textbf{Temporal (time)}
    \begin{itemize}
        \item \textbf{Prefix:} \textit{\CAtem-}
        \item \textbf{Example:} \textit{\CAtem-\CLab-loe} (at two o'clock)
    \end{itemize}
    \item \textbf{Instrumental (means or tool)}
    \begin{itemize}
        \item \textbf{Prefix:} \textit{\CAins-}
        \item \textbf{Example:} \textit{\CAins-\CLin-rith} (with the air)
    \end{itemize}
        \item \textbf{Purpose (goal or reason)}
    \begin{itemize}
        \item \textbf{Prefix:} \textit{\CApur-}
        \item \textbf{Example:} \textit{\CApur-\CLab-aray} (for the purpose of wisdom)
    \end{itemize}

    \item \textbf{Ablative (away)}
    \begin{itemize}
        \item \textbf{Prefix:} \textit{\CAabl-}
        \item \textbf{Example:} \textit{\CAabl-\CLab-loe} (before 2)
    \end{itemize}

     \item \textbf{Comlative (together)}
    \begin{itemize}
        \item \textbf{Prefix:} \textit{\CAcom-}
        \item \textbf{Example:} \textit{\CAcom-\CLan-ta} (together with him/her/them)

             \item \textbf{Allative (towards)}
    \begin{itemize}
        \item \textbf{Prefix:} \textit{\CAall-}
        \item \textbf{Example:} \textit{\CAall-\CLan-tal} (towards the east)
    \end{itemize}
        

     \item \textbf{Essive (time or state)}
    \begin{itemize}
        \item \textbf{Prefix:} \textit{\CAess-}
        \item \textbf{Example:} \textit{\CAess-\CLan-ninadriel} (as a child)
    \end{itemize}


    % \end{itemize}
       % \item \textbf{Excessive (too)}
    % \begin{itemize}
        % \item \textbf{Prefix:} \textit{\CAreq-}
        % \item \textbf{Example:} \textit{\CAreq-\CLab-ara} (requires wisdom)
    % \end{itemize}
           % \item \textbf{Relative (relative clause)}
    % \begin{itemize}
        % \item \textbf{Prefix:} \textit{\CAreq-}
        % \item \textbf{Example:} \textit{\CAreq-\CLab-ara} (requires wisdom)
    % \end{itemize}
\end{itemize}


\subsection*{Combining Classes and Cases}
To use these prefixes effectively, the structure would be Case + Class + Noun.
\begin{itemize}
    \item \textbf{Example:} \textit{\CAacc-\Clan-driel} (the animate person as the direct object)
\end{itemize}

\subsection*{Examples}
\begin{itemize}
    \item \textbf{The person (subject)}: \newline \textit{\CLan-driel} or \textit{\CAnom-\CLan-driel}
    \item \textbf{The person (direct object)}: \newline \textit{\CAacc-\CLan-driel}
    \item \textbf{Of the person (possession)}: \newline \textit{\CAgen-\CLan-driel}
    \item \textbf{To/for the person (indirect object)}:\newline  \textit{\CAdat-\CLan-driel}
    \item \textbf{In the city (location)}:\newline  \textit{\CAloc-\CLin-pelor}
    \item \textbf{With the air (instrumental)}:\newline  \textit{\CAins-\CLin-rith}
        \item \textbf{With the air (comlative)}:\newline  \textit{\CAcom-\CLin-rith}
\end{itemize}

}



\onecolumn

\section*{\centering Alphabet}

\medskip

\begin{tabular}{l r}
{\large
\begin{tblr}{|Q[c,m] Q[l,m] Q[c,m]|}
\hline
\textbf{Romanization} & \textbf{Pronunciation} & \textbf{Symbol} \\
\hline
A& {\huge /a/}&{\fontsize{60pt}{10pt}\tovian a} \\
\hline
B& {\huge /b/}&{\fontsize{60pt}{10pt}\tovian b}\vspace{-0pt}\\
\hline
D& {\huge /d/}&{\fontsize{60pt}{10pt}\tovian d} \\ % could be \textrtaild 
\hline
E& {\huge /e/}&{\fontsize{60pt}{10pt}\tovian e} \vspace{-30pt}\\
\hline

F& {\huge /f/}&{\fontsize{60pt}{10pt}\tovian f} \vspace{-45pt}\\
\hline
G& {\huge /g/}&{\fontsize{60pt}{10pt}\tovian g}\\
\hline
H & {\huge /h/ }&{\fontsize{60pt}{10pt}\tovian h}\\
\hline
I& {\huge /i/}&{\fontsize{60pt}{10pt}\tovian i}\\
\hline
K& {\huge /k/}&{\fontsize{60pt}{10pt}\tovian k}\vspace{-10pt}\\

\hline
LH & {\huge /\textbeltl/ }&{\fontsize{60pt}{10pt}\tovian lh}\\\hline % I really wanted this to be \textbeltl, but could also be textsubring{l} or 
L & {\huge /l/ }&{\fontsize{60pt}{10pt}\tovian l}\\
\hline
M & {\huge /m/ }&{\fontsize{60pt}{10pt}\tovian m}\vspace{-20pt}\\
\hline
N& {\huge /n/}&{\fontsize{60pt}{10pt}\tovian n}\vspace{-20pt}\\
\hline
O& {\huge /o/}&{\fontsize{60pt}{10pt}\tovian o} \vspace{-20pt}\\
\hline
\end{tblr}
}
&
{\large
\begin{tblr}{|Q[c,m] Q[c,m] Q[c,m]|}
\hline
\textbf{Romanization} &\textbf{Pronunciation} & \textbf{Symbol} \\
\hline
P& {\huge/p/}&{\fontsize{60pt}{10pt}\tovian p}\\
\hline
R& {\huge /r/}&{\fontsize{60pt}{10pt}\tovian r}\vspace{-35pt}\\
\hline
S& {\huge /s/}&{\fontsize{60pt}{10pt}\tovian s} \vspace{-30pt}\\
\hline
T& {\huge /\textsubbridge{t}/}&{\fontsize{60pt}{10pt}\tovian t}\\
\hline
U & {\huge /u/  }&{\fontsize{60pt}{10pt}\tovian u} \vspace{-20pt} \\
\hline
V& {\huge /v/}&{\fontsize{60pt}{10pt}\tovian v} \vspace{-25pt}\\
\hline
W& {\huge /\textturnw/}&{\fontsize{60pt}{10pt}\tovian w} \\
\hline
X& {\huge /ks/}&{\fontsize{60pt}{10pt}\tovian x}\vspace{-30pt}\\
\hline
Y & {\huge /j/ }&{\fontsize{60pt}{10pt}\tovian y} \vspace{-20pt}\\
\hline
\:EA& {\huge /e.a/ }&{\fontsize{60pt}{10pt}\tovian ea}\\
\hline
\:SS & {\huge /\textipa{s:}/ }&{\fontsize{60pt}{10pt}\tovian ss}\vspace{-20pt}\\
\hline
SH & {\huge /{\textesh}/ }&{\fontsize{60pt}{10pt}\tovian sh} \\
\hline
TH& {\huge /\texttheta/} %ð?
&  {\fontsize{60pt}{10pt}\tovian th} \\
\hline

\end{tblr}
}

\end{tabular}

\newpage

\subsection*{vowels}

\hspace{1pt}

{\huge
\begin{table}[h]
    \centering
    \begin{tblr}{Q[c,m] Q[c,m] Q[c,m]}
         Romanized Vowel & Pronunciation & Glyph\\
         \hline
         {\huge A}& {\huge /a/}&{\fontsize{40pt}{10pt}\tovian a} \\\hline
         {\huge A}& {\huge /\textsubumlaut{a}/}&{\fontsize{40pt}{10pt}\tovian ah} \\\hline
         {\huge E}& {\huge /e/}&{\fontsize{40pt}{10pt}\tovian e}\\
         \hline
         {\huge I}& {\huge /i/}&{\fontsize{40pt}{10pt}\tovian i}\\
         \hline
         {\huge O}& {\huge /o/}&{\fontsize{40pt}{10pt}\tovian o}\\
         \hline
         {\huge U} & {\huge /u/}&{\fontsize{40pt}{10pt}\tovian u}\\
         
    \end{tblr}
    \label{tab:my_label}
\end{table}
}

\medskip
\begin{figure}[h]
    \centering

\begin{tikzpicture}[vowel/.style={fill, circle, inner sep=0pt, text height=1.25ex}]
\coordinate (hf) at (0,3); % the high front vertex
\coordinate (hb) at (4,3); % the high back vertex
\coordinate (lb) at (4,0); % the low back vertex
\coordinate (lf) at (2,0); % the low front vertex

\draw (hf) -- (hb) -- (lb) -- (lf) -- cycle; % draws the trapezoid

\draw ($(hf)!1/3!(lf)$) -- ($(hb)!1/3!(lb)$); % the high-mid line
\draw ($(hf)!2/3!(lf)$) -- ($(hb)!2/3!(lb)$); % the low-mid line
\draw ($(hf)!0.5!(hb)$) -- ($(lf)!0.5!(lb)$); % the center line

\node[vowel,label={[label distance=-1pt]below right:{\fontsize{30pt}{0pt}\tovian i}}] at (0.6,2.8) {}; 
\node[vowel,label={[label distance=-3pt]below left:{\fontsize{30pt}{0pt}\tovian o}}] at (3.7,1.6) {};
\node[vowel,label={[label distance=-3pt]below left:{\fontsize{30pt}{0pt}\tovian u}}] at (3.6,2.8) {};
\node[vowel,label={[label distance=-3pt]below right:{\fontsize{30pt}{0pt}\tovian e}}] at (1.2,1.6) {};
\node[vowel,label={[label distance=-3pt]above left:{\fontsize{30pt}{0pt}\tovian a}}] at (3.25,0.2) {};
\end{tikzpicture}
\end{figure}



% example oradriel: 
    % \textipa{'ɔrad͡ʒriɛlɛ}
% hlumelore ʎ̤uːmɛloːɹeɪ

\twocolumn

\section*{Consonant and Vowel Inventory}
\begin{itemize}
    \item \textbf{Consonants}: \textipa{/b/, /d/, f, 
    % /\textphi/,
    /g/, /h/, /k/, /\textbeltl/, /l/, /m/, /\ng/, /n/, /p/, /r/, /s/, /\textsubbridge{t}/, /v/, /w/, /j/, /\textesh/, /\texttheta/}
    \item \textbf{Vowels}: \textipa{/a/, /\textsubumlaut{a}/, /e/, /i/, /o/, /u/}
    \item \textbf{Dipthongs}: \textipa{/ai/, /uj/}
\end{itemize} 


% \subsection*{Rules for /f/ and /\textphi/}

% \textbf{/f/ (Voiceless Labiodental Fricative)}:

% At the beginning of words:
% \begin{itemize}
% \item \textipa{/falor/} (leader)
% \item \textipa{/fethr/} (speaker)
% \end{itemize}
% Between vowels:
% \begin{itemize}
% \item /afa/ (...)
% \item /ujfae/ (...)
% \end{itemize}

% When adjacent to voiced consonants:
% \begin{itemize}
% \item ...
% \end{itemize}
% \textbf{/\textphi/ (Voiceless Bilabial Fricative)}:

% At the end of words or plurals:
% \begin{itemize}
% \item \textipa{/te\textphi/} (fortress)
% \item \textipa{/te\textphi e/} (fortresses)
% \end{itemize}


% In newer or borrowed words to indicate foreign or technical terms:
% \begin{itemize}
% \item ... (machine, from gnomish)
% \end{itemize}

% \subsection*{Rules for /l/ and /\textbeltl/}

% ...

\section*{Syllable Structure}
\begin{itemize}
    \item \textbf{Basic Structure}: Each syllable should follow the general structure of (C)(C)V(C)(C), where:
    \begin{itemize}
        \item \textbf{C} = Consonant
        \item \textbf{V} = Vowel
    \end{itemize}
    \item \textbf{Consonant Clusters}:
    \begin{itemize}
        \item At the beginning of a syllable, allow a maximum of two consonants.
        \item At the end of a syllable, allow a single consonant or specific consonant clusters.
    \end{itemize}
\end{itemize}

\section*{Syllable Patterns}
\redcolor{TODO: CONFIRM WITH DICTIONARY}
\begin{enumerate}
    \item \textbf{V (Vowel Only)}: Rare but can occur in specific cases, especially with prefix and suffix modifications.
    \begin{itemize}
        \item \textipa{/a/, /e/, /i/}
    \end{itemize}
    \item \textbf{CV (Consonant + Vowel)}: The most common syllable pattern.
    \begin{itemize}
        \item Example: \textipa{/la/, /me/, /ti/}
    \end{itemize}
    \item \textbf{CVC (Consonant + Vowel + Consonant)}: Also common, adding a consonant to the end.
    \begin{itemize}
        \item Example: \textipa{/lan/, /mujl/, /tir/}
    \end{itemize}
    \item \textbf{CCV (Consonant Cluster + Vowel)}: Allowed at the beginning of words.
    \begin{itemize}
        \item Example: \textipa{/tla/, /fre/, /tru/}
    \end{itemize}
     \item \textbf{CCVC (Consonant Cluster + Vowel + Consonant)}: Complex but permissible.
    \begin{itemize}
        \item Example: \textipa{/tl\textsubumlaut{a}n/, /frel/, /trus/}
    \end{itemize}
    \item \textbf{CVCC (Consonant + Vowel + Consonant Cluster)}: Possible only at the end of words or as whole words, and only including /\texttheta/ or /\textesh/.
    \begin{itemize}
        \item Example: \textipa{/fe\texttheta r/, /pa\texttheta r/, /lor\textesh/}
    \end{itemize}
\end{enumerate}

\section*{Stress Patterns}
\begin{enumerate}
    \item \textbf{Primary Stress}: On the penultimate syllable.
    \begin{itemize}
        \item Example: /la-\textbf{dri}-el/, /lo-\textbf{pe}-lor/
    \end{itemize}
    \item \textbf{Secondary Stress}: On the third to last syllable if the word has more than three.
    \begin{itemize}
        \item Example: /ti-\textbf{la}-dri-el/, /ti-\textbf{lo}-pe-lor/, /mel-or-\textbf{e}-ra-e/
    \end{itemize}
\end{enumerate}

\section*{Diphthongs and Vowel Clusters}
Clearly enunciate most vowel clusters to maintain the phonetic distinctiveness and breathy quality.
\begin{itemize}
\item \textbf{Diphthongs}: Diphthongs are combinations of two vowel sounds within the same syllable, gliding smoothly from one vowel to the other. The primary diphthongs in are:
\begin{itemize}
\item \textipa{/ai/}: Pronounced as in "eye."
\item \textipa{/uj/}: Pronounced as in "buoy."
\end{itemize}
\item \textbf{Vowel Clusters}: When two vowels appear together but do not form a diphthong, they must be clearly enunciated to distinguish them as separate sounds.
\begin{itemize}
\item Example: \textipa{/ea/} similar to "idea", where both vowels are pronounced distinctly.
\end{itemize}
\end{itemize}

\section*{Examples}
\begin{description}
        \item /la-\textbf{dri}-el ya-\textbf{lo}-pe-lor \textbf{lo}-re \textbf{em}-ril \textbf{si}-ne\textesh/ 
        \hanging The person quickly sees the city.
\end{description}





Lefethr is an \textit{agglutanative} language, meaning morphemes do not change form


\section*{Plurals}
\textit{-e} indicates plurality. Verbs do not agree with number. Singular nouns therefore never end in \textit{-e}.

\begin{center}
\begin{tabular}{|c c|}
     \hline

    \textit{lothr} &  river \\
     \textit{lothre} & rivers \\

     \hline
     \hline

     \textit{nor} & sea \\

     \textit{nore} & seas \\
     \hline
     
\end{tabular}
\end{center}

\section*{Verbs}

Verbs are constructed around a consonant root that denotes its aspect. This root is combined with a suffix to indicate tense. The tense suffixes are \textit{-e} for past,\textit{ -a} for present, and \textit{-o} for future. Optional prefixes can be added to denote mood and voice. This final conjugated auxiliary verb is placed in front of the infinitive. The verb that indicates the action is never conjugated, only the auxiliary verb changes form to show aspect, tense, mood, and voice.

\begin{table}[H]
    \centering
    \begin{tabular}{|c|c|}
        \hline
        \textbf{Tense} & \textbf{Auxiliary Verb Suffix} \\
        \hline
        Past & -e \\
        Present & -a \\
        Future & -o \\
        \hline
    \end{tabular}
    % \caption{Person Suffixes in Tovian}
    \label{tab:verb-tense}
\end{table}


For example, to say "the person is speaking", start with the verb ``to speak": \textit{fethr}. Use the continuous aspect consonant\textit{sh} (from the verb \textit{shotl}, ``to flow") to form the auxiliary verb \textit{sha}. The auxilary verb goes before the infinitive, resulting in \textit{sha fethr}, ``is speaking". \textit{Driel} is ``the person", resulting in the final sentence: \textit{driel sha fethr}, ``the person is speaking". A very literal translation could be ``The person flows speaking".

\begin{table}[h!]
    \centering
    \subsubsection{Aspects with present tense examples}
    \begin{tabular}{c|c|c|c}
    \hline
         \textbf{Aspect}& \makecell{\textbf{Root}\\\textbf{Verb}}& \makecell{\textbf{Aux.}\\\textbf{root}} & \makecell{\textbf{Pres. tense}\\\textbf{Example}} \\
\hline
         Simple&\makecell{\textit{fethr} \\ speak}  & f- & \makecell{\textit{fa mrila}\\``sees"} \\ \hline
         \red{Imperfective}&\makecell{\textit{thraf} \\ wonder}  & th- & \makecell{\textit{tha mrila}\\ ``is seeing"} \\ \hline
         Perfect&\makecell{\textit{lhara} \\ know}  & lh- & \makecell{\textit{lha mrila} \\ ``has seen"} \\ \hline
         Near&\makecell{\textit{tlun} \\ share}& tl- & \makecell{\textit{tla mrila} \\ ``is about to see" }\\ \hline
         Immediate&\makecell{\textit{kesh} \\ burn}& k- & \makecell{\textit{ka mrila} \\ ``is seeing right now" }\\ \hline
         Habitual&\makecell{\textit{mel} \\ live}  & m- & \makecell{\textit{ma mrila} \\``sees regularly" }\\ \hline
         Progressive&\makecell{\textit{sethr} \\ run}& s- & \makecell{\textit{sa mrila} \\``is seeing now" }\\ \hline
         Continuous&\makecell{\textit{shotl} \\ flow}& sh- & \makecell{\textit{sha mrila} \\ ``is seeing" }\\ \hline
         Iterative&\makecell{\textit{nalh} \\ think}& n- & \makecell{\textit{na mrila} \\ ``sees repeatedly" }\\ \hline
         Inceptive&\makecell{\textit{yalor} \\ begin}& y- & \makecell{\textit{ya mrila} \\ ``begins  seeing" }\\ \hline
         Cessative&\makecell{\textit{panor} \\ end}& p- & \makecell{\textit{pa mrila} \\ ``stops seeing" }\\ \hline
         Remote&\makecell{\textit{huysh} \\ throw}  & h- & \makecell{\textit{ha mrila} \\``sees (distant)" \footnotemark[1]} \\
     \end{tabular}
     \\
     \medskip
     \raggedright
     {\footnotesize \footnotemark[1] \raggedleft In past and future tenses, the remote aspect refers to actions that took place long ago or far in the future, but the meaning in present tense is more nuanced. Elves tend to use the rare present remote aspect as an insult, for example \textit{wele ha fethr}: ``they speak but it is irrelevant"}
    \label{tab:aspects}
\end{table}

\medskip


\subsection*{Mood and Voice}

% \subsection*{All verb forms}
\begin{tabular}{rl}
&\textbf{Mood}\\
\textbf{unmarked}& $\rightarrow$ indicative \\
\textbf{wi-}& $\rightarrow$ subjunctive \\
\textbf{se-}& $\rightarrow$ imperative \\
\textbf{ko-} & $\rightarrow$ conditional \\
\textbf{xo-}& $\rightarrow$ counterfactual \\
\textbf{hle-}& $\rightarrow$ optative \\
\textbf{tlo-}& $\rightarrow$ obligative \\
\textbf{ef-}& $\rightarrow$ necessative \\
\textbf{sho-}& $\rightarrow$ potential \\



&\textbf{Voices}\\
\textbf{unmarked}& $\rightarrow$ active \\
\textbf{te-}& $\rightarrow$ reflexive \\
\textbf{pa-}& $\rightarrow$ passive \\
\textbf{mo-}& $\rightarrow$ middle \\
\textbf{ke-}& $\rightarrow$ causative \\
\textbf{ra-}& $\rightarrow$ reciprocal \\

% &\textbf{Other (not combined)}\\
% \textbf{li-}& $\rightarrow$ gerund \\

\end{tabular}

\medskip

Mood and voice can be combined by adding the mood prefix before the voice prefix. For example, adding the obligative mood marker \textit{tlo-} and the reflexive voice marker \textit{te-} to the habitual aspect auxilary root stem \textit{m-} with the present tense conjugation \textit{-a}, the final auxilary verb is \textit{tlotema}. 

\ex
\begingl
\gla idriel tlotema lorsel //
\glb NOM-person OBLIG-REFL-HAB.PRES wash-INF //
\glft "She should habitually wash herself." //
\endgl
\xe



\subsection{Unused Combinations}
Tovian is very flexible when it comes to combining different tenses, aspects, moods, and voices. However, certain verb conjugations result in combinations that are unlikely or never used due to their impractical or nonsensical nature. For example, the Past Imperative is very rare because commands most often refer to actions in the present or future.



\section{Copulas}
Copulas are formed by combining verbs and nouns to describe states and qualities. For example the word "mel" (life/living) can be used as a copula: "i-la-driel fa mela" means "The person is alive," or equally "The person lives" This method applies to various attributes by changing the verb root and noun. For instance, "driel fe araya" translates to "The person was wise." 

 \section{Comparative Constructions}

    Tovian uses comparative constructions to express excessiveness, for example: "bigger than appropriate." The ablative case is often used for this comparison: "the tree is too big" could be expressed as \textit{ilo-glar \CAabl-le-velar fa mota} ("The tree is big away-from-worthiness" or simply "The tree is bigger than what is worthy/right").

\section{Adjectives}

Tovian does not use a separate class of adjectives. Instead, all descriptive qualities are expressed through nouns. These attributive nouns follow the noun they modify in a fixed noun–noun structure. For example, the phrase \textit{driel aray} translates literally as “person wisdom,” conveying the meaning “the wise person.” To convey "people wisdom" or "the wisdom of people," one would switch the order to: \textit{aray driel}. This structure does not need inflection or agreement between adjectives and nouns, for example "the wise people" would simply be \textit{driele aray}. In cases requiring emphasis or clarification, a copular construction can be used instead: \textit{i-la-driel fa araya} (“The person is wise”). Many commonly used descriptive terms—such as colors, emotional states, or physical traits—exist as standalone nouns and follow the same syntactic pattern. This noun-only system reinforces the conceptual unity of properties and entities, treating attributes not as modifiers but as qualities with independent substance.


\section{Agreement}

Suffixes are added to verbs to indicate the person of the subject.

\begin{table}[H]
    \centering
    \begin{tabular}{|c|c|}
        \hline
        \textbf{Person} & \textbf{Suffix} \\
        \hline
        1st person & i \\
        2nd person & o \\
        3rd person & a \\
        \hline
    \end{tabular}
    % \caption{Person Suffixes in Tovian}
    \label{tab:suffixes}
\end{table}


\begin{table}[H]
    \centering
    \begin{tabular}{|c|c|c|}
        \hline
        \textbf{Phrase} & \textbf{Translation} & \textbf{Person} \\
        \hline
        ilane sa nalh\textbf{i} & "We think" & 1st \\
        ilana sa nalh\textbf{i} & "I think" & 1st \\
        ilawa sa nalh\textbf{o} & "You think" & 2nd \\
        ilawe sa nalh\textbf{o} & "You (plural) think" & 2nd \\
        ilata sa nalh\textbf{a} & "He/she/they think" & 3rd \\
        ilate sa nalh\textbf{a} & "They (plural) think" & 3rd \\
        \hline
    \end{tabular}
    % \caption{Agreement in Tovian}
    \label{tab:agreement}
\end{table}

%  Verb stems are added to nouns and always end in ``lore". Different tenses, aspects, moods, and voices can be applied by adding prefixes. The prefixes in the order they should be applied are as follows:\footnote{\noindent Complex verb forms are saved for formal communication.}

% \subsection{Simple Verb Forms}

% \begin{tabular}{rl}
% &\textbf{Tense}\\
% \textbf{} &$\rightarrow$$ present tense \\
% \textbf{vi-} &$\rightarrow$$ future tense \\
% \textbf{in-} &$\rightarrow$$ past tense \\

% &\textbf{Aspect}\\
% \textbf{de-}& $\rightarrow$ perfect \\
% \textbf{me-} & $\rightarrow$ participle \\
% \textbf{shi-}& $\rightarrow$ continuous \\

% &\textbf{Mood}\\
% \textbf{ko-} & $\rightarrow$ conditional \\
% \textbf{se-}& $\rightarrow$ imperative \\
% \textbf{wi-}& $\rightarrow$ subjunctive \\

% &\textbf{Voices}\\
% \textbf{te-}& $\rightarrow$ reflexive \\
% \textbf{pe-}& $\rightarrow$ passive \\

% &\textbf{Other (not combined)}\\
% \textbf{li-}& $\rightarrow$ gerund \\


% \end{tabular}
% % 55 combinations


% \subsection*{All verb forms}
% \begin{tabular}{rl}
% &\textbf{Tense}\\
% \textbf{} &$\rightarrow$$ present tense \\
% \textbf{vi-} &$\rightarrow$$ future tense \\
% \textbf{in-} &$\rightarrow$$ past tense \\
% &\textbf{Aspect}\\
% \textbf{de-}& $\rightarrow$ perfect \\
% \textbf{me-} & $\rightarrow$ participle \\
% \textbf{re-} & $\rightarrow$ habitual \\
% \textbf{ro-}& $\rightarrow$ iterative \\
% \textbf{ve-} & $\rightarrow$ immediate \\
% \textbf{shi-}& $\rightarrow$ continuous \\
% \textbf{wo-}& $\rightarrow$ remote \\
% \textbf{to-}& $\rightarrow$ near \\
% \textbf{yi-}& $\rightarrow$ inceptive \\
% \textbf{hlo-}& $\rightarrow$ cessative \\

% &\textbf{Mood}\\
% \textbf{ko-} & $\rightarrow$ conditional \\
% \textbf{se-}& $\rightarrow$ imperative \\
% \textbf{wi-}& $\rightarrow$ subjunctive \\
% \textbf{xo-}& $\rightarrow$ counterfactual \\

% &\textbf{Voices}\\
% \textbf{te-}& $\rightarrow$ reflexive \\
% \textbf{pa-}& $\rightarrow$ passive \\
% \textbf{mo-}& $\rightarrow$ middle \\
% \textbf{ke-}& $\rightarrow$ causative \\
% \textbf{ra-}& $\rightarrow$ reciprocal \\

% &\textbf{Other (not combined)}\\
% \textbf{li-}& $\rightarrow$ gerund \\


% \end{tabular}
% % 601 combinations

% \medskip

% \subsection*{Verb Conjugations}
% Using "speak" as an example: \textit{fethr}

% \subsubsection*{Tense}
% \begin{itemize}
%     \item \textbf{Present Tense:} \textit{} \\
%           \hanging Speak: \textit{lore-fethr}
          
%     \item \textbf{Future Tense:} \textit{vi-} \\
%           \hanging Will speak: \textit{vilore-fethr}
          
%     \item \textbf{Past Tense:} \textit{in-} \\
%           \hanging Spoke: \textit{inlore-fethr}
% \end{itemize}

% \subsubsection*{Aspect}
% \begin{itemize}
%     \item \textbf{Perfect:} \textit{de-} \\
%           \hanging Has spoken: \textit{delore-fethr}
          
%     \item \textbf{Continuous:} \textit{shi-} \\
%           \hanging Is speaking: \textit{shilore-fethr}
          
%     \item \textbf{Habitual:} \textit{re-} \\
%           \hanging Speaks regularly: \textit{relore-fethr}
          
%     \item \textbf{Immediate:} \textit{ve-} \\
%           \hanging Is about to speak: \textit{velore-fethr}
          
%     \item \textbf{Participle:} \textit{ma-} \\
%           \hanging Speaking (as a participle): \textit{malore-fethr}
% \end{itemize}

% \subsubsection*{Mood}
% \begin{itemize}
%     \item \textbf{Conditional:} \textit{ko-} \\
%           \hanging Would speak: \textit{kolore-fethr}
          
%     \item \textbf{Imperative:} \textit{se-} \\
%           \hanging Speak!: \textit{selore-fethr}
          
%     \item \textbf{Subjunctive:} \textit{wi-} \\
%           \hanging May speak: \textit{wilore-fethr}
          
%     \item \textbf{Counterfactual:} \textit{xo-} \\
%           \hanging Would have spoken: \textit{xolore-fethr}
% \end{itemize}

% \subsubsection*{Voice}
% \begin{itemize}
%     \item \textbf{Reflexive:} \textit{te-} \\
%           \hanging Speaks to oneself: \textit{telore-fethr}
          
%     \item \textbf{Passive:} \textit{pa-} \\
%           \hanging Is spoken: \textit{palore-fethr}
          
%     \item \textbf{Middle:} \textit{mo-} \\
%           \hanging Speaks for oneself: \textit{molore-fethr}
          
%     \item \textbf{Causative:} \textit{ke-} \\
%           \hanging Makes someone speak: \textit{kelore-fethr}
          
%     \item \textbf{Reciprocal:} \textit{ra-} \\
%           \hanging They speak to each other: \textit{ralore-fethr}
% \end{itemize}

% \subsubsection*{Other}
% \begin{itemize}
%     \item \textbf{Gerund:} \textit{li-} \\
%           \hanging Speaking (as a noun): \textit{lilore-fethr}
% \end{itemize}

% \subsubsection*{Combined Forms}
% \begin{itemize}
%     \item \textbf{Past Perfect:} \textit{de- + in-} \\
%           \hanging Had spoken: \textit{deinlore-fethr}
          
%     \item \textbf{Future Continuous:} \textit{shi- + vi-} \\
%           \hanging Will be speaking: \textit{shivilore-fethr}
          
%     \item \textbf{Conditional Habitual:} \textit{re- + ko-} \\
%           \hanging Would speak regularly: \textit{rekolore-fethr}
          
%     \item \textbf{Immediate Imperative:} \textit{ve- + se-} \\
%           \hanging Speak immediately!: \textit{veselore-fethr}
          
%     \item \textbf{Subjunctive Participle:} \textit{wi- + ma-} \\
%           \hanging Speaking in a hypothetical or wished-for sense: \textit{wimalore-fethr}
% \end{itemize}


% \subsection*{To be and to be like}
% All verbs with the exception of two follow the above structure. To express that the subject of the sentence is equivalent to the object, use ``'as" as a postfix on the object. To express similarity, use ``'asl" instead. For example:

% \begin{itemize}
%   \item \textbf{To be:} \\
%         \textit{Tel \CAacc-\CLin-helor'as.} (The lake is glass.) \\
%         - \textit{Tel} (subject: the lake) \\
%         - \textit{\CAacc-\CLin-helor'as} (object: glass with ``as" indicating "is")

%  \item \textbf{To be like:} \\
%         \textit{Tel \CAacc-\CLin-helor'asl.} (The lake is like glass.) \\
%         - \textit{Tel} (subject: the lake) \\
%         - \textit{helor'as} (object: glass with ``asl'" indicating "is like")

% \end{itemize}





\section*{Definiteness, Classes, and Cases}

Speakers usually on context to convey definiteness or indefiniteness.
In situations where there is ambiguity or in very formal communication, there are different prefixes:

\begin{table}[H]
    \centering
    \begin{tabular}{ccc}
         &  Subject& Non-subject\\
         Definite&  \textbf{i}-& \textbf{a}-\\
         Indefinite&  \textbf{e}-& \textbf{o}-\\
    \end{tabular}
\end{table}

{
\hspace{-30pt}
\begin{tabular}{l l}
\hline
    The sun shines in the sky  &   \textit{\textbf{i}-\CLin-efol fa ana \textbf{a}\CAloc-rithil}\\
    A sun shines in the sky   &  \textit{\textbf{e}-\CLin-efol fa ana \textbf{a}\CAloc-rithil}\\

        The sun shines in a sky  &   \textit{\textbf{i}-\CLin-efol fa ana \textbf{o}\CAloc-rithil}\\
    A sun shines in a sky   &  \textit{\textbf{e}-\CLin-efol fa ana \textbf{o}\CAloc-rithil}\\
\hline
     
\end{tabular}
}



\medskip

Nouns are categorized into three distinct \textbf{classes}, each identified by a unique prefix: \textit{\CLan-} for animate beings, \textit{\CLin-} for inanimate objects and places, and \textit{\CLab-} for abstract concepts and ideas. Nouns can also be inflected for grammatical \textbf{cases} to indicate their role within a sentence. Classes are not always used when the meaning is obvious, for example ``the person sees" would be written formally as \textit{\CLan-driel fa mrila}, but in informal communication, the class might be dropped to create \textit{driel fa mrila}, since the subject is obviously not a dead person (\textit{\CLin-driel}) or the abstract concept of people (\textit{\CLab-drie}l). When both cases and classes are required, cases are added before classes. For example \textit{\CLin-nume \textbf{\CApur}-\textbf{\CLab}-aray }is ``the paths for wisdom", where \textbf{\CApur} represents ``for", \textbf{\CLab} represents ``abstract concept of", and \textbf{ara} is ``wisdom." In formal or educational contexts, hyphens (romanization: ``-", Tovian: ``\vc{\fontsize{50pt}{-200pt}\tovian\textasciitilde-\textasciitilde}") are used to separate prefixes from their nouns. For example, in this document, a sentence could be written as \textit{\CAnom-\CLan-driel \CAacc-\CLin-tel fa mrila} (the person sees the lake), but in practice one might use \textit{driel \CAacc\CLin tel fa mrila}, dropping the hyphens, the nomative case, and the animate class.



\casestable

\subsection*{Pronouns}
\begin{tabular}{rl}
\textbf{na} & $\rightarrow$ me, I \\
\textbf{hina} & $\rightarrow$ my, mine \\
\textbf{ne} & $\rightarrow$ us or we \\
\textbf{hine} & $\rightarrow$ ours \\
\textbf{wa} & $\rightarrow$ you or your \\
\textbf{hiwa} & $\rightarrow$ yours \\
\textbf{we} & $\rightarrow$ you or (plural) \\
\textbf{hiwe} & $\rightarrow$ yours (plural) \\
\textbf{lata} & $\rightarrow$ them/she/he (animate)\\
\textbf{lahita} & $\rightarrow$ theirs/her/his \\
\textbf{late} & $\rightarrow$ them (plural, animate) \\
\textbf{hahite} & $\rightarrow$ theirs (plural) \\

\textbf{loda} & $\rightarrow$ it (inanimate)\\
\textbf{lohita} & $\rightarrow$ theirs \\
\textbf{lode} & $\rightarrow$ them (inanimate) \\
\textbf{lohite} & $\rightarrow$ theirs (plural, inanimate) \\

\end{tabular}



\subsection*{Word Order}
Sentences historically used a Subject-Object-Verb (SOV) word order. In this structure, the subject of the sentence comes first, followed by the object, and then the verb. For instance, "The person lives in the city" would be structured as "\CLan-driel \CAloc-\CLin-pelor fa mela", where "\CLan-driel" (the person) is the subject, "\CAloc-\CLin-pelor" (the city) is the direct (locative) object, and "fa mela" (lives) is the verb.


For emphasis, modern Tovian allows flexibility in word order. Elements can be fronted to the beginning of the sentence to add emphasis or focus. For example, to emphasize the city being lived in by the person, one might say "\CAloc-\CLin-pelor \CLan-driel fa mela," placing the object "\CAloc-\CLin-pelor" (the city) at the beginning of the sentence. This doesn't introduce ambiguity only because of the locative marker (\CAloc-) on the object of the sentence, which indicates it is never the subject despite coming first in word-order.

When changing word order, it is important to leave adjectives (remember, they are unmarked nouns) directly after the noun they modify. For example, "the person lives in the big city" would be either "\CLan-driel \CAloc-\CLin-pelor mot fa mela" or "\CAloc-\CLin-pelor mot \CLan-driel fa mela." Separating "mot" (big) and "pelor" (city) changes the meaning of the sentence. For example: "\CAloc-\CLin-pelor \CLan-driel mot fa mela" would be "the big person lives in the city." 


\subsection*{Questions}
\subsubsection{Yes/No questions}
To turn a declarative statement into a yes or no question, \textit{lhan} is added to the front of the sentence.

\textbf{Example: } \textit{lhan ila-we yalo-ta fa mrila?}

``Do you see it?"

\textbf{Response:} \textit{ila-na yalo-ta fa mrila}

``I see it"

In response to a question, declarative statements are sometimes simplified to just the conjugated auxilary + infinative, for example \textit{na yalo-ta fa mrila} could be shortened to \textit{fa mrila}, literally: ``sees", but the subject and object are inferred to be the same as the question since they are left out of the response. 

\medskip

\subsubsection{WH- questions}
 To form a non-yes/no question, add the question marker \textit{lhan} to the beginning of the sentence with a case-marking prefix to specify what question is being asked. 
 Examples:
     \begin{tabular}{c c}
        \textbf{ti}-lhan ta fe fethra & ``Where did he speak" \\
        \textbf{to}-lhan ta fe fethra & ``when did he speak?" \\
        \textbf{si}-lhan ta fe fethra & ``how did he speak?" \\
        \textbf{ya}-lhan ta fe fethra & ``to what/whom did he speak?" \\
          & 
     \end{tabular}

\subsubsection{Example conversation}

\noindent A: \textit{lhan driel se kwilora tolon?} % was the person dancing yesterday? 

\noindent B: \textit{ilata se nikwilora tolon, shan ilata lho kwilora ditom hitirlhon. lhan ilawa yata fo mrilo?} % she was not dancing yesterday, but she will have danced by the end of the week. Will you see her?

\noindent A: \textit{na yata fo mrili, tolhan ta fo kwilora?} % I am going to see her, what time will she dance?

\noindent B: \textit{tolerep tohethn} % At five o'clock tomorrow

{\tovian
\noindent lhan driel se kwilora tolon

\noindent ilata se nikwilora tolon, shan ilata lho kwilora ditom hitirlhon. lhan ilawa yata fo mrilo

\noindent na yata fo mrili, tolhan ta fo kwilora

\noindent torep tohethn
}


\ex
\begingl
\gla lhan driel se kwilora to-lon? //
\glb Q person PROG.PAST dance.1SG LOC-yesterday // 
\glft "Was the person dancing yesterday?" //
\endgl
\xe

\ex
\begingl
\gla ilata se nikwilora tolon, shan ilata lho kwilora ditom hitirlhon. lhan ilawa yata fo mrilo? //
\glb SBJ.ANIM.3SG PROG.PAST NEG-dance.1SG LOC-yesterday, but 3SG PERF dance.1SG ABL-end GEN-week. Q 2SG ACC-3SG FUT see.2SG? //
\glft "She was not dancing yesterday, but she will have danced by the end of the week. Will you see her?" //
\endgl
\xe

\ex
\begingl
\gla ila-na ya-ta fo mrili, to-lhan ta fo kwilora? //
\glb 1SG ACC-3SG FUT see.1SG, when-Q 3SG FUT dance.1SG? //
\glft "I will see her, what time will she dance?" //
\endgl
\xe
Sentence D

\ex
\begingl
\gla tole-rep to-hethn//
\glb LOC-five LOC-tomorrow//
\glft "At five o'clock tomorrow." //
\endgl
\xe


\newpage


% \section{Conjunctions}

% \subsection{Coordinating}
% \begin{itemize}
%     \item{\makebox[2cm][l]{And}  \_\_\_\_\_\_\_\_}
%     \item{\makebox[2cm][l]{But}  \_\_\_\_\_\_\_\_\_}
%     \item{\makebox[2cm][l]{So} \_\_\_\_\_\_\_\_}
% \end{itemize}

% \subsection{Subordinating}
% \begin{itemize}
%     \item{\makebox[2cm][l]{Because} \_\_\_\_\_\_\_\_}
%     \item{\makebox[2cm][l]{If} \_\_\_\_\_\_\_\_}
%     \item{\makebox[2cm][l]{Although} \_\_\_\_\_\_\_\_}
%     \item{\makebox[2cm][l]{When} \_\_\_\_\_\_\_\_}
%     \item{\makebox[2cm][l]{That} \_\_\_\_\_\_\_\_}
% \end{itemize}

% \subsection{Correlative}
% \begin{itemize}
%     \item{\makebox[2cm][l]{Either ... or} \_\_\_\_\_\_\_\_}
%     \item{\makebox[2cm][l]{Neither ... nor} \_\_\_\_\_\_\_\_}
% \end{itemize}



% 
% \section*{Vocabulary of Tovian}
% {\large \red{NEEDS TO BE UPDATED WITH NEW DICTIONARY}}


% \subsection*{Prefixes and Suffixes}
% \begin{tabular}{rl}
% \textbf{a-} & $\rightarrow$ of \\
% \textbf{y-} & $\rightarrow$ from \\
% \textbf{u-} & $\rightarrow$ with \\
% \textbf{hlu-} & $\rightarrow$ for \\
% \textbf{sho-} & $\rightarrow$ can (prefix) \\
% \textbf{ef-} & $\rightarrow$ requires (prefix) \\
% \textbf{ea-} & $\rightarrow$ noun to verb (prefix) \\
% \textbf{e} & $\rightarrow$ plural (postfix) \\
% \textbf{ni'} & $\rightarrow$ not/without (prefix) \\
% \textbf{nit'} & $\rightarrow$ opposite (prefix) \\
% \textbf{ana'} & $\rightarrow$ that which (prefix) \\
% \textbf{na'} & $\rightarrow$ which (prefix) \\
% \textbf{'as} & $\rightarrow$ to be (postfix to pronoun/noun) \\
% \textbf{'als} & $\rightarrow$ to be like (postfix to pronoun/noun) \\
% \textbf{thil} & $\rightarrow$ place or realm (postfix) \\
% \textbf{'nin} & $\rightarrow$ diminutive (postfix) \\
% \textbf{lum} & $\rightarrow$ good/desirable (postfix) \\
% \textbf{mah'} & $\rightarrow$ through (prefix).

% \end{tabular}

% \subsection*{Conjunctions, Prepositions, and Adverbs}
% \begin{tabular}{rl}
% \textbf{sa} & $\rightarrow$ but/however \\
% \textbf{tes} & $\rightarrow$ in order to \\
% \textbf{el} & $\rightarrow$ in \\
% \textbf{ta} & $\rightarrow$ while \\
% \textbf{sor} & $\rightarrow$ near/by \\
% \textbf{heth} & $\rightarrow$ ahead/front \\
% \textbf{hyul} & $\rightarrow$ behind/back \\
% \textbf{mah} & $\rightarrow$ through \\
% \textbf{wa} & $\rightarrow$ or \\
% \end{tabular}


% \subsection*{Natural Elements and Geography}
% \begin{tabular}{rl}
% \textbf{erathil} & $\rightarrow$ desert \\
% \end{tabular}

% \subsection*{Societal and Cultural Concepts}
% \begin{tabular}{rl}
% \textbf{nalhtlun} & $\rightarrow$ council \\
% \textbf{pelor} & $\rightarrow$ city \\
% \textbf{driel} & $\rightarrow$ person \\
% \textbf{ni'driel} & $\rightarrow$ non-person \\
% \textbf{nan} & $\rightarrow$ ancient \\
% \textbf{ni'nan} & $\rightarrow$ youth (not ancient) \\
% \textbf{ni'nan'nin} & $\rightarrow$ child (not ancient + diminutive) \\
% \end{tabular}

% \subsection*{Conceptual and Abstract Terms}
% \begin{tabular}{rl}
% \textbf{ara} & $\rightarrow$ wisdom \\
% \textbf{cana} & $\rightarrow$ change \\
% \textbf{thivm} & $\rightarrow$ important \\
% \textbf{haru} & $\rightarrow$ truth \\
% \textbf{kar} & $\rightarrow$ balance \\
% \textbf{bral} & $\rightarrow$ darkness \\
% \textbf{ahn} & $\rightarrow$ light \\
% \textbf{velar} & $\rightarrow$ worthy \\
% \end{tabular}

% \subsection*{Compound Words and Phrases}
% \begin{tabular}{r|l}
% \textbf{tkeshsul} & "under foot" \\
%  & = floor/ground \\
% \textbf{ana'leth} & "that which tastes" \\
%  & = food \\
% \textbf{ana'leth adrix} & meat \\
%  & = that which tastes of blood \\
% \textbf{shoiss} & bird \\
%  & = can fly \\
% \textbf{shoiss'mah'lothre} & river fish \\
%  & = can fly through rivers \\
% \textbf{shoiss'mah'nor} & sea fish \\
%  & = can fly through sea \\
% \textbf{leth'haru'sel} & true taste of water \\
% \textbf{marth'sul'melore} & forest air \\
%  & = clean air underneath \\
% \textbf{ruy'nin} & small feather \\
%  & = term of endearment \\
% \textbf{velar'vehkuilor} & worthy death by combat \\
% \textbf{num'hlu'ara} & path for wisdom \\
% \textbf{alu'sel'sir} & dread \\
% & = Deep water above\\
% \textbf{sura'lothr} & daily life \\
%  & = The routine river \\
% \textbf{yul'silytel} & the song created from the Lake \\
% \textbf{tkeshsil} & standing \\
%  & = over foot \\
% \textbf{kharànane} & old ways \\
%  & = balance of ancients \\
% \textbf{beth'har} & pain \\
%  & = feels alert \\
% \textbf{beth'asc} & fire \\
%  & = feels hot \\
% \textbf{eätul'har} & interesting \\
%  & = makes alert \\
% \textbf{heth'nume} & future \\
%  & = paths ahead \\
% \textbf{hyul'nume} & past \\
%  & = paths behind \\
% \textbf{rith'driel} & breath \\
%  & = person air \\
% \textbf{rith'driel'yul} & whistle \\
%  & = person air song \\
% \end{tabular}





% \subsection*{Numbers}
% \begin{tabular}{rl}
% \textbf{lo} & $\rightarrow$ one \\
% \textbf{loe} & $\rightarrow$ two \\
% \textbf{fem} & $\rightarrow$ three \\
% \textbf{tev} & $\rightarrow$ four \\
% \textbf{rep} & $\rightarrow$ five \\
% \textbf{tep} & $\rightarrow$ six \\
% \textbf{pol} & $\rightarrow$ seven \\
% \textbf{ath} & $\rightarrow$ eight \\
% \textbf{ki} & $\rightarrow$ nine \\
% \textbf{tir} & $\rightarrow$ ten \\
% \textbf{bey} & $\rightarrow$ eleven \\
% \textbf{hliya} & $\rightarrow$ twelve \\
% \textbf{hliya-lo} & $\rightarrow$ thirteen \\
% \textbf{hliya-loe} & $\rightarrow$ fourteen \\
% \textbf{hliya-fem} & $\rightarrow$ fifteen \\
% \textbf{hliya-ath} & $\rightarrow$ twenty \\
% \textbf{loe-hliyae} & $\rightarrow$ twenty four \\
% \textbf{loe-hliyae-lo} & $\rightarrow$ twenty five \\
% \textbf{loe-hliya-loe} & $\rightarrow$ twenty six \\
% \textbf{fem-hliyae} & $\rightarrow$ thirty six \\
% \textbf{tev-hliyae} & $\rightarrow$ forty eight \\
% \textbf{hliya-hliyae} & $\rightarrow$ one hundred forty four\\
% \textbf{emiyr} & $\rightarrow$ one thousand and eight\\


% \end{tabular}

% \newpage

% \subsection*{Root Words}
% \begin{tabular}{r|l}
% % \textbf{trom} & $\rightarrow$ no \\
% % \textbf{te} & $\rightarrow$ yes \\
% \textbf{lor} & $\rightarrow$ do\\
% \textbf{or} & $\rightarrow$ group\\
% \textbf{driel} & $\rightarrow$ person\\
% \textbf{lem} & $\rightarrow$ sun\\
% \textbf{elan} & $\rightarrow$ hand\\
% \textbf{bral} & $\rightarrow$ darkness \\
% \textbf{ahn} & $\rightarrow$ light \\
% \textbf{sor} & $\rightarrow$ closeness \\
% \textbf{heth} & $\rightarrow$ front \\
% \textbf{hul} & $\rightarrow$ back \\
% \textbf{melor} & $\rightarrow$ plant\\
% \textbf{glar} & $\rightarrow$ forest \\
% \textbf{era} & $\rightarrow$ sand \\
% \textbf{lothr} & $\rightarrow$ river \\
% \textbf{nor} & $\rightarrow$ sea \\
% \textbf{sel} & $\rightarrow$ water \\
% \textbf{rith} & $\rightarrow$ air \\
% \textbf{sul} & $\rightarrow$ earth \\
% \textbf{ask} & $\rightarrow$ hot \\
% \textbf{tkshu} & $\rightarrow$ cold \\
% \textbf{pathr} & $\rightarrow$ weight\\
% \textbf{har} & $\rightarrow$ pain\\
% \textbf{???} & $\rightarrow$ size\\

% \textbf{green} & $\rightarrow$ \\
% \textbf{red} & $\rightarrow$ \\
% \textbf{blue} & $\rightarrow$dora\\
% \textbf{???} & $\rightarrow$ \\
% \textbf{???} & $\rightarrow$ \\


% \textbf{kana} & $\rightarrow$ change \\
% \textbf{thul} & $\rightarrow$ top \\
% \textbf{sul} & $\rightarrow$ bottom \\
% \textbf{het} & $\rightarrow$ left \\
% \textbf{lhoy} & $\rightarrow$ right \\
% \textbf{lhel} & $\rightarrow$ center\\
% \textbf{lae} & $\rightarrow$ north \\
% \textbf{tal} & $\rightarrow$ east \\
% \textbf{ria} & $\rightarrow$ south \\
% \textbf{fil} & $\rightarrow$ west \\

% \textbf{ara} & $\rightarrow$ wisdom \\
% \textbf{velar} & $\rightarrow$ worth \\
% \textbf{thivem} & $\rightarrow$ importance \\
% \textbf{haru} & $\rightarrow$ correct/truth \\
% \textbf{kar} & $\rightarrow$ balance\\
% \textbf{quil} & $\rightarrow$ movement\\
% \textbf{tlesh} & $\rightarrow$ speed\\
% \textbf{lis} & $\rightarrow$ possession\\
% \textbf{lor} & $\rightarrow$ action\\
% \textbf{} & $\rightarrow$ name\\
% \textbf{tom} & $\rightarrow$ end\\

% \hline

% \textbf{hethnume} & $\rightarrow$ future\\
% \textbf{hulnume} & $\rightarrow$ past\\
% \textbf{lhelnume} & $\rightarrow$ present\\

% \textbf{lhonum} & $\rightarrow$ day (period)\\
% \textbf{tirlhon} & $\rightarrow$ week (period)\\
% \textbf{lheln} & $\rightarrow$ today\\
% \textbf{hethn} & $\rightarrow$ tomorrow\\
% \textbf{huln} & $\rightarrow$ yesterday\\
% \textbf{lorsel} & $\rightarrow$ wash\\
% \textbf{quilhum} & $\rightarrow$ dance\\

% \end{tabular}

% 
% \newpage
\onecolumn
\section*{Example Poem}
\textbf{Original Poem:}


\begin{quote}
{\fontsize{30pt}{10pt}\tovian
eth drixadrielas\\
na el lothre ea-iss\\
Emah glaranane\\
peloresul lea-kuil \\
analeth hlumelore
}
\end{quote}



\begin{quote}
Eth drixádriel'as\\
Ti-lothre lore-iss \\
Ti-glaranane\\
Peloresul lilore-quil \\
Ana'leth hlumelore
\end{quote}

\textbf{Translation:}
\begin{quote}
``I am the blood of elves,\\
Which swims in rivers,\\
Through ancient forests,\\
Moving under cities,\\
That which sustains the plants."
\end{quote}

This verse expresses an person's connection to their heritage, their harmony with nature, and their journey through both ancient and urban landscapes, ultimately acknowledging their role in the cycle of life as nourishment for flora.


% 
\section*{Example Sentences [WIP]}


\begin{tabular}{l|l c}
\textbf{Sentence} & \textbf{Tovian Translation} & &{\fontsize{20pt}{10pt}\tovian }\\ 

The sun shines. & lem lore-arthul &{\fontsize{20pt}{10pt}\tovian lem lore-arthul}\\
 
The sun is shining. & lem lore-arthul &{\fontsize{20pt}{10pt}\tovian lem lore-arthul}\\
 
The sun shone. & lem inlore-arthul &{\fontsize{20pt}{10pt}\tovian lem inlore-arthul}\\
 
The sun will shine. & lem vilore-arthul&{\fontsize{20pt}{10pt}\tovian lem vilore-arthul}\\
 
The sun has been shining. & lem deshilore-arthul &{\fontsize{20pt}{10pt}\tovian lem deshilore-arthul}\\
 
The sun is shining again. & &{\fontsize{20pt}{10pt}\tovian }\\
 
The sun will shine tomorrow. & &{\fontsize{20pt}{10pt}\tovian }\\
 
The sun shines brightly. & &{\fontsize{20pt}{10pt}\tovian }\\
 
The bright sun shines. & &{\fontsize{20pt}{10pt}\tovian }\\
 
The sun is rising now. & &{\fontsize{20pt}{10pt}\tovian }\\
 
All the people shouted. & &{\fontsize{20pt}{10pt}\tovian }\\
 
Some of the people shouted. & &{\fontsize{20pt}{10pt}\tovian }\\
 
Many of the people shouted twice. & &{\fontsize{20pt}{10pt}\tovian }\\
 
Happy people often shout. & &{\fontsize{20pt}{10pt}\tovian }\\
 
The kitten jumped up. & &{\fontsize{20pt}{10pt}\tovian }\\
 
The kitten jumped onto the table. & &{\fontsize{20pt}{10pt}\tovian }\\
 
My little kitten walked away. & &{\fontsize{20pt}{10pt}\tovian }\\
 
It's raining. & &{\fontsize{20pt}{10pt}\tovian }\\
 
The rain came down. & &{\fontsize{20pt}{10pt}\tovian }\\
 
The kitten is playing in the rain. & &{\fontsize{20pt}{10pt}\tovian }\\
 
The rain has stopped. & &{\fontsize{20pt}{10pt}\tovian }\\
 
Soon the rain will stop. & &{\fontsize{20pt}{10pt}\tovian }\\
 
I hope the rain stops soon. & &{\fontsize{20pt}{10pt}\tovian }\\
 
Once wild animals lived here. & &{\fontsize{20pt}{10pt}\tovian }\\
 
Slowly she looked around. & &{\fontsize{20pt}{10pt}\tovian }\\
 
Go away! & &{\fontsize{20pt}{10pt}\tovian }\\
 
Let's go! & &{\fontsize{20pt}{10pt}\tovian }\\
 
You should go. & &{\fontsize{20pt}{10pt}\tovian }\\
 
I will be happy to go. & &{\fontsize{20pt}{10pt}\tovian }\\
 
He will arrive soon. & &{\fontsize{20pt}{10pt}\tovian }\\
 
The baby's ball has rolled away. & &{\fontsize{20pt}{10pt}\tovian }\\
 
The two boys are working together. & &{\fontsize{20pt}{10pt}\tovian }\\
 
This mist will probably clear away. & &{\fontsize{20pt}{10pt}\tovian }\\
 
Lovely flowers are growing everywhere. & &{\fontsize{20pt}{10pt}\tovian }\\
 
We should eat more slowly. & &{\fontsize{20pt}{10pt}\tovian }\\
 
You have come too soon. & &{\fontsize{20pt}{10pt}\tovian }\\
 
You must write more neatly. & &{\fontsize{20pt}{10pt}\tovian }\\
 
Directly opposite stands a wonderful palace. & &{\fontsize{20pt}{10pt}\tovian }\\
 
Henry's dog is lost. & &{\fontsize{20pt}{10pt}\tovian }\\
 
My cat is black. & &{\fontsize{20pt}{10pt}\tovian }\\
 
The little girl's doll is broken. & &{\fontsize{20pt}{10pt}\tovian }\\
 
I usually sleep soundly. & &{\fontsize{20pt}{10pt}\tovian }\\
 
The children ran after Jack. & &{\fontsize{20pt}{10pt}\tovian }\\
 
I can play after school. & &{\fontsize{20pt}{10pt}\tovian }\\
 
We went to the village for a visit. & &{\fontsize{20pt}{10pt}\tovian }\\
 
We arrived at the river. & &{\fontsize{20pt}{10pt}\tovian }\\
 
I have been waiting for you. & &{\fontsize{20pt}{10pt}\tovian }\\
 
The campers sat around the fire. & &{\fontsize{20pt}{10pt}\tovian }\\
 
A little girl with a kitten sat near me. & &{\fontsize{20pt}{10pt}\tovian }\\
 
\end{tabular}

\begin{tabular}{l|l l}
\textbf{Sentence} & \textbf{Tovian Translation} \\ 
The child waited at the door for her father. & &{\fontsize{20pt}{10pt}\tovian }\\
 
Yesterday the oldest girl in the village lost her kitten. & &{\fontsize{20pt}{10pt}\tovian }\\
 
Were you born in this village? & &{\fontsize{20pt}{10pt}\tovian }\\
 

Can your brother dance well? & &{\fontsize{20pt}{10pt}\tovian }\\
 
Did the man leave? & &{\fontsize{20pt}{10pt}\tovian }\\
 
Is your sister coming for you? & &{\fontsize{20pt}{10pt}\tovian }\\
 
Can you come tomorrow? & &{\fontsize{20pt}{10pt}\tovian }\\
Have the neighbors gone away for the winter? & &{\fontsize{20pt}{10pt}\tovian }\\
 
Does the robin sing in the rain? & &{\fontsize{20pt}{10pt}\tovian }\\
 
Are you going with us to the concert? & &{\fontsize{20pt}{10pt}\tovian }\\
 
Have you ever travelled in the jungle? & &{\fontsize{20pt}{10pt}\tovian }\\
 
We sailed down the river for several miles. & &{\fontsize{20pt}{10pt}\tovian }\\
 
Everybody knows about hunting. & &{\fontsize{20pt}{10pt}\tovian }\\
 
On a Sunny morning after the solstice we \\ \indent started for the mountains. & &{\fontsize{20pt}{10pt}\tovian }\\
 
Tom laughed at the monkey's tricks. & &{\fontsize{20pt}{10pt}\tovian }\\
 
An old man with a walking stick stood beside the fence. & &{\fontsize{20pt}{10pt}\tovian }\\
 
The squirrel's nest was hidden by drooping boughs. & &{\fontsize{20pt}{10pt}\tovian }\\
 
The little seeds waited patiently under the snow for \\ \indent the warm spring sun. & &{\fontsize{20pt}{10pt}\tovian }\\
 
 
Many little girls with wreaths of flowers on their heads \\ \indent danced around the bonfire. & &{\fontsize{20pt}{10pt}\tovian }\\

 
The cover of the basket fell to the floor. & &{\fontsize{20pt}{10pt}\tovian }\\
 
The first boy in the line stopped at the entrance. & &{\fontsize{20pt}{10pt}\tovian }\\
On the top of the hill in a little hut lived a wise old woman. & &{\fontsize{20pt}{10pt}\tovian }\\
 
During our residence in the country we often \\ \indent walked in the pastures. & &{\fontsize{20pt}{10pt}\tovian }\\
 
When will your guests from the city arrive? & &{\fontsize{20pt}{10pt}\tovian }\\
 
Near the mouth of the river, its course turns sharply \\ \indent towards the East. & &{\fontsize{20pt}{10pt}\tovian }\\
 
 
Between the two lofty mountains lay a fertile valley. & &{\fontsize{20pt}{10pt}\tovian }\\
 
Among the wheat grew tall red poppies. & &{\fontsize{20pt}{10pt}\tovian }\\
 
The strong roots of the oak trees were torn from the ground. & &{\fontsize{20pt}{10pt}\tovian }\\
 
The sun looked down through the branches \\ \indent upon the children at play. & &{\fontsize{20pt}{10pt}\tovian }\\
 
The west wind blew across my face like a friendly caress. & &{\fontsize{20pt}{10pt}\tovian }\\
 
The spool of thread rolled across the floor. & &{\fontsize{20pt}{10pt}\tovian }\\
 
A box of growing plants stood in the Window. & &{\fontsize{20pt}{10pt}\tovian }\\
 
I am very happy. & &{\fontsize{20pt}{10pt}\tovian }\\
 
These oranges are juicy. & &{\fontsize{20pt}{10pt}\tovian }\\
 
Sea water is salty. & &{\fontsize{20pt}{10pt}\tovian }\\
 
The streets are full of people. & &{\fontsize{20pt}{10pt}\tovian }\\
 
Sugar tastes sweet. & &{\fontsize{20pt}{10pt}\tovian }\\
 
The fire feels hot. & &{\fontsize{20pt}{10pt}\tovian }\\
 
The little girl seemed lonely. & &{\fontsize{20pt}{10pt}\tovian }\\
 
The little boy's father had once been a sailor. & &{\fontsize{20pt}{10pt}\tovian }\\
 
I have lost my blanket. & &{\fontsize{20pt}{10pt}\tovian }\\
 
A robin has built his nest in the apple tree. & &{\fontsize{20pt}{10pt}\tovian }\\
 
At noon we ate our lunch by the roadside. & &{\fontsize{20pt}{10pt}\tovian }\\
 
Mr. Jones made a knife for his little boy. & &{\fontsize{20pt}{10pt}\tovian }\\
 
Their voices sound very happy. & &{\fontsize{20pt}{10pt}\tovian }\\
 
Is today Monday? & &{\fontsize{20pt}{10pt}\tovian }\\
 
Have all the leaves fallen from the tree? & &{\fontsize{20pt}{10pt}\tovian }\\
 
Will you be ready on time? & &{\fontsize{20pt}{10pt}\tovian }\\
 
Will you send this message for me? & &{\fontsize{20pt}{10pt}\tovian }\\
 
Are you waiting for me? & &{\fontsize{20pt}{10pt}\tovian }\\
 

\end{tabular}

\begin{tabular}{l|l l}
Is this the first kitten of the litter? & &{\fontsize{20pt}{10pt}\tovian }\\
 
Are these shoes too big for you? & &{\fontsize{20pt}{10pt}\tovian }\\
 
How wide is the River? & &{\fontsize{20pt}{10pt}\tovian }\\
 
Listen. & &{\fontsize{20pt}{10pt}\tovian }\\
 
Sit here by me. & &{\fontsize{20pt}{10pt}\tovian }\\
 
Keep this secret until tomorrow. & &{\fontsize{20pt}{10pt}\tovian }\\
 
Come with us. & &{\fontsize{20pt}{10pt}\tovian }\\
 
Bring your friends with you. & &{\fontsize{20pt}{10pt}\tovian }\\
 
Be careful. & &{\fontsize{20pt}{10pt}\tovian }\\
 
Have some tea. & &{\fontsize{20pt}{10pt}\tovian }\\
 
Pip and his dog were great friends. & &\\
 
John and Elizabeth are brother and sister. & &\\
 
You and I will go together. & &\\
 
They opened all the doors and windows. & &\\
 
He is small, but strong. & &\\
 
Is this tree an oak or a maple? & &\\
 
Does the sky look blue or gray? & &\\
 
Come with your father or mother. & &\\
 
I am tired, but very happy. & &\\
 
He played a tune on his wonderful flute. & &\\
 
Toward the end of August the days grow much shorter. & &\\
 
A company of soldiers marched over the hill  &\\ \indent and across the meadow. & &\\
 
The first part of the story is very interesting. & &\\
 
The crow dropped some pebbles into the pitcher  &\\ \indent and raised the water to the brim. & &\\
 
The baby clapped her hands and laughed in glee. & &\\
 
Stop your game and be quiet. & &\\
 
The sound of the drums grew louder and louder. & &\\
 
Do you like summer or winter better? & &\\
 
That boy will have a wonderful trip. & &\\
 
They popped corn, and then sat around  &\\ \indent  the fire and ate it. & &\\
 
They won the first two games, but lost the last one. & &\\
 
Take this note, carry it to your  &\\ \indent mother; and wait for an answer. & &\\
 
I awoke early, dressed hastily, and went down to breakfast. & &\\
 
Aha! I have caught you! & &\\
 
This string is too short! & &\\
 
Oh, dear! the wind has blown my hat away! & &\\
 
Alas! that news is sad indeed! & &\\
 
Whew! that cold wind freezes my nose! & &\\
 
Are you warm enough now? & &\\
 
They heard the warning too late. & &\\
 
We are a brave people, and love our country. & &\\
 
All the children came except Mary. & &\\
 
Jack seized a handful of pebbles and threw  &\\ \indent  them into the lake. & &\\
 
This cottage stood on a low hill, at some  &\\ \indent  distance from the village. & &\\
 
On a fine summer evening, the two old people  &\\ \indent  were sitting  outside the door of their cottage. & &\\
 
We visited my uncle's village, the largest village in the world. & &\\
 
We learn something new each day. & &\\
 
The market begins five minutes earlier this week. & &\\
 
Did you find the distance too great? & &\\
 
Hurry, children. & &\\
 
Madam, I will obey your command. & &\\
 
Here under this tree they gave their guests a splendid feast. & &\\
 
 
 

\end{tabular}


\begin{tabular}{l|l l }
\textbf{Sentence} & \textbf{Tovian Translation} &\\ 
In winter I get up at night, and dress by yellow candlelight. & &\\
 
Tell the last part of that story again. & &\\
 
 
Our bird's name is Jacko. & &\\
 
The river knows the way to the sea. & &\\
 
The boat sails away, like a bird on the wing. & &\\
 
They looked cautiously about, but saw nothing. & &\\
 
The little house had three rooms, a sitting room, a bedroom, &\\ \indent and a tiny kitchen. & &\\
 
Be quick or you will be too late. & ersh'seloreas wa shol vilore-fe'ninesh & \\ 
&{\fontsize{20pt}{10pt}\tovian ersh'seloreas wa shol vilore-fe'ninesh}&\\
 
Will you go with us or wait here? & &\\
 
She was always, shabby, often ragged, and on cold  &\\ \indent days very uncomfortable. & &\\
 
Think first and then act. & &\\
I stood, a little mite of a girl, upon a chair by the window, &\\ \indent and watched the falling snowflakes. & &\\
 
Show the guests these shells, my son, and tell them  &\\ \indent their strange history. & &\\
Be satisfied with nothing but your best. & &\\

We consider them our faithful friends. & &\\
 
We will make this place our home. & &\\
 
The squirrels make their nests warm and snug with  &\\ \indent soft moss and leaves. & &\\
 
The little girl made the doll's dress herself. & &\\
 
I hurt myself. & &\\
 
She was talking to herself. & &\\
 
He proved himself trustworthy. & &\\
 
We could see ourselves in the water. & &\\
 
Do it yourself. & &\\
 
I feel ashamed of myself. & &\\
 
Sit here by yourself. & &\\
 
The dress of the little princess was embroidered with roses,  &\\ \indent the national flower of the Country. & &\\
 
They wore red caps, the symbol of liberty. & &\\
 
With him as our protector, we fear no danger. & &\\
 
All her finery, lace, ribbons, and feathers,  &\\ \indent  was packed away in a trunk. & &\\
 
Light he thought her, like a feather. & &\\
 
Every spring and fall our cousins pay us a long visit. & &\\
 
In our climate the grass remains green all winter. & &\\
 
The boy who brought the book has gone. & &\\
 
These are the flowers that you ordered. & &\\
 
I have lost the book that you gave me. & &\\
 
The fisherman who owned the boat now demanded payment. & &\\
 
Come when you are called. & &\\
 
I shall stay at home if it rains. & &\\
 
When he saw me, he stopped. & &\\
 
Do not laugh at me because I seem so absent minded. & &\\
 
I shall lend you the books that you need. & &\\
 
Come early next Monday if you can. & &\\
 

\end{tabular}


\begin{tabular}{l|l l}
\textbf{Sentence} & \textbf{Tovian Translation} &\\ 
 
If you come early, wait in the hall. & shol kolore-quilsor nesh, selore-dolr im'moryuth!&\\
 
I had a younger brother whose name was Antonio. & &\\
 
Gnomes are little men who live under the ground. & &\\
 
He is loved by everybody, because he has  &\\ \indent  a gentle disposition. & &\\
 
Hold the horse while I run and get my cap. & &\\
 
I have found the ring I lost. & &\\
 
Play and I will sing. & &\\
 
That is the funniest story I ever heard. & &\\
 
She is taller than her brother. & &\\
 
They are no wiser than we. & &\\
 
Light travels faster than sound. & &\\
 
We have more time than they. & &\\
 
She has more friends than enemies. & &\\
 
He was very poor, and with his wife and five children lived  &\\ \indent in a little low cabin of logs and stones. & &\\
 
When the wind blew, the traveler wrapped his mantle more  &\\ \indent closely around him. & &\\
 
I am sure that we can go. & &\\
 
We went back to the place where we saw the roses. & &\\

"This tree is fifty feet high," said the gardener. & &\\
 
I think that this train leaves five minutes earlier today. & &\\
 
My opinion is that the governor will grant him a pardon. & &\\
 
Why he has left the city is a mystery. & risa wel pelor delore-quilnisor pfetr'as  &\\ 
 
The house stands where three roads meet. & &\\
 
He has far more money than brains. & &\\
 
Evidently that gate is never opened, for the long grass and the &\\ \indent  great hemlocks grow close against it. & &\\
 
I met a little cottage girl; she was eight years old, she said. & &\\
&&\\

If he has wisdom (he might), he would speak truthfully. &
La-wel yale-ara indekolore-lis, la-wel kolore-fethr sile-haru &\\
&
{\fontsize{14pt}{10pt}\tovian la-wel yale-ara indekolore-lis la-wel kolore-fethr sile-haru}&
\\

If he had wisdom (he doesn't), he would speak truthfully. &
La-wel yale-ara indexolore-lis, la-wel xolore-fethr sile-haru &\\
&
{\fontsize{14pt}{10pt}\tovian la-wel yale-ara xolore-lis la-wel xolore-fethr sile-haru}&
\\

If he had wisdom (he might), would he speak truthfully? &
La-wel yale-ara indekolore-lis, la-wel kolore-fethr sile-haru? &\\
&
{\fontsize{14pt}{10pt}\tovian la-wel yale-ara indekolore-lis, la-wel kolore-fethr sile-haru}&
\\

\end{tabular}

% \twocolumn

\section*{Additional Vocabulary}

\begin{tabular}{r c c l}
\textbf{???} &&$\rightarrow$ hello \\
\textbf{???} &&$\rightarrow$ goodbye \\
\textbf{kului} &&$\rightarrow$ cinnamon \\
\textbf{helor} &&$\rightarrow$ glass \\
\textbf{dora} &&$\rightarrow$ blue \\
\textbf{fethr} &&$\rightarrow$ speaker \\
\textbf{drix} &&$\rightarrow$ blood \\
\textbf{iss} &&$\rightarrow$ fly/swim \\
\textbf{tes} &&$\rightarrow$ sky \\
\textbf{imalor} &&$\rightarrow$ experience \\
\textbf{emethl} &&$\rightarrow$ intelligence \\
\textbf{fer} &&$\rightarrow$ outdoors \\
\textbf{lom} &&$\rightarrow$ soft \\
\textbf{marth} &&$\rightarrow$ clean \\
\textbf{sura} &&$\rightarrow$ regular/routine \\
\textbf{thu} &&$\rightarrow$ high \\
\textbf{gol} &&$\rightarrow$ passive \\
\textbf{val} &&$\rightarrow$ skilled \\
\textbf{har} &&$\rightarrow$ alert \\
\textbf{curth} &&$\rightarrow$ expensive \\
\textbf{sepa} &&$\rightarrow$ fresh \\
\textbf{asc} &&$\rightarrow$ hot \\
\textbf{bem} &&$\rightarrow$ sweet \\
\textbf{bhal} &&$\rightarrow$ sour \\
\textbf{leth'har} &&$\rightarrow$ spicy \\
\textbf{fex} &&$\rightarrow$ bitter \\
\textbf{som} &&$\rightarrow$ umami \\
\textbf{alu} &&$\rightarrow$ deep \\
\textbf{tkesh} &&$\rightarrow$ foot \\
\textbf{leth} &&$\rightarrow$ taste \\
\textbf{melor} &&$\rightarrow$ plant \\
\textbf{tith} &&$\rightarrow$ quiet \\
\textbf{nin} &&$\rightarrow$ small (diminutive) \\
\textbf{num} &&$\rightarrow$ path \\
\textbf{yul} &&$\rightarrow$ song \\
\textbf{mril} &&$\rightarrow$ eye \\
\textbf{ruy} &&$\rightarrow$ feather \\
\textbf{pila} &&$\rightarrow$ love \\
\textbf{vehk} &&$\rightarrow$ kill \\
\textbf{ilor} &&$\rightarrow$ death \\
\textbf{quil} &&$\rightarrow$ movement \\
\textbf{???} &&$\rightarrow$ carry \\
\textbf{wen} &&$\rightarrow$ beauty \\
\textbf{asp} &&$\rightarrow$ sharp \\
\textbf{???} &&$\rightarrow$ hold \\
\textbf{lim} &&$\rightarrow$ star \\
\textbf{???} &&$\rightarrow$ within \\
\end{tabular}

\begin{tabular}{rl}
\textbf{???} & $\rightarrow$ vast \\
\textbf{elan} & $\rightarrow$ hand \\
\textbf{???} & $\rightarrow$ search \\
\textbf{melorerae} & $\rightarrow$ oasis \\
\textbf{???} & $\rightarrow$ dusk \\
\textbf{???} & $\rightarrow$ dawn \\
\textbf{iro} & $\rightarrow$ morning \\
\textbf{} & $\rightarrow$ afternoon \\
\textbf{tuml} & $\rightarrow$ night \\
\textbf{???} & $\rightarrow$ attack \\
\textbf{???} & $\rightarrow$ friend \\
\textbf{yush} & $\rightarrow$ block \\
\textbf{a} & $\rightarrow$ of \\
\textbf{ul} & $\rightarrow$ and \\
\textbf{???} & $\rightarrow$ to \\
\textbf{???} & $\rightarrow$ in \\
\textbf{???} & $\rightarrow$ that \\
\textbf{???} & $\rightarrow$ was \\
\textbf{???} & $\rightarrow$ being \\
\textbf{???} & $\rightarrow$ with \\
\textbf{???} & $\rightarrow$ is \\
\textbf{???} & $\rightarrow$ for \\
\textbf{???} & $\rightarrow$ as \\
\textbf{???} & $\rightarrow$ had \\
\textbf{???} & $\rightarrow$ you \\
\textbf{???} & $\rightarrow$ not \\
\textbf{???} & $\rightarrow$ on \\
\textbf{???} & $\rightarrow$ at \\
\textbf{???} & $\rightarrow$ by \\
\textbf{???} & $\rightarrow$ which \\
\textbf{lore-lis} & $\rightarrow$ have \\
\textbf{lis} & $\rightarrow$ possession\\
\textbf{wa} & $\rightarrow$ or \\
\textbf{???} & $\rightarrow$ from \\
\textbf{???} & $\rightarrow$ this \\
\textbf{???} & $\rightarrow$ but \\
\textbf{???} & $\rightarrow$ all \\
\textbf{nesh} & $\rightarrow$ early \\
\end{tabular}


\begin{tabular}{rl}
\textbf{???} & $\rightarrow$ were \\
\textbf{???} & $\rightarrow$ are \\
\textbf{???} & $\rightarrow$ one \\
\textbf{???} & $\rightarrow$ so \\
\textbf{???} & $\rightarrow$ an \\
\textbf{???} & $\rightarrow$ said \\
\textbf{???} & $\rightarrow$ they \\
\textbf{???} & $\rightarrow$ who \\
\textbf{???} & $\rightarrow$ would \\
\textbf{???} & $\rightarrow$ been \\
\textbf{???} & $\rightarrow$ will (N) \\
\textbf{???} & $\rightarrow$ when \\
\textbf{???} & $\rightarrow$ there \\
\textbf{???} & $\rightarrow$ if \\
\textbf{???} & $\rightarrow$ more \\
\textbf{???} & $\rightarrow$ out \\
\textbf{???} & $\rightarrow$ up \\
\textbf{???} & $\rightarrow$ into \\
\textbf{???} & $\rightarrow$ do \\
\textbf{???} & $\rightarrow$ any \\
\textbf{???} & $\rightarrow$ your \\
\textbf{???} & $\rightarrow$ what \\
\textbf{???} & $\rightarrow$ has \\
\textbf{or} & $\rightarrow$ person \\
\textbf{???} & $\rightarrow$ could \\
\textbf{???} & $\rightarrow$ other \\
\textbf{???} & $\rightarrow$ than \\
\textbf{???} & $\rightarrow$ our \\
\textbf{???} & $\rightarrow$ some \\
\textbf{???} & $\rightarrow$ very \\
\textbf{???} & $\rightarrow$ time \\
\textbf{???} & $\rightarrow$ upon \\
\textbf{???} & $\rightarrow$ about \\
\textbf{???} & $\rightarrow$ may \\
\textbf{???} & $\rightarrow$ its \\
\textbf{???} & $\rightarrow$ only \\
\textbf{im'num} & $\rightarrow$ now \\
\textbf{???} & $\rightarrow$ like \\
\textbf{???} & $\rightarrow$ then \\
\textbf{???} & $\rightarrow$ can \\
\textbf{???} & $\rightarrow$ should \\
\textbf{???} & $\rightarrow$ did \\
\textbf{???} & $\rightarrow$ such \\
\textbf{???} & $\rightarrow$ great \\
\textbf{???} & $\rightarrow$ before \\
\textbf{???} & $\rightarrow$ must \\
\textbf{???} & $\rightarrow$ two \\
\textbf{???} & $\rightarrow$ these \\
\textbf{???} & $\rightarrow$ see \\
\textbf{arash} & $\rightarrow$ knowledge \\
\textbf{???} & $\rightarrow$ over \\
\textbf{???} & $\rightarrow$ down \\
\textbf{???} & $\rightarrow$ after \\
\textbf{???} & $\rightarrow$ first \\
\textbf{???} & $\rightarrow$ good \\
\textbf{???} & $\rightarrow$ never \\
\end{tabular}


\begin{tabular}{rl}
\textbf{???} & $\rightarrow$ most \\
\textbf{???} & $\rightarrow$ old \\
\textbf{???} & $\rightarrow$ day \\
\textbf{???} & $\rightarrow$ where \\
\textbf{asor} & $\rightarrow$ here \\
\textbf{sor} & $\rightarrow$ close \\
\textbf{quilsor} & $\rightarrow$ to come \\
\textbf{???} & $\rightarrow$ way \\
\textbf{???} & $\rightarrow$ work \\
\textbf{???} & $\rightarrow$ life \\
\textbf{???} & $\rightarrow$ without \\
\textbf{quil} & $\rightarrow$ to go \\
\textbf{???} & $\rightarrow$ well \\
\textbf{ema} & $\rightarrow$ through \\
\textbf{???} & $\rightarrow$ long \\
\textbf{???} & $\rightarrow$ say \\
\textbf{???} & $\rightarrow$ might \\
\textbf{???} & $\rightarrow$ how \\
\textbf{???} & $\rightarrow$ too \\
\textbf{???} & $\rightarrow$ even \\
\textbf{???} & $\rightarrow$ again \\
\textbf{???} & $\rightarrow$ many \\
\textbf{???} & $\rightarrow$ back \\
\textbf{???} & $\rightarrow$ think \\
\textbf{???} & $\rightarrow$ every \\
\textbf{???} & $\rightarrow$ people \\
\textbf{???} & $\rightarrow$ went \\
\textbf{???} & $\rightarrow$ same \\
\textbf{???} & $\rightarrow$ last \\
\textbf{tason} & $\rightarrow$ worse \\
\textbf{áemethl} & $\rightarrow$ thought \\
\textbf{???} & $\rightarrow$ away \\
\textbf{???} & $\rightarrow$ under \\
\textbf{???} & $\rightarrow$ take \\
\textbf{???} & $\rightarrow$ found \\
\textbf{elan} & $\rightarrow$ hand \\
\textbf{elane} & $\rightarrow$ hands \\
\textbf{mrile} & $\rightarrow$ eyes \\
\textbf{???} & $\rightarrow$ still \\
\textbf{???} & $\rightarrow$ place \\
\textbf{???} & $\rightarrow$ while \\
\textbf{???} & $\rightarrow$ just \\
\textbf{???} & $\rightarrow$ also \\
\textbf{???} & $\rightarrow$ young \\
\textbf{???} & $\rightarrow$ yet \\
\textbf{???} & $\rightarrow$ though \\
\textbf{???} & $\rightarrow$ against \\
\textbf{???} & $\rightarrow$ things \\
\textbf{???} & $\rightarrow$ get \\
\textbf{???} & $\rightarrow$ ever \\
\textbf{???} & $\rightarrow$ give \\
\textbf{???} & $\rightarrow$ years \\
\textbf{???} & $\rightarrow$ off \\
\textbf{???} & $\rightarrow$ face \\
\end{tabular}


\begin{tabular}{rl}
\textbf{???} & $\rightarrow$ nothing \\
\textbf{???} & $\rightarrow$ right \\
\textbf{???} & $\rightarrow$ once \\
\textbf{???} & $\rightarrow$ left \\
\textbf{???} & $\rightarrow$ part \\
\textbf{???} & $\rightarrow$ saw \\
\textbf{mril} & $\rightarrow$ world \\
\textbf{lem} & $\rightarrow$ sun \\
\textbf{ea-arthul} & $\rightarrow$ shine \\
\textbf{arthul} & $\rightarrow$ bright \\
\textbf{nisil} & $\rightarrow$ break \\
\textbf{???} & $\rightarrow$ head \\
\textbf{???} & $\rightarrow$ took \\
\textbf{???} & $\rightarrow$ new \\
\textbf{simra} & $\rightarrow$ always \\
\textbf{???} & $\rightarrow$ put \\
\textbf{???} & $\rightarrow$ night \\
\textbf{???} & $\rightarrow$ each \\
\textbf{rethin} & $\rightarrow$ between \\
\textbf{???} & $\rightarrow$ tell \\
\textbf{???} & $\rightarrow$ mind \\
\textbf{arpel} & $\rightarrow$ heart \\
\textbf{???} & $\rightarrow$ few \\
\textbf{elan} & $\rightarrow$ because \\
\textbf{???} & $\rightarrow$ thing \\
\textbf{???} & $\rightarrow$ far \\
\end{tabular}


\begin{tabular}{rl}
\textbf{???} & $\rightarrow$ seemed \\
\textbf{???} & $\rightarrow$ look \\
\textbf{???} & $\rightarrow$ called \\
\textbf{???} & $\rightarrow$ whole \\
\textbf{???} & $\rightarrow$ set \\
\textbf{???} & $\rightarrow$ both \\
\textbf{???} & $\rightarrow$ got \\
\textbf{???} & $\rightarrow$ find \\
\textbf{sil} & $\rightarrow$ done \\
\textbf{???} & $\rightarrow$ heard \\
\textbf{???} & $\rightarrow$ name \\
\textbf{???} & $\rightarrow$ days \\
\textbf{???} & $\rightarrow$ told \\
\textbf{???} & $\rightarrow$ let \\
\textbf{edom} & $\rightarrow$ country \\
\textbf{???} & $\rightarrow$ asked \\
\textbf{???} & $\rightarrow$ going \\
\textbf{???} & $\rightarrow$ seen \\
\textbf{???} & $\rightarrow$ better \\
\textbf{toki} & $\rightarrow$ dog \\
\textbf{???} & $\rightarrow$ having \\
\textbf{???} & $\rightarrow$ home \\
\textbf{???} & $\rightarrow$ knew \\
\textbf{???} & $\rightarrow$ side \\
\textbf{???} & $\rightarrow$ something \\
\textbf{???} & $\rightarrow$ moment \\
\textbf{???} & $\rightarrow$ among \\
\textbf{???} & $\rightarrow$ course \\
\textbf{???} & $\rightarrow$ enough \\
\textbf{???} & $\rightarrow$ word \\
\textbf{???} & $\rightarrow$ soon \\
\textbf{bon} & $\rightarrow$ full \\
\textbf{???} & $\rightarrow$ end \\
\textbf{???} & $\rightarrow$ gave \\
\textbf{???} & $\rightarrow$ room \\
\textbf{???} & $\rightarrow$ almost \\
\textbf{???} & $\rightarrow$ want \\
\textbf{???} & $\rightarrow$ however \\
\end{tabular}


\begin{tabular}{rl}
\textbf{ahn} & $\rightarrow$ light \\
\textbf{???} & $\rightarrow$ quite \\
\textbf{???} & $\rightarrow$ bring \\
\textbf{???} & $\rightarrow$ given \\
\textbf{yuthil} & $\rightarrow$ door \\
\textbf{thul} & $\rightarrow$ best \\
\textbf{???} & $\rightarrow$ turned \\
\textbf{???} & $\rightarrow$ taken \\
\textbf{???} & $\rightarrow$ does \\
\textbf{???} & $\rightarrow$ use \\
\textbf{???} & $\rightarrow$ felt \\
\textbf{???} & $\rightarrow$ until \\
\textbf{???} & $\rightarrow$ since \\
\textbf{lathn} & $\rightarrow$ power \\
\textbf{aipte} & $\rightarrow$ themselves \\
\textbf{???} & $\rightarrow$ used \\
\textbf{???} & $\rightarrow$ rather \\
\textbf{???} & $\rightarrow$ began \\
\textbf{???} & $\rightarrow$ present \\
\textbf{pfetr} & $\rightarrow$ mystery \\
\textbf{hlufethr} & $\rightarrow$ voice \\
\textbf{???} & $\rightarrow$ others \\
\textbf{???} & $\rightarrow$ works \\
\textbf{???} & $\rightarrow$ less \\
\textbf{???} & $\rightarrow$ money \\
\textbf{???} & $\rightarrow$ next \\
\textbf{???} & $\rightarrow$ poor \\
\textbf{???} & $\rightarrow$ stood \\
\textbf{???} & $\rightarrow$ form \\
\textbf{???} & $\rightarrow$ within \\
\textbf{usol} & $\rightarrow$ together \\
\textbf{???} & $\rightarrow$ till \\
\textbf{???} & $\rightarrow$ large \\
\textbf{???} & $\rightarrow$ matter \\
\textbf{???} & $\rightarrow$ kind \\
\textbf{???} & $\rightarrow$ often \\
\textbf{???} & $\rightarrow$ year \\
\textbf{???} & $\rightarrow$ friend \\
\textbf{???} & $\rightarrow$ order \\
\textbf{???} & $\rightarrow$ round \\
\textbf{???} & $\rightarrow$ true \\
\textbf{???} & $\rightarrow$ anything \\
\textbf{???} & $\rightarrow$ keep \\
\end{tabular}


\begin{tabular}{rl}
\textbf{???} & $\rightarrow$ sent \\
\textbf{???} & $\rightarrow$ means \\
\textbf{???} & $\rightarrow$ believe \\
\textbf{yu'quil}  & $\rightarrow$ pass \\
\textbf{???} & $\rightarrow$ feet \\
\textbf{???} & $\rightarrow$ near \\
\textbf{???} & $\rightarrow$ public \\
\textbf{???} & $\rightarrow$ state \\
\textbf{???} & $\rightarrow$ hundred \\
\textbf{???} & $\rightarrow$ hope \\
\textbf{???} & $\rightarrow$ alone \\
\textbf{sir} & $\rightarrow$ above \\
\textbf{???} & $\rightarrow$ case \\
\textbf{???} & $\rightarrow$ high \\
\textbf{???} & $\rightarrow$ read \\
\textbf{???} & $\rightarrow$ received \\
\textbf{haru} & $\rightarrow$ fact \\
\textbf{???} & $\rightarrow$ gone \\
\textbf{???} & $\rightarrow$ girl \\
\textbf{pea-arash} & $\rightarrow$ known \\
\textbf{???} & $\rightarrow$ hear \\
\textbf{???} & $\rightarrow$ times \\
\textbf{???} & $\rightarrow$ least \\
\textbf{???} & $\rightarrow$ perhaps \\
\textbf{???} & $\rightarrow$ sure \\
\textbf{???} & $\rightarrow$ indeed \\
\textbf{???} & $\rightarrow$ open \\
\textbf{???} & $\rightarrow$ body \\
\textbf{aipt} & $\rightarrow$ itself \\
\textbf{???} & $\rightarrow$ along \\
\textbf{???} & $\rightarrow$ land \\
\textbf{???} & $\rightarrow$ return \\
\textbf{ea-quilnisor} & $\rightarrow$ leave (depart) \\
\textbf{ea-wushed} & $\rightarrow$ leave (abandon) \\
\textbf{wushed} & $\rightarrow$ abandonment \\
\textbf{ishma} & $\rightarrow$ guide \\
\textbf{rith} & $\rightarrow$ air \\
\textbf{anafer} & $\rightarrow$ nature \\
\textbf{???} & $\rightarrow$ answer \\
\textbf{???} & $\rightarrow$ either \\
\textbf{???} & $\rightarrow$ law \\
\textbf{???} & $\rightarrow$ help \\
\textbf{ea-quil'sul} & $\rightarrow$ lay \\
\end{tabular}


\begin{tabular}{rl}
\textbf{???} & $\rightarrow$ point \\
\textbf{ni'nan'nin} & $\rightarrow$ child \\
\textbf{???} & $\rightarrow$ letter \\
\textbf{???} & $\rightarrow$ wish \\
\textbf{???} & $\rightarrow$ cried \\
\textbf{ersh} & $\rightarrow$ fast \\
\textbf{???} & $\rightarrow$ number \\
\textbf{???} & $\rightarrow$ therefore \\
\textbf{???} & $\rightarrow$ hour \\
\textbf{???} & $\rightarrow$ held \\
\textbf{???} & $\rightarrow$ free \\
\textbf{keshl} & $\rightarrow$ war \\
\textbf{???} & $\rightarrow$ during \\
\textbf{???} & $\rightarrow$ several \\
\textbf{shissan} & $\rightarrow$ business \\
\textbf{???} & $\rightarrow$ whether \\
\textbf{???} & $\rightarrow$ manner \\
\textbf{???} & $\rightarrow$ second \\
\textbf{???} & $\rightarrow$ reason \\
\textbf{???} & $\rightarrow$ reply \\
\textbf{???} & $\rightarrow$ call \\
\textbf{???} & $\rightarrow$ general \\
\textbf{risa} & $\rightarrow$ why \\
\textbf{???} & $\rightarrow$ behind \\
\textbf{???} & $\rightarrow$ become \\
\textbf{???} & $\rightarrow$ lost \\
\textbf{???} & $\rightarrow$ forth \\
\textbf{emiyr} & $\rightarrow$ thousand \\
\textbf{???} & $\rightarrow$ family \\
\textbf{ea-beth} & $\rightarrow$ feel \\
\textbf{hluara} & $\rightarrow$ soul \\
\textbf{hluara} & $\rightarrow$ spirit \\
\textbf{asher} & $\rightarrow$ question \\
\textbf{???} & $\rightarrow$ care \\
\textbf{aharu} & $\rightarrow$ truth \\
\textbf{tkesh'sul} & $\rightarrow$ ground \\
\textbf{???} & $\rightarrow$ really \\
\textbf{???} & $\rightarrow$ rest \\
\textbf{???} & $\rightarrow$ mean \\
\textbf{???} & $\rightarrow$ different \\
\textbf{ea-sil} & $\rightarrow$ make \\
\textbf{???} & $\rightarrow$ possible \\
\textbf{???} & $\rightarrow$ fell \\
\textbf{yu} & $\rightarrow$ towards \\
\textbf{tasondriel} & $\rightarrow$ human \\
\textbf{aglardriel} & $\rightarrow$ wood-person \\
\textbf{or'keshl} & $\rightarrow$ orc \\
\textbf{???} & $\rightarrow$ kept \\
\textbf{tros} & $\rightarrow$ sad \\
\textbf{trossel} & $\rightarrow$ tear \\
\textbf{dren} & $\rightarrow$ flower \\
\textbf{tis} & $\rightarrow$ stone \\
\textbf{tisàhlúara} & $\rightarrow$ soul stone \\
\textbf{cyran} & $\rightarrow$ dapple \\
\textbf{lerth} & $\rightarrow$ tree \\
\textbf{merth} & $\rightarrow$ sound\\
\textbf{yulara}  & $\rightarrow$ name\\
\textbf{merthyara} & $\rightarrow$ prayer\\
\textbf{dorl} & $\rightarrow$ wait\\
\textbf{passage} & $\rightarrow$ moryuth\\
\textbf{hallway} & $\rightarrow$ moryuth\\
\end{tabular}


\iffalse
# See the documentation linked above for full details.

# These set up assimilations ("np" will be changed to "mp") and a few other
# minor things.  It is possible for assimilations to produce phonemes you
# don't want, which is why one of the filters below is: ŋ > n.
with: std-ipa-features std-assimilations coronal-metathesis

# This determines the sort order of the output, and also helps
# the 'with:' line settings understand what is going on.
letters: a b d e f g h i q l m n o p r s t u v w x y sh th

# THE ORDER OF PHONEME CLASSES MATTERS.  The very first phoneme
# will be picked much, much more often than the last.  The order
# here is approximately natural for a lot of languages.  You can
# change the character of a language a lot by shuffling which
# phonemes occur most often, which the least.
C = th sh r f d n s h l m p t g v q w x y b
D = f d n p t
V = a e i o u

# Macros are only a convenience to reduce typing, and do simple
# substitution only.  You can't use macros inside other macros.
$S = CVD?

# The first word shape will be picked most often, the last least
# often.
words: V?$S$S V?$S V?$S$S$S

# Rejections and filters use Javascript regular expressions.
# 'reject:' simply throws away a word.  Filter turns certain
# paterns into something else.
reject: wu yi w$ y$ h$ ʔʔ (p|t|k|ʔ)h
filter: nr > tr; mr > pr; ŋ > n

# Haplology - remove repeats.
reject: (..+)\1+
\fi



Potential words:

gashl

pathr






% {\centering Vowels
\begin{center}
    \begin{tabular}{cccccccc}
        & \textbf{Front} & & \textbf{Central} & & \textbf{Back} & & \\
        \textbf{Close} & i & &  & & u & &  \\
        \textbf{Close-mid} & e & &  & & o & &  \\
        \textbf{Mid} &  & &  & &  & &  \\
        \textbf{Open-mid} &  & &  & &  & &  \\
        \textbf{Open} & a & &  & &  & &  \\
    \end{tabular}
\end{center}

\medskip

{\huge WIP, unfinished/incorrect:}

\setlength{\tabcolsep}{2pt} % Adjust column separation (default is 6pt)
\begin{center}
{\fontsize{7pt}{10pt}
    \begin{tabular}{ccccccccccccccc}
        & & \textbf{Bilabial} & & \textbf{Labiodental} & & \textbf{Dental} & & \textbf{Alveolar} & & \textbf{Postalveolar} & & \textbf{Retroflex} & & \textbf{Palatal} & & \textbf{Velar} & & \textbf{Uvular} & & \textbf{Pharyngeal} & & \textbf{Glottal} \\
        \textbf{Plosive} & & p & &  & & \textsubbridge{t} & & & &  & & \textrtaild & & q & &  & &  & &  \\
        \textbf{Nasal} & & m & &  & &  & & n & &  & &  & & ŋ & &  & &  & &  \\
        \textbf{Trill} & &  & &  & &  & & r & &  & &  & &  & &  & &  & &  \\
        \textbf{Tap/Flap} & &  & &  & &  & &  & &  & &  & &  & &  & &  & &  \\
        \textbf{Fricative} & &  & &  & &  & & s & & \textesh & &\texttheta & & \textphi & &  & & h \\
        \textbf{Lateral Fricative} & &  & &  & &  & &  & &  & &  & &  & &  \\
        \textbf{Approximant} & &  & &  & &  & &  & &  & &  & & j & &  & &  \\
        \textbf{Lateral Approximant} & &  & &  & &  & & ɫ & &  & &  & &  & &  & &  & &  \\
    \end{tabular}
}

Other sounds:

\textturnw

\textltilde


\end{center}
}




\onecolumn

\section{Phonetic Inventory}

{\fontsize{8pt}{10pt}
\setlength{\tabcolsep}{8pt} % Adjust the column spacing here
\begin{center}
\begin{tabular}{|r|c|c|c|c|c|c|c|c|}
\hline
 & \textbf{Bilabial} & \textbf{Labiodental} & \textbf{Dental} & \textbf{Alveolar} & \textbf{Palatal} & \textbf{Postalveolar} & \textbf{Velar} & \textbf{Glottal} \\
\hline
\textbf{Plosive} & \textipa{p b} &  &  & \textipa{t d} & \textipa{\textbardotlessj} &  & \textipa{k g} &  \\
\hline
\textbf{Nasal} & \textipa{m} &  &  & \textipa{n} &  &  & \textipa{\ng} &  \\
\hline
\textbf{Trill} &  &  &  & \textipa{r} &  &  &  &  \\
\hline
\textbf{Fricative} & \textipa{\textphi} & \textipa{f v} & \textipa{\texttheta \red{\textipa{\dh}}} & \textipa{s \textcolor{red}{z}} &  & \textipa{\textesh \textcolor{red}{\textyogh}} &  & \textipa{h} \\
\hline
\textbf{Lateral Fricative} &  &  &  & \textipa{\textbeltl} &  &  &  &  \\
\hline
\textbf{Approximant} &  &  &  &  &\textipa{j} &  &  & \\
\hline
\textbf{Lateral Approximant} &  &  &  & \textipa{l} &  &  &  &  \\
\hline
\end{tabular}
\\
other: \textipa{w}. Sounds in \red{red} were in Ancient Lefethr but have since been lost.
\end{center}
}
\medskip
\medskip

\section{History of Sound Changes}
{
\setlength{\tabcolsep}{8pt} % Adjust the column spacing here
\begin{tabular}{ll}
\textbf{Year} & \textbf{Change}\\
\hline
1000 & Vowel loss between voiceless consonants in unstressed syllables \\& (p t f \texttheta\ s \textesh\ h k m n \ng) \\

\hline
2000 & Assimilation: Voiceless stop between voiced sounds become voiced \\& \{p t k\} $\rightarrow$ \{ b d g\} \\

\hline
2300 & \textschwa\ lost \\

\hline
3000 & No voiceless stops in clusters, \\& e.g., ttelama $\rightarrow$ telama \\

\hline
3500 & \textbardotlessj\ $\rightarrow$ j \\

\hline
3500 & Rhotacism GsG $\rightarrow$ GrG and GʒG $\rightarrow$ GrG \\

\hline
3501 & Rhotacism VsV $\rightarrow$ VrV to VʒV $\rightarrow$ VrV \\

\hline
4500 & No stops after fricatives \\

\hline
4501 & No stops after liquids \\

\hline
4502 & No fricative clusters \\

\hline
4503 & No stops after glides \\

\hline
5000 & h is lost between vowel and at the end of words, \\&
e.g., drihel $\rightarrow$ diel\\

\hline
5500 & ai $\rightarrow$ i / \_C, aa $\rightarrow$ a\textlengthmark\rightarrow a, ei 
$\rightarrow$ e \\

\hline
6000 & Nasal assimilation: \\& md $\rightarrow$ nd, np $\rightarrow$ mp, nk $\rightarrow$ \ng g \\

\hline
6500 & No consonant reduplication\\

\hline
7500 & Word-initial vowel loss: \\& \#V $\rightarrow$ \# \O \\

\hline
8000 & Vowel loss \texttheta Vr $\rightarrow$ \texttheta r unless stressed \\

\hline
8500 & No stops after nasals: \\& \ng k $\rightarrow$ \ng, mp $\rightarrow$ m, \ng g $\rightarrow$ g, etc. \\

\hline
8750 & No stops after any sonorant \\

\hline
9500 & Word-final vowel loss, unless stressed: \\& \#V $\rightarrow$ \O \\

\hline
10000 & ae $\rightarrow$ a \\

\hline
11000 & No coda stops: \\& e.g., \{p, b\} $\rightarrow$ m /V\_ \\

\hline
12000 & \textipa{\textyogh} $\rightarrow$ \textesh \\

\hline
12001 & \textipa{\dh} $\rightarrow$ \texttheta \\

\hline
12003, No repeated vowels\\
\hline
    12004 & No word-final e\\
\hline
    13000 & No fricative clusters\\
\hline
    14000 & No fricatives after affricates\\
\hline
   14001 & No affricates after fricatives\\
\end{tabular}
}


\newpage

\section{Phrases}
good morning

wishe-lhano yi-aray




\newpage
\section{Dictionary}
\twocolumn
\vspace{15pt}
\begin{nopagebreak}
\noindent{\fontsize{20pt}{10pt}\textbf{di-} } \textit{ABL/ablative case marker} (CASE)\\
\noindent {\tovian \fontsize{20pt}{10pt} \textbf{di-} }\\
\noindent /d{\textprimstress}i/\\


\noindent History:

\vspace{-0pt}
\hspace{40pt}
\begin{tabular}{ccc}
\textit{0} & /di-/& \\
\end{tabular}

\vspace{20pt}\hline

\end{nopagebreak}
\filbreak



\vspace{15pt}
\begin{nopagebreak}
\noindent{\fontsize{20pt}{10pt}\textbf{le-} } \textit{abstract noun class} (CLASS)\\
\noindent {\tovian \fontsize{20pt}{10pt} \textbf{le-} }\\
\noindent /l{\textprimstress}e/\\


\noindent History:

\vspace{-0pt}
\hspace{40pt}
\begin{tabular}{ccc}
\textit{0} & /le-/& \\
\end{tabular}

\vspace{20pt}\hline

\end{nopagebreak}
\filbreak



\vspace{15pt}
\begin{nopagebreak}
\noindent{\fontsize{20pt}{10pt}\textbf{yu-} } \textit{ACC/accusative case marker} (CASE)\\
\noindent {\tovian \fontsize{20pt}{10pt} \textbf{yu-} }\\
\noindent /{\textprimstress}u/\\


\noindent History:

\vspace{-0pt}
\hspace{40pt}
\begin{tabular}{ccc}
\textit{0} & /yu-/& \\
\end{tabular}

\vspace{20pt}\hline

\end{nopagebreak}
\filbreak



\vspace{15pt}
\begin{nopagebreak}
\noindent{\fontsize{20pt}{10pt}\textbf{lor} } \textit{action/doing} (N)\\
\noindent {\tovian \fontsize{20pt}{10pt} \textbf{lor} }\\
\noindent /l{\textprimstress}or/\\


\noindent History:

\vspace{-0pt}
\hspace{40pt}
\begin{tabular}{ccc}
\textit{0} & /lore/&$\rightarrow$ & \textit{9500} & /lor/& \\
\end{tabular}

\vspace{20pt}\hline

\end{nopagebreak}
\filbreak



\vspace{15pt}
\begin{nopagebreak}
\noindent{\fontsize{20pt}{10pt}\textbf{rith} } \textit{air} (N)\\
\noindent {\tovian \fontsize{20pt}{10pt} \textbf{rith} }\\
\noindent /r{\textprimstress}i{\texttheta}/\\


\noindent History:

\vspace{-0pt}
\hspace{40pt}
\begin{tabular}{ccc}
\textit{0} & /rai{\texttheta}/&$\rightarrow$ & \textit{5500} & /ri{\texttheta}/& \\
\end{tabular}

\vspace{20pt}\hline

\end{nopagebreak}
\filbreak



\vspace{15pt}
\begin{nopagebreak}
\noindent{\fontsize{20pt}{10pt}\textbf{su-} } \textit{ALL/allative case marker} (CASE)\\
\noindent {\tovian \fontsize{20pt}{10pt} \textbf{su-} }\\
\noindent /s{\textprimstress}u/\\


\noindent History:

\vspace{-0pt}
\hspace{40pt}
\begin{tabular}{ccc}
\textit{0} & /su-/& \\
\end{tabular}

\vspace{20pt}\hline

\end{nopagebreak}
\filbreak



\vspace{15pt}
\begin{nopagebreak}
\noindent{\fontsize{20pt}{10pt}\textbf{nan} } \textit{ancient/old} (N)\\
\noindent {\tovian \fontsize{20pt}{10pt} \textbf{nan} }\\
\noindent /n{\textprimstress}an/\\


\noindent History:

\vspace{-0pt}
\hspace{40pt}
\begin{tabular}{ccc}
\textit{0} & /nan/& \\
\end{tabular}

\vspace{20pt}\hline

\end{nopagebreak}
\filbreak



\vspace{15pt}
\begin{nopagebreak}
\noindent{\fontsize{20pt}{10pt}\textbf{la-} } \textit{animate noun class} (CLASS)\\
\noindent {\tovian \fontsize{20pt}{10pt} \textbf{la-} }\\
\noindent /l{\textprimstress}a/\\


\noindent History:

\vspace{-0pt}
\hspace{40pt}
\begin{tabular}{ccc}
\textit{0} & /la-/& \\
\end{tabular}

\vspace{20pt}\hline

\end{nopagebreak}
\filbreak



\vspace{15pt}
\begin{nopagebreak}
\noindent{\fontsize{20pt}{10pt}\textbf{kar} } \textit{balance} (N)\\
\noindent {\tovian \fontsize{20pt}{10pt} \textbf{kar} }\\
\noindent /k{\textprimstress}ar/\\


\noindent History:

\vspace{-0pt}
\hspace{40pt}
\begin{tabular}{ccc}
\textit{0} & /kara/&$\rightarrow$ & \textit{9500} & /kar/& \\
\end{tabular}

\vspace{20pt}\hline

\end{nopagebreak}
\filbreak



\vspace{15pt}
\begin{nopagebreak}
\noindent{\fontsize{20pt}{10pt}\textbf{wen} } \textit{beauty} (N)\\
\noindent {\tovian \fontsize{20pt}{10pt} \textbf{wen} }\\
\noindent /w{\textprimstress}en/\\


\noindent History:

\vspace{-0pt}
\hspace{40pt}
\begin{tabular}{ccc}
\textit{0} & /wen/& \\
\end{tabular}

\vspace{20pt}\hline

\end{nopagebreak}
\filbreak



\vspace{15pt}
\begin{nopagebreak}
\noindent{\fontsize{20pt}{10pt}\textbf{kthan} } \textit{becoming} (N)\\
\noindent {\tovian \fontsize{20pt}{10pt} \textbf{kthan} }\\
\noindent /k{\texttheta}{\textprimstress}an/\\
\noindent lit. arriving+change\\


\noindent History:

\vspace{-0pt}
\hspace{40pt}
\begin{tabular}{ccc}
\textit{0} & /ki{\texttheta}sakana/&$\rightarrow$ & \textit{1000} & /k{\texttheta}skana/&$\rightarrow$ & \textit{3000} & /k{\texttheta}sana/&$\rightarrow$ & \textit{4502} & /k{\texttheta}ana/&$\rightarrow$ & \textit{9500} & /k{\texttheta}an/& \\
\end{tabular}

\vspace{20pt}\hline

\end{nopagebreak}
\filbreak



\vspace{15pt}
\begin{nopagebreak}
\noindent{\fontsize{20pt}{10pt}\textbf{yalor} } \textit{beginning/starting} (N)\\
\noindent {\tovian \fontsize{20pt}{10pt} \textbf{yalor} }\\
\noindent /y{\textprimstress}alor/\\


\noindent History:

\vspace{-0pt}
\hspace{40pt}
\begin{tabular}{ccc}
\textit{0} & /yalor/& \\
\end{tabular}

\vspace{20pt}\hline

\end{nopagebreak}
\filbreak



\vspace{15pt}
\begin{nopagebreak}
\noindent{\fontsize{20pt}{10pt}\textbf{hul} } \textit{behind} (N)\\
\noindent {\tovian \fontsize{20pt}{10pt} \textbf{hul} }\\
\noindent /h{\textprimstress}ul/\\


\noindent History:

\vspace{-0pt}
\hspace{40pt}
\begin{tabular}{ccc}
\textit{0} & /hul/& \\
\end{tabular}

\vspace{20pt}\hline

\end{nopagebreak}
\filbreak



\vspace{15pt}
\begin{nopagebreak}
\noindent{\fontsize{20pt}{10pt}\textbf{sil} } \textit{being/existence/created thing} (N)\\
\noindent {\tovian \fontsize{20pt}{10pt} \textbf{sil} }\\
\noindent /s{\textprimstress}il/\\


\noindent History:

\vspace{-0pt}
\hspace{40pt}
\begin{tabular}{ccc}
\textit{0} & /sil/& \\
\end{tabular}

\vspace{20pt}\hline

\end{nopagebreak}
\filbreak



\vspace{15pt}
\begin{nopagebreak}
\noindent{\fontsize{20pt}{10pt}\textbf{mot} } \textit{big/large} (N)\\
\noindent {\tovian \fontsize{20pt}{10pt} \textbf{mot} }\\
\noindent /m{\textprimstress}o{\textsubbridge{t}}/\\


\noindent History:

\vspace{-0pt}
\hspace{40pt}
\begin{tabular}{ccc}
\textit{0} & /mo{\textsubbridge{t}}/& \\
\end{tabular}

\vspace{20pt}\hline

\end{nopagebreak}
\filbreak



\vspace{15pt}
\begin{nopagebreak}
\noindent{\fontsize{20pt}{10pt}\textbf{driks} } \textit{blood} (N)\\
\noindent {\tovian \fontsize{20pt}{10pt} \textbf{driks} }\\
\noindent /dr{\textprimstress}iks/\\


\noindent History:

\vspace{-0pt}
\hspace{40pt}
\begin{tabular}{ccc}
\textit{0} & /driks/& \\
\end{tabular}

\vspace{20pt}\hline

\end{nopagebreak}
\filbreak



\vspace{15pt}
\begin{nopagebreak}
\noindent{\fontsize{20pt}{10pt}\textbf{dor} } \textit{blueness/blue} (N)\\
\noindent {\tovian \fontsize{20pt}{10pt} \textbf{dor} }\\
\noindent /d{\textprimstress}or/\\


\noindent History:

\vspace{-0pt}
\hspace{40pt}
\begin{tabular}{ccc}
\textit{0} & /dora/&$\rightarrow$ & \textit{9500} & /dor/& \\
\end{tabular}

\vspace{20pt}\hline

\end{nopagebreak}
\filbreak



\vspace{15pt}
\begin{nopagebreak}
\noindent{\fontsize{20pt}{10pt}\textbf{selith} } \textit{boat/ship/canoe} (N)\\
\noindent {\tovian \fontsize{20pt}{10pt} \textbf{selith} }\\
\noindent /s{\textprimstress}eli{\texttheta}/\\
\noindent lit. water+traveling\\


\noindent History:

\vspace{-0pt}
\hspace{40pt}
\begin{tabular}{ccc}
\textit{3000} & /zdelki{\texttheta}sa/&$\rightarrow$ & \textit{3000} & /zdeli{\texttheta}sa/&$\rightarrow$ & \textit{4500} & /zeli{\texttheta}sa/&$\rightarrow$ & \textit{4502} & /zeli{\texttheta}a/&$\rightarrow$ & \textit{9500} & /zeli{\texttheta}/&$\rightarrow$ & \textit{12000} & /seli{\texttheta}/& \\
\end{tabular}

\vspace{20pt}\hline

\end{nopagebreak}
\filbreak



\vspace{15pt}
\begin{nopagebreak}
\noindent{\fontsize{20pt}{10pt}\textbf{sul} } \textit{bottom} (N)\\
\noindent {\tovian \fontsize{20pt}{10pt} \textbf{sul} }\\
\noindent /s{\textprimstress}ul/\\


\noindent History:

\vspace{-0pt}
\hspace{40pt}
\begin{tabular}{ccc}
\textit{0} & /zul/&$\rightarrow$ & \textit{12000} & /sul/& \\
\end{tabular}

\vspace{20pt}\hline

\end{nopagebreak}
\filbreak



\vspace{15pt}
\begin{nopagebreak}
\noindent{\fontsize{20pt}{10pt}\textbf{ke-} } \textit{CAUS/causative voice marker} (VOICE)\\
\noindent {\tovian \fontsize{20pt}{10pt} \textbf{ke-} }\\
\noindent /k{\textprimstress}e/\\


\noindent History:

\vspace{-0pt}
\hspace{40pt}
\begin{tabular}{ccc}
\textit{0} & /ke-/& \\
\end{tabular}

\vspace{20pt}\hline

\end{nopagebreak}
\filbreak



\vspace{15pt}
\begin{nopagebreak}
\noindent{\fontsize{20pt}{10pt}\textbf{lhel} } \textit{center} (N)\\
\noindent {\tovian \fontsize{20pt}{10pt} \textbf{lhel} }\\
\noindent /{\textbeltl}{\textprimstress}el/\\


\noindent History:

\vspace{-0pt}
\hspace{40pt}
\begin{tabular}{ccc}
\textit{0} & /{\textbeltl}el/& \\
\end{tabular}

\vspace{20pt}\hline

\end{nopagebreak}
\filbreak



\vspace{15pt}
\begin{nopagebreak}
\noindent{\fontsize{20pt}{10pt}\textbf{kan} } \textit{change} (N)\\
\noindent {\tovian \fontsize{20pt}{10pt} \textbf{kan} }\\
\noindent /k{\textprimstress}an/\\


\noindent History:

\vspace{-0pt}
\hspace{40pt}
\begin{tabular}{ccc}
\textit{0} & /kana/&$\rightarrow$ & \textit{9500} & /kan/& \\
\end{tabular}

\vspace{20pt}\hline

\end{nopagebreak}
\filbreak



\vspace{15pt}
\begin{nopagebreak}
\noindent{\fontsize{20pt}{10pt}\textbf{nadriel} } \textit{child} (N)\\
\noindent {\tovian \fontsize{20pt}{10pt} \textbf{nadriel} }\\
\noindent /nadr{\textprimstress}iel/\\
\noindent lit. small+person\\


\noindent History:

\vspace{-0pt}
\hspace{40pt}
\begin{tabular}{ccc}
\textit{0} & /ninadrihela/&$\rightarrow$ & \textit{5000} & /ninadriela/&$\rightarrow$ & \textit{9200} & /n{\textschwa}nadriela/&$\rightarrow$ & \textit{9500} & /n{\textschwa}nadriel/&$\rightarrow$ & \textit{11990} & /nnadriel/&$\rightarrow$ & \textit{12005} & /nadriel/& \\
\end{tabular}

\vspace{20pt}\hline

\end{nopagebreak}
\filbreak



\vspace{15pt}
\begin{nopagebreak}
\noindent{\fontsize{20pt}{10pt}\textbf{tot} } \textit{circle/cycle} (N)\\
\noindent {\tovian \fontsize{20pt}{10pt} \textbf{tot} }\\
\noindent /{\textsubbridge{t}}{\textprimstress}o{\textsubbridge{t}}/\\


\noindent History:

\vspace{-0pt}
\hspace{40pt}
\begin{tabular}{ccc}
\textit{0} & /{\textsubbridge{t}}o{\textsubbridge{t}}/& \\
\end{tabular}

\vspace{20pt}\hline

\end{nopagebreak}
\filbreak



\vspace{15pt}
\begin{nopagebreak}
\noindent{\fontsize{20pt}{10pt}\textbf{shithil} } \textit{city} (N)\\
\noindent {\tovian \fontsize{20pt}{10pt} \textbf{shithil} }\\
\noindent /{\textesh}{\textprimstress}i{\texttheta}il/\\
\noindent lit. wakefulness+place\\


\noindent History:

\vspace{-0pt}
\hspace{40pt}
\begin{tabular}{ccc}
\textit{6000} & /{\textyogh}i{\texttheta}{\texttheta}ile/&$\rightarrow$ & \textit{6500} & /{\textyogh}i{\texttheta}ile/&$\rightarrow$ & \textit{9500} & /{\textyogh}i{\texttheta}il/&$\rightarrow$ & \textit{12001} & /{\textesh}i{\texttheta}il/& \\
\end{tabular}

\vspace{20pt}\hline

\end{nopagebreak}
\filbreak



\vspace{15pt}
\begin{nopagebreak}
\noindent{\fontsize{20pt}{10pt}\textbf{tush} } \textit{coldness/cold} (N)\\
\noindent {\tovian \fontsize{20pt}{10pt} \textbf{tush} }\\
\noindent /{\textsubbridge{t}}{\textprimstress}u{\textesh}/\\


\noindent History:

\vspace{-0pt}
\hspace{40pt}
\begin{tabular}{ccc}
\textit{0} & /{\textsubbridge{t}}ku{\textesh}u/&$\rightarrow$ & \textit{3000} & /{\textsubbridge{t}}u{\textesh}u/&$\rightarrow$ & \textit{9500} & /{\textsubbridge{t}}u{\textesh}/& \\
\end{tabular}

\vspace{20pt}\hline

\end{nopagebreak}
\filbreak



\vspace{15pt}
\begin{nopagebreak}
\noindent{\fontsize{20pt}{10pt}\textbf{yi-} } \textit{COM/comlative case marker} (CASE)\\
\noindent {\tovian \fontsize{20pt}{10pt} \textbf{yi-} }\\
\noindent /{\textprimstress}i/\\


\noindent History:

\vspace{-0pt}
\hspace{40pt}
\begin{tabular}{ccc}
\textit{0} & /yi-/& \\
\end{tabular}

\vspace{20pt}\hline

\end{nopagebreak}
\filbreak



\vspace{15pt}
\begin{nopagebreak}
\noindent{\fontsize{20pt}{10pt}\textbf{roladil} } \textit{comparison} (N)\\
\noindent {\tovian \fontsize{20pt}{10pt} \textbf{roladil} }\\
\noindent /rol{\textprimstress}adil/\\


\noindent History:

\vspace{-0pt}
\hspace{40pt}
\begin{tabular}{ccc}
\textit{0} & /roladil/& \\
\end{tabular}

\vspace{20pt}\hline

\end{nopagebreak}
\filbreak



\vspace{15pt}
\begin{nopagebreak}
\noindent{\fontsize{20pt}{10pt}\textbf{om} } \textit{completeness/wholeness} (N)\\
\noindent {\tovian \fontsize{20pt}{10pt} \textbf{om} }\\
\noindent /{\textprimstress}om/\\


\noindent History:

\vspace{-0pt}
\hspace{40pt}
\begin{tabular}{ccc}
\textit{0} & /oma/&$\rightarrow$ & \textit{9500} & /om/& \\
\end{tabular}

\vspace{20pt}\hline

\end{nopagebreak}
\filbreak



\vspace{15pt}
\begin{nopagebreak}
\noindent{\fontsize{20pt}{10pt}\textbf{ko-} } \textit{COND/conditional mood marker} (MOOD)\\
\noindent {\tovian \fontsize{20pt}{10pt} \textbf{ko-} }\\
\noindent /k{\textprimstress}o/\\


\noindent History:

\vspace{-0pt}
\hspace{40pt}
\begin{tabular}{ccc}
\textit{0} & /ko-/& \\
\end{tabular}

\vspace{20pt}\hline

\end{nopagebreak}
\filbreak



\vspace{15pt}
\begin{nopagebreak}
\noindent{\fontsize{20pt}{10pt}\textbf{ilor} } \textit{conflict/fighting} (N)\\
\noindent {\tovian \fontsize{20pt}{10pt} \textbf{ilor} }\\
\noindent /{\textprimstress}ilor/\\


\noindent History:

\vspace{-0pt}
\hspace{40pt}
\begin{tabular}{ccc}
\textit{0} & /ahilor/&$\rightarrow$ & \textit{5000} & /ailor/&$\rightarrow$ & \textit{5500} & /ilor/& \\
\end{tabular}

\vspace{20pt}\hline

\end{nopagebreak}
\filbreak



\vspace{15pt}
\begin{nopagebreak}
\noindent{\fontsize{20pt}{10pt}\textbf{leth} } \textit{consumption/tasting/eating} (N)\\
\noindent {\tovian \fontsize{20pt}{10pt} \textbf{leth} }\\
\noindent /l{\textprimstress}e{\texttheta}/\\


\noindent History:

\vspace{-0pt}
\hspace{40pt}
\begin{tabular}{ccc}
\textit{0} & /ale{\dh}a/&$\rightarrow$ & \textit{7500} & /le{\dh}a/&$\rightarrow$ & \textit{9500} & /le{\dh}/&$\rightarrow$ & \textit{12002} & /le{\texttheta}/& \\
\end{tabular}

\vspace{20pt}\hline

\end{nopagebreak}
\filbreak



\vspace{15pt}
\begin{nopagebreak}
\noindent{\fontsize{20pt}{10pt}\textbf{nalhun} } \textit{council} (N)\\
\noindent {\tovian \fontsize{20pt}{10pt} \textbf{nalhun} }\\
\noindent /n{\textprimstress}a{\textbeltl}un/\\
\noindent lit. thinking+joining\\


\noindent History:

\vspace{-0pt}
\hspace{40pt}
\begin{tabular}{ccc}
\textit{0} & /na{\textbeltl}{\texttoptiebar{t\textbeltl}}una/&$\rightarrow$ & \textit{9500} & /na{\textbeltl}{\texttoptiebar{t\textbeltl}}un/&$\rightarrow$ & \textit{14001} & /na{\textbeltl}un/& \\
\end{tabular}

\vspace{20pt}\hline

\end{nopagebreak}
\filbreak



\vspace{15pt}
\begin{nopagebreak}
\noindent{\fontsize{20pt}{10pt}\textbf{ks-o} } \textit{COUNTER/counterfactual mood marker} (MOOD)\\
\noindent {\tovian \fontsize{20pt}{10pt} \textbf{ks-o} }\\
\noindent /{\textprimstress}o/\\


\noindent History:

\vspace{-0pt}
\hspace{40pt}
\begin{tabular}{ccc}
\textit{0} & /ks-o/& \\
\end{tabular}

\vspace{20pt}\hline

\end{nopagebreak}
\filbreak



\vspace{15pt}
\begin{nopagebreak}
\noindent{\fontsize{20pt}{10pt}\textbf{kwilor} } \textit{dance} (N)\\
\noindent {\tovian \fontsize{20pt}{10pt} \textbf{kwilor} }\\
\noindent /kw{\textprimstress}ilor/\\
\noindent lit. movement+action\\


\noindent History:

\vspace{-0pt}
\hspace{40pt}
\begin{tabular}{ccc}
\textit{0} & /kwillore/&$\rightarrow$ & \textit{6500} & /kwilore/&$\rightarrow$ & \textit{9500} & /kwilor/& \\
\end{tabular}

\vspace{20pt}\hline

\end{nopagebreak}
\filbreak



\vspace{15pt}
\begin{nopagebreak}
\noindent{\fontsize{20pt}{10pt}\textbf{mral} } \textit{darkness/dark} (N)\\
\noindent {\tovian \fontsize{20pt}{10pt} \textbf{mral} }\\
\noindent /mr{\textprimstress}al/\\


\noindent History:

\vspace{-0pt}
\hspace{40pt}
\begin{tabular}{ccc}
\textit{0} & /bral/&$\rightarrow$ & \textit{11000} & /mral/& \\
\end{tabular}

\vspace{20pt}\hline

\end{nopagebreak}
\filbreak



\vspace{15pt}
\begin{nopagebreak}
\noindent{\fontsize{20pt}{10pt}\textbf{mi-} } \textit{DAT/dative case marker} (CASE)\\
\noindent {\tovian \fontsize{20pt}{10pt} \textbf{mi-} }\\
\noindent /m{\textprimstress}i/\\


\noindent History:

\vspace{-0pt}
\hspace{40pt}
\begin{tabular}{ccc}
\textit{0} & /mi-/& \\
\end{tabular}

\vspace{20pt}\hline

\end{nopagebreak}
\filbreak



\vspace{15pt}
\begin{nopagebreak}
\noindent{\fontsize{20pt}{10pt}\textbf{puloan} } \textit{dawn/morning} (N)\\
\noindent {\tovian \fontsize{20pt}{10pt} \textbf{puloan} }\\
\noindent /pul{\textprimstress}oan/\\
\noindent lit. first+light\\


\noindent History:

\vspace{-0pt}
\hspace{40pt}
\begin{tabular}{ccc}
\textit{500} & /pu-loahana/&$\rightarrow$ & \textit{1000} & /puloahana/&$\rightarrow$ & \textit{5000} & /puloaana/&$\rightarrow$ & \textit{5500} & /puloana/&$\rightarrow$ & \textit{9500} & /puloan/& \\
\end{tabular}

\vspace{20pt}\hline

\end{nopagebreak}
\filbreak



\vspace{15pt}
\begin{nopagebreak}
\noindent{\fontsize{20pt}{10pt}\textbf{lhonum} } \textit{day} (N)\\
\noindent {\tovian \fontsize{20pt}{10pt} \textbf{lhonum} }\\
\noindent /{\textbeltl}{\textprimstress}onum/\\
\noindent \textit{Period of time}\\


\noindent History:

\vspace{-0pt}
\hspace{40pt}
\begin{tabular}{ccc}
\textit{0} & /{\textbeltl}onum/& \\
\end{tabular}

\vspace{20pt}\hline

\end{nopagebreak}
\filbreak



\vspace{15pt}
\begin{nopagebreak}
\noindent{\fontsize{20pt}{10pt}\textbf{shil} } \textit{desert} (N)\\
\noindent {\tovian \fontsize{20pt}{10pt} \textbf{shil} }\\
\noindent /{\textesh}{\textprimstress}il/\\
\noindent lit. sand+place\\


\noindent History:

\vspace{-0pt}
\hspace{40pt}
\begin{tabular}{ccc}
\textit{1000} & /{\textesh}o{\texttoptiebar{t\textbeltl}}{\texttheta}k{\textesh}{\textsubbridge{t}}{\texttheta}ile/&$\rightarrow$ & \textit{1000} & /{\textesh}{\texttoptiebar{t\textbeltl}}{\texttheta}k{\textesh}{\textsubbridge{t}}{\texttheta}ile/&$\rightarrow$ & \textit{3000} & /{\textesh}{\texttoptiebar{t\textbeltl}}{\texttheta}k{\textesh}{\texttheta}ile/&$\rightarrow$ & \textit{4500} & /{\textesh}{\texttoptiebar{t\textbeltl}}{\texttheta}{\textesh}{\texttheta}ile/&$\rightarrow$ & \textit{4502} & /{\textesh}{\texttoptiebar{t\textbeltl}}{\texttheta}{\texttheta}ile/&$\rightarrow$ & \textit{6500} & /{\textesh}{\texttoptiebar{t\textbeltl}}{\texttheta}ile/&$\rightarrow$ & \textit{9500} & /{\textesh}{\texttoptiebar{t\textbeltl}}{\texttheta}il/&$\rightarrow$ & \textit{14000} & /{\textesh}{\texttoptiebar{t\textbeltl}}il/&$\rightarrow$ & \textit{14001} & /{\textesh}il/& \\
\end{tabular}

\vspace{20pt}\hline

\end{nopagebreak}
\filbreak



\vspace{15pt}
\begin{nopagebreak}
\noindent{\fontsize{20pt}{10pt}\textbf{silary} } \textit{discovery} (N)\\
\noindent {\tovian \fontsize{20pt}{10pt} \textbf{silary} }\\
\noindent /s{\textprimstress}ilary/\\
\noindent lit. make+wisdom\\


\noindent History:

\vspace{-0pt}
\hspace{40pt}
\begin{tabular}{ccc}
\textit{50} & /silahe{\textsubbridge{t}}aray/&$\rightarrow$ & \textit{1000} & /silah{\textsubbridge{t}}aray/&$\rightarrow$ & \textit{2100} & /silah{\textsubbridge{t}}ary/&$\rightarrow$ & \textit{3000} & /silahary/&$\rightarrow$ & \textit{5000} & /silaary/&$\rightarrow$ & \textit{5500} & /silary/& \\
\end{tabular}

\vspace{20pt}\hline

\end{nopagebreak}
\filbreak



\vspace{15pt}
\begin{nopagebreak}
\noindent{\fontsize{20pt}{10pt}\textbf{norin} } \textit{dog} (N)\\
\noindent {\tovian \fontsize{20pt}{10pt} \textbf{norin} }\\
\noindent /n{\textprimstress}orin/\\


\noindent History:

\vspace{-0pt}
\hspace{40pt}
\begin{tabular}{ccc}
\textit{0} & /am{\textsubbridge{t}}orin/&$\rightarrow$ & \textit{2000} & /amdorin/&$\rightarrow$ & \textit{6000} & /andorin/&$\rightarrow$ & \textit{7500} & /ndorin/&$\rightarrow$ & \textit{8750} & /norin/& \\
\end{tabular}

\vspace{20pt}\hline

\end{nopagebreak}
\filbreak



\vspace{15pt}
\begin{nopagebreak}
\noindent{\fontsize{20pt}{10pt}\textbf{thok} } \textit{down} (N)\\
\noindent {\tovian \fontsize{20pt}{10pt} \textbf{thok} }\\
\noindent /{\texttheta}{\textprimstress}ok/\\


\noindent History:

\vspace{-0pt}
\hspace{40pt}
\begin{tabular}{ccc}
\textit{0} & /{\texttheta}ok/& \\
\end{tabular}

\vspace{20pt}\hline

\end{nopagebreak}
\filbreak



\vspace{15pt}
\begin{nopagebreak}
\noindent{\fontsize{20pt}{10pt}\textbf{lhen} } \textit{dream} (N)\\
\noindent {\tovian \fontsize{20pt}{10pt} \textbf{lhen} }\\
\noindent /{\textbeltl}{\textprimstress}en/\\


\noindent History:

\vspace{-0pt}
\hspace{40pt}
\begin{tabular}{ccc}
\textit{0} & /{\textbeltl}ena/&$\rightarrow$ & \textit{9500} & /{\textbeltl}en/& \\
\end{tabular}

\vspace{20pt}\hline

\end{nopagebreak}
\filbreak



\vspace{15pt}
\begin{nopagebreak}
\noindent{\fontsize{20pt}{10pt}\textbf{thek} } \textit{ear} (N)\\
\noindent {\tovian \fontsize{20pt}{10pt} \textbf{thek} }\\
\noindent /{\texttheta}{\textprimstress}ek/\\


\noindent History:

\vspace{-0pt}
\hspace{40pt}
\begin{tabular}{ccc}
\textit{0} & /{\texttheta}ek/& \\
\end{tabular}

\vspace{20pt}\hline

\end{nopagebreak}
\filbreak



\vspace{15pt}
\begin{nopagebreak}
\noindent{\fontsize{20pt}{10pt}\textbf{tal} } \textit{east} (N)\\
\noindent {\tovian \fontsize{20pt}{10pt} \textbf{tal} }\\
\noindent /{\textsubbridge{t}}{\textprimstress}al/\\


\noindent History:

\vspace{-0pt}
\hspace{40pt}
\begin{tabular}{ccc}
\textit{0} & /{\textsubbridge{t}}al/& \\
\end{tabular}

\vspace{20pt}\hline

\end{nopagebreak}
\filbreak



\vspace{15pt}
\begin{nopagebreak}
\noindent{\fontsize{20pt}{10pt}\textbf{ath} } \textit{eight/8} (N)\\
\noindent {\tovian \fontsize{20pt}{10pt} \textbf{ath} }\\
\noindent /{\textprimstress}a{\texttheta}/\\


\noindent History:

\vspace{-0pt}
\hspace{40pt}
\begin{tabular}{ccc}
\textit{0} & /a{\texttheta}/& \\
\end{tabular}

\vspace{20pt}\hline

\end{nopagebreak}
\filbreak



\vspace{15pt}
\begin{nopagebreak}
\noindent{\fontsize{20pt}{10pt}\textbf{bear} } \textit{eleven/11} (N)\\
\noindent {\tovian \fontsize{20pt}{10pt} \textbf{bear} }\\
\noindent /b{\textprimstress}ear/\\


\noindent History:

\vspace{-0pt}
\hspace{40pt}
\begin{tabular}{ccc}
\textit{0} & /beyar/&$\rightarrow$ & \textit{1000} & /bear/& \\
\end{tabular}

\vspace{20pt}\hline

\end{nopagebreak}
\filbreak



\vspace{15pt}
\begin{nopagebreak}
\noindent{\fontsize{20pt}{10pt}\textbf{remen} } \textit{empty/hollow} (N)\\
\noindent {\tovian \fontsize{20pt}{10pt} \textbf{remen} }\\
\noindent /r{\textprimstress}emen/\\


\noindent History:

\vspace{-0pt}
\hspace{40pt}
\begin{tabular}{ccc}
\textit{0} & /oremen/&$\rightarrow$ & \textit{7500} & /remen/& \\
\end{tabular}

\vspace{20pt}\hline

\end{nopagebreak}
\filbreak



\vspace{15pt}
\begin{nopagebreak}
\noindent{\fontsize{20pt}{10pt}\textbf{panor} } \textit{end/final} (N)\\
\noindent {\tovian \fontsize{20pt}{10pt} \textbf{panor} }\\
\noindent /p{\textprimstress}anor/\\


\noindent History:

\vspace{-0pt}
\hspace{40pt}
\begin{tabular}{ccc}
\textit{0} & /panor/& \\
\end{tabular}

\vspace{20pt}\hline

\end{nopagebreak}
\filbreak



\vspace{15pt}
\begin{nopagebreak}
\noindent{\fontsize{20pt}{10pt}\textbf{pu-} } \textit{ESS/essive case marker} (CASE)\\
\noindent {\tovian \fontsize{20pt}{10pt} \textbf{pu-} }\\
\noindent /p{\textprimstress}u/\\


\noindent History:

\vspace{-0pt}
\hspace{40pt}
\begin{tabular}{ccc}
\textit{0} & /pu-/& \\
\end{tabular}

\vspace{20pt}\hline

\end{nopagebreak}
\filbreak



\vspace{15pt}
\begin{nopagebreak}
\noindent{\fontsize{20pt}{10pt}\textbf{yal} } \textit{evil/wickedness/badness} (N)\\
\noindent {\tovian \fontsize{20pt}{10pt} \textbf{yal} }\\
\noindent /y{\textprimstress}al/\\


\noindent History:

\vspace{-0pt}
\hspace{40pt}
\begin{tabular}{ccc}
\textit{0} & /oyal/&$\rightarrow$ & \textit{2100} & /yal/& \\
\end{tabular}

\vspace{20pt}\hline

\end{nopagebreak}
\filbreak



\vspace{15pt}
\begin{nopagebreak}
\noindent{\fontsize{20pt}{10pt}\textbf{mril} } \textit{eye/sight/seeing/watching} (N)\\
\noindent {\tovian \fontsize{20pt}{10pt} \textbf{mril} }\\
\noindent /mr{\textprimstress}il/\\


\noindent History:

\vspace{-0pt}
\hspace{40pt}
\begin{tabular}{ccc}
\textit{0} & /em{\textschwa}rila/&$\rightarrow$ & \textit{2300} & /emrila/&$\rightarrow$ & \textit{7500} & /mrila/&$\rightarrow$ & \textit{9500} & /mril/& \\
\end{tabular}

\vspace{20pt}\hline

\end{nopagebreak}
\filbreak



\vspace{15pt}
\begin{nopagebreak}
\noindent{\fontsize{20pt}{10pt}\textbf{paner} } \textit{father} (N)\\
\noindent {\tovian \fontsize{20pt}{10pt} \textbf{paner} }\\
\noindent /p{\textprimstress}aner/\\


\noindent History:

\vspace{-0pt}
\hspace{40pt}
\begin{tabular}{ccc}
\textit{0} & /pander/&$\rightarrow$ & \textit{8750} & /paner/& \\
\end{tabular}

\vspace{20pt}\hline

\end{nopagebreak}
\filbreak



\vspace{15pt}
\begin{nopagebreak}
\noindent{\fontsize{20pt}{10pt}\textbf{ary} } \textit{feather} (N)\\
\noindent {\tovian \fontsize{20pt}{10pt} \textbf{ary} }\\
\noindent /{\textprimstress}ary/\\


\noindent History:

\vspace{-0pt}
\hspace{40pt}
\begin{tabular}{ccc}
\textit{0} & /aruya/&$\rightarrow$ & \textit{2100} & /arya/&$\rightarrow$ & \textit{9500} & /ary/& \\
\end{tabular}

\vspace{20pt}\hline

\end{nopagebreak}
\filbreak



\vspace{15pt}
\begin{nopagebreak}
\noindent{\fontsize{20pt}{10pt}\textbf{lhiafem} } \textit{fifteen/15} (N)\\
\noindent {\tovian \fontsize{20pt}{10pt} \textbf{lhiafem} }\\
\noindent /{\textbeltl}i{\textprimstress}afem/\\
\noindent lit. twelve+three\\


\noindent History:

\vspace{-0pt}
\hspace{40pt}
\begin{tabular}{ccc}
\textit{4000} & /{\textbeltl}iafem/& \\
\end{tabular}

\vspace{20pt}\hline

\end{nopagebreak}
\filbreak



\vspace{15pt}
\begin{nopagebreak}
\noindent{\fontsize{20pt}{10pt}\textbf{kesh} } \textit{fire/burning} (N)\\
\noindent {\tovian \fontsize{20pt}{10pt} \textbf{kesh} }\\
\noindent /k{\textprimstress}e{\textesh}/\\


\noindent History:

\vspace{-0pt}
\hspace{40pt}
\begin{tabular}{ccc}
\textit{0} & /ke{\textesh}/& \\
\end{tabular}

\vspace{20pt}\hline

\end{nopagebreak}
\filbreak



\vspace{15pt}
\begin{nopagebreak}
\noindent{\fontsize{20pt}{10pt}\textbf{pul} } \textit{first} (N)\\
\noindent {\tovian \fontsize{20pt}{10pt} \textbf{pul} }\\
\noindent /p{\textprimstress}ul/\\
\noindent lit. ESS+one\\


\noindent History:

\vspace{-0pt}
\hspace{40pt}
\begin{tabular}{ccc}
\textit{0} & /pu-lo/&$\rightarrow$ & \textit{1000} & /pulo/&$\rightarrow$ & \textit{9500} & /pul/& \\
\end{tabular}

\vspace{20pt}\hline

\end{nopagebreak}
\filbreak



\vspace{15pt}
\begin{nopagebreak}
\noindent{\fontsize{20pt}{10pt}\textbf{rem} } \textit{five/5} (N)\\
\noindent {\tovian \fontsize{20pt}{10pt} \textbf{rem} }\\
\noindent /r{\textprimstress}em/\\


\noindent History:

\vspace{-0pt}
\hspace{40pt}
\begin{tabular}{ccc}
\textit{0} & /rep/&$\rightarrow$ & \textit{11000} & /rem/& \\
\end{tabular}

\vspace{20pt}\hline

\end{nopagebreak}
\filbreak



\vspace{15pt}
\begin{nopagebreak}
\noindent{\fontsize{20pt}{10pt}\textbf{tum} } \textit{flatness} (N)\\
\noindent {\tovian \fontsize{20pt}{10pt} \textbf{tum} }\\
\noindent /{\textsubbridge{t}}{\textprimstress}um/\\


\noindent History:

\vspace{-0pt}
\hspace{40pt}
\begin{tabular}{ccc}
\textit{0} & /{\textsubbridge{t}}um/& \\
\end{tabular}

\vspace{20pt}\hline

\end{nopagebreak}
\filbreak



\vspace{15pt}
\begin{nopagebreak}
\noindent{\fontsize{20pt}{10pt}\textbf{thoksh} } \textit{floor/ground} (N)\\
\noindent {\tovian \fontsize{20pt}{10pt} \textbf{thoksh} }\\
\noindent /{\texttheta}{\textprimstress}ok{\textesh}/\\
\noindent lit. down+foot\\


\noindent History:

\vspace{-0pt}
\hspace{40pt}
\begin{tabular}{ccc}
\textit{10} & /{\texttheta}ok{\textesh}o{\textsubbridge{t}}/&$\rightarrow$ & \textit{1000} & /{\texttheta}ok{\textesh}{\textsubbridge{t}}/&$\rightarrow$ & \textit{3000} & /{\texttheta}ok{\textesh}/& \\
\end{tabular}

\vspace{20pt}\hline

\end{nopagebreak}
\filbreak



\vspace{15pt}
\begin{nopagebreak}
\noindent{\fontsize{20pt}{10pt}\textbf{shotl} } \textit{flow} (N)\\
\noindent {\tovian \fontsize{20pt}{10pt} \textbf{shotl} }\\
\noindent /{\textesh}{\textprimstress}o{\texttoptiebar{t\textbeltl}}/\\


\noindent History:

\vspace{-0pt}
\hspace{40pt}
\begin{tabular}{ccc}
\textit{0} & /{\textesh}o{\texttoptiebar{t\textbeltl}}/& \\
\end{tabular}

\vspace{20pt}\hline

\end{nopagebreak}
\filbreak



\vspace{15pt}
\begin{nopagebreak}
\noindent{\fontsize{20pt}{10pt}\textbf{drem} } \textit{flower} (N)\\
\noindent {\tovian \fontsize{20pt}{10pt} \textbf{drem} }\\
\noindent /dr{\textprimstress}em/\\


\noindent History:

\vspace{-0pt}
\hspace{40pt}
\begin{tabular}{ccc}
\textit{0} & /drenpok/&$\rightarrow$ & \textit{1000} & /drenpk/&$\rightarrow$ & \textit{3000} & /drenp/&$\rightarrow$ & \textit{6000} & /dremp/&$\rightarrow$ & \textit{8500} & /drem/& \\
\end{tabular}

\vspace{20pt}\hline

\end{nopagebreak}
\filbreak



\vspace{15pt}
\begin{nopagebreak}
\noindent{\fontsize{20pt}{10pt}\textbf{shotl} } \textit{flowing} (N)\\
\noindent {\tovian \fontsize{20pt}{10pt} \textbf{shotl} }\\
\noindent /{\textesh}{\textprimstress}o{\texttoptiebar{t\textbeltl}}/\\


\noindent History:

\vspace{-0pt}
\hspace{40pt}
\begin{tabular}{ccc}
\textit{0} & /{\textesh}o{\texttoptiebar{t\textbeltl}}/& \\
\end{tabular}

\vspace{20pt}\hline

\end{nopagebreak}
\filbreak



\vspace{15pt}
\begin{nopagebreak}
\noindent{\fontsize{20pt}{10pt}\textbf{sialeth} } \textit{food} (N)\\
\noindent {\tovian \fontsize{20pt}{10pt} \textbf{sialeth} }\\
\noindent /si{\textprimstress}ale{\texttheta}/\\
\noindent lit. INS+eating\\


\noindent History:

\vspace{-0pt}
\hspace{40pt}
\begin{tabular}{ccc}
\textit{10} & /si-ale{\dh}a/&$\rightarrow$ & \textit{1000} & /siale{\dh}a/&$\rightarrow$ & \textit{9500} & /siale{\dh}/&$\rightarrow$ & \textit{12002} & /siale{\texttheta}/& \\
\end{tabular}

\vspace{20pt}\hline

\end{nopagebreak}
\filbreak



\vspace{15pt}
\begin{nopagebreak}
\noindent{\fontsize{20pt}{10pt}\textbf{shot} } \textit{foot} (N)\\
\noindent {\tovian \fontsize{20pt}{10pt} \textbf{shot} }\\
\noindent /{\textesh}{\textprimstress}o{\textsubbridge{t}}/\\
\noindent \textit{_            word-wrap: break-word;}\\


\noindent History:

\vspace{-0pt}
\hspace{40pt}
\begin{tabular}{ccc}
\textit{0} & /{\textesh}o{\textsubbridge{t}}/& \\
\end{tabular}

\vspace{20pt}\hline

\end{nopagebreak}
\filbreak



\vspace{15pt}
\begin{nopagebreak}
\noindent{\fontsize{20pt}{10pt}\textbf{glarethil} } \textit{forest} (N)\\
\noindent {\tovian \fontsize{20pt}{10pt} \textbf{glarethil} }\\
\noindent /glar{\textprimstress}e{\texttheta}il/\\
\noindent lit. tree+place\\


\noindent History:

\vspace{-0pt}
\hspace{40pt}
\begin{tabular}{ccc}
\textit{1000} & /aklare{\texttheta}ile/&$\rightarrow$ & \textit{2000} & /aglare{\texttheta}ile/&$\rightarrow$ & \textit{7500} & /glare{\texttheta}ile/&$\rightarrow$ & \textit{9500} & /glare{\texttheta}il/& \\
\end{tabular}

\vspace{20pt}\hline

\end{nopagebreak}
\filbreak



\vspace{15pt}
\begin{nopagebreak}
\noindent{\fontsize{20pt}{10pt}\textbf{tevi} } \textit{forty eight/48} (N)\\
\noindent {\tovian \fontsize{20pt}{10pt} \textbf{tevi} }\\
\noindent /{\textsubbridge{t}}{\textprimstress}evi/\\
\noindent lit. four+twelve\\


\noindent History:

\vspace{-0pt}
\hspace{40pt}
\begin{tabular}{ccc}
\textit{4000} & /{\textsubbridge{t}}ev{\textbeltl}ia/&$\rightarrow$ & \textit{4502} & /{\textsubbridge{t}}evia/&$\rightarrow$ & \textit{9500} & /{\textsubbridge{t}}evi/& \\
\end{tabular}

\vspace{20pt}\hline

\end{nopagebreak}
\filbreak



\vspace{15pt}
\begin{nopagebreak}
\noindent{\fontsize{20pt}{10pt}\textbf{tev} } \textit{four/4} (N)\\
\noindent {\tovian \fontsize{20pt}{10pt} \textbf{tev} }\\
\noindent /{\textsubbridge{t}}{\textprimstress}ev/\\


\noindent History:

\vspace{-0pt}
\hspace{40pt}
\begin{tabular}{ccc}
\textit{0} & /{\textsubbridge{t}}ev/& \\
\end{tabular}

\vspace{20pt}\hline

\end{nopagebreak}
\filbreak



\vspace{15pt}
\begin{nopagebreak}
\noindent{\fontsize{20pt}{10pt}\textbf{lhialo} } \textit{fourteen/14} (N)\\
\noindent {\tovian \fontsize{20pt}{10pt} \textbf{lhialo} }\\
\noindent /{\textbeltl}i{\textprimstress}alo/\\
\noindent lit. twelve+two\\


\noindent History:

\vspace{-0pt}
\hspace{40pt}
\begin{tabular}{ccc}
\textit{4000} & /{\textbeltl}ialoe/&$\rightarrow$ & \textit{9500} & /{\textbeltl}ialo/& \\
\end{tabular}

\vspace{20pt}\hline

\end{nopagebreak}
\filbreak



\vspace{15pt}
\begin{nopagebreak}
\noindent{\fontsize{20pt}{10pt}\textbf{heth} } \textit{front} (N)\\
\noindent {\tovian \fontsize{20pt}{10pt} \textbf{heth} }\\
\noindent /h{\textprimstress}e{\texttheta}/\\


\noindent History:

\vspace{-0pt}
\hspace{40pt}
\begin{tabular}{ccc}
\textit{0} & /he{\texttheta}/& \\
\end{tabular}

\vspace{20pt}\hline

\end{nopagebreak}
\filbreak



\vspace{15pt}
\begin{nopagebreak}
\noindent{\fontsize{20pt}{10pt}\textbf{luraolan} } \textit{frown} (N)\\
\noindent {\tovian \fontsize{20pt}{10pt} \textbf{luraolan} }\\
\noindent /lura{\textprimstress}olan/\\
\noindent lit. sad+mouth\\


\noindent History:

\vspace{-0pt}
\hspace{40pt}
\begin{tabular}{ccc}
\textit{400} & /luraeolan/&$\rightarrow$ & \textit{10000} & /luraolan/& \\
\end{tabular}

\vspace{20pt}\hline

\end{nopagebreak}
\filbreak



\vspace{15pt}
\begin{nopagebreak}
\noindent{\fontsize{20pt}{10pt}\textbf{doremen} } \textit{full} (N)\\
\noindent {\tovian \fontsize{20pt}{10pt} \textbf{doremen} }\\
\noindent /dor{\textprimstress}emen/\\
\noindent lit. not+empty\\


\noindent History:

\vspace{-0pt}
\hspace{40pt}
\begin{tabular}{ccc}
\textit{0} & /dooremen/&$\rightarrow$ & \textit{12003} & /doremen/& \\
\end{tabular}

\vspace{20pt}\hline

\end{nopagebreak}
\filbreak



\vspace{15pt}
\begin{nopagebreak}
\noindent{\fontsize{20pt}{10pt}\textbf{hethnum} } \textit{future} (Ns)\\
\noindent {\tovian \fontsize{20pt}{10pt} \textbf{hethnum} }\\
\noindent /h{\textprimstress}e{\texttheta}num/\\
\noindent lit. front+path\\


\noindent History:

\vspace{-0pt}
\hspace{40pt}
\begin{tabular}{ccc}
\textit{10} & /he{\texttheta}num/& \\
\end{tabular}

\vspace{20pt}\hline

\end{nopagebreak}
\filbreak



\vspace{15pt}
\begin{nopagebreak}
\noindent{\fontsize{20pt}{10pt}\textbf{hi-} } \textit{GEN/genitive case marker} (CASE)\\
\noindent {\tovian \fontsize{20pt}{10pt} \textbf{hi-} }\\
\noindent /h{\textprimstress}i/\\


\noindent History:

\vspace{-0pt}
\hspace{40pt}
\begin{tabular}{ccc}
\textit{0} & /hi-/& \\
\end{tabular}

\vspace{20pt}\hline

\end{nopagebreak}
\filbreak



\vspace{15pt}
\begin{nopagebreak}
\noindent{\fontsize{20pt}{10pt}\textbf{elakorim} } \textit{god of memory} (N)\\
\noindent {\tovian \fontsize{20pt}{10pt} \textbf{elakorim} }\\
\noindent /elak{\textprimstress}orim/\\


\noindent History:

\vspace{-0pt}
\hspace{40pt}
\begin{tabular}{ccc}
\textit{0} & /elakorim/& \\
\end{tabular}

\vspace{20pt}\hline

\end{nopagebreak}
\filbreak



\vspace{15pt}
\begin{nopagebreak}
\noindent{\fontsize{20pt}{10pt}\textbf{lhum} } \textit{goodness/virtue} (N)\\
\noindent {\tovian \fontsize{20pt}{10pt} \textbf{lhum} }\\
\noindent /{\textbeltl}{\textprimstress}um/\\


\noindent History:

\vspace{-0pt}
\hspace{40pt}
\begin{tabular}{ccc}
\textit{0} & /{\textbeltl}um/& \\
\end{tabular}

\vspace{20pt}\hline

\end{nopagebreak}
\filbreak



\vspace{15pt}
\begin{nopagebreak}
\noindent{\fontsize{20pt}{10pt}\textbf{naner} } \textit{grandfather} (N)\\
\noindent {\tovian \fontsize{20pt}{10pt} \textbf{naner} }\\
\noindent /n{\textprimstress}aner/\\
\noindent lit. old+father\\


\noindent History:

\vspace{-0pt}
\hspace{40pt}
\begin{tabular}{ccc}
\textit{642} & /nanpander/&$\rightarrow$ & \textit{2000} & /nanbander/&$\rightarrow$ & \textit{8750} & /nananer/&$\rightarrow$ & \textit{9200} & /n{\textschwa}naner/&$\rightarrow$ & \textit{11990} & /nnaner/&$\rightarrow$ & \textit{12005} & /naner/& \\
\end{tabular}

\vspace{20pt}\hline

\end{nopagebreak}
\filbreak



\vspace{15pt}
\begin{nopagebreak}
\noindent{\fontsize{20pt}{10pt}\textbf{lothr} } \textit{grasp/grabbing} (N)\\
\noindent {\tovian \fontsize{20pt}{10pt} \textbf{lothr} }\\
\noindent /l{\textprimstress}o{\texttheta}r/\\


\noindent History:

\vspace{-0pt}
\hspace{40pt}
\begin{tabular}{ccc}
\textit{0} & /lo{\texttheta}ir/&$\rightarrow$ & \textit{8000} & /lo{\texttheta}r/& \\
\end{tabular}

\vspace{20pt}\hline

\end{nopagebreak}
\filbreak



\vspace{15pt}
\begin{nopagebreak}
\noindent{\fontsize{20pt}{10pt}\textbf{klat} } \textit{greenness/green} (N)\\
\noindent {\tovian \fontsize{20pt}{10pt} \textbf{klat} }\\
\noindent /kl{\textprimstress}a{\textsubbridge{t}}/\\


\noindent History:

\vspace{-0pt}
\hspace{40pt}
\begin{tabular}{ccc}
\textit{0} & /kla{\textsubbridge{t}}/& \\
\end{tabular}

\vspace{20pt}\hline

\end{nopagebreak}
\filbreak



\vspace{15pt}
\begin{nopagebreak}
\noindent{\fontsize{20pt}{10pt}\textbf{malur} } \textit{hair/person hair} (N)\\
\noindent {\tovian \fontsize{20pt}{10pt} \textbf{malur} }\\
\noindent /m{\textprimstress}alur/\\


\noindent History:

\vspace{-0pt}
\hspace{40pt}
\begin{tabular}{ccc}
\textit{0} & /amalur/&$\rightarrow$ & \textit{7500} & /malur/& \\
\end{tabular}

\vspace{20pt}\hline

\end{nopagebreak}
\filbreak



\vspace{15pt}
\begin{nopagebreak}
\noindent{\fontsize{20pt}{10pt}\textbf{thang} } \textit{hand/leaf} (N)\\
\noindent {\tovian \fontsize{20pt}{10pt} \textbf{thang} }\\
\noindent /{\texttheta}{\textprimstress}a{\ng}/\\


\noindent History:

\vspace{-0pt}
\hspace{40pt}
\begin{tabular}{ccc}
\textit{0} & /a{\dh}ange/&$\rightarrow$ & \textit{6000} & /a{\dh}a{\ng}ge/&$\rightarrow$ & \textit{7500} & /{\dh}a{\ng}ge/&$\rightarrow$ & \textit{8500} & /{\dh}a{\ng}e/&$\rightarrow$ & \textit{9500} & /{\dh}a{\ng}/&$\rightarrow$ & \textit{12002} & /{\texttheta}a{\ng}/& \\
\end{tabular}

\vspace{20pt}\hline

\end{nopagebreak}
\filbreak



\vspace{15pt}
\begin{nopagebreak}
\noindent{\fontsize{20pt}{10pt}\textbf{mat} } \textit{happy} (N)\\
\noindent {\tovian \fontsize{20pt}{10pt} \textbf{mat} }\\
\noindent /m{\textprimstress}a{\textsubbridge{t}}/\\


\noindent History:

\vspace{-0pt}
\hspace{40pt}
\begin{tabular}{ccc}
\textit{0} & /ma{\textsubbridge{t}}{\textsubbridge{t}}a/&$\rightarrow$ & \textit{3000} & /ma{\textsubbridge{t}}a/&$\rightarrow$ & \textit{9500} & /ma{\textsubbridge{t}}/& \\
\end{tabular}

\vspace{20pt}\hline

\end{nopagebreak}
\filbreak



\vspace{15pt}
\begin{nopagebreak}
\noindent{\fontsize{20pt}{10pt}\textbf{darunenan} } \textit{hat} (N)\\
\noindent {\tovian \fontsize{20pt}{10pt} \textbf{darunenan} }\\
\noindent /darun{\textprimstress}enan/\\
\noindent lit. head+covering\\


\noindent History:

\vspace{-0pt}
\hspace{40pt}
\begin{tabular}{ccc}
\textit{3024} & /darunbenani/&$\rightarrow$ & \textit{8750} & /darunenani/&$\rightarrow$ & \textit{9500} & /darunenan/& \\
\end{tabular}

\vspace{20pt}\hline

\end{nopagebreak}
\filbreak



\vspace{15pt}
\begin{nopagebreak}
\noindent{\fontsize{20pt}{10pt}\textbf{ta} } \textit{he/she/them} (P)\\
\noindent {\tovian \fontsize{20pt}{10pt} \textbf{ta} }\\
\noindent /{\textsubbridge{t}}{\textprimstress}a/\\


\noindent History:

\vspace{-0pt}
\hspace{40pt}
\begin{tabular}{ccc}
\textit{0} & /{\textsubbridge{t}}a/& \\
\end{tabular}

\vspace{20pt}\hline

\end{nopagebreak}
\filbreak



\vspace{15pt}
\begin{nopagebreak}
\noindent{\fontsize{20pt}{10pt}\textbf{darun} } \textit{head} (N)\\
\noindent {\tovian \fontsize{20pt}{10pt} \textbf{darun} }\\
\noindent /d{\textprimstress}arun/\\


\noindent History:

\vspace{-0pt}
\hspace{40pt}
\begin{tabular}{ccc}
\textit{0} & /darun/& \\
\end{tabular}

\vspace{20pt}\hline

\end{nopagebreak}
\filbreak



\vspace{15pt}
\begin{nopagebreak}
\noindent{\fontsize{20pt}{10pt}\textbf{bum} } \textit{hiding} ()\\
\noindent {\tovian \fontsize{20pt}{10pt} \textbf{bum} }\\
\noindent /b{\textprimstress}um/\\
\noindent lit. bbu\\


\noindent History:

\vspace{-0pt}
\hspace{40pt}
\begin{tabular}{ccc}
\textit{220} & /bubu/&$\rightarrow$ & \textit{9500} & /bub/&$\rightarrow$ & \textit{11000} & /bum/& \\
\end{tabular}

\vspace{20pt}\hline

\end{nopagebreak}
\filbreak



\vspace{15pt}
\begin{nopagebreak}
\noindent{\fontsize{20pt}{10pt}\textbf{melorethan} } \textit{hoe} (N)\\
\noindent {\tovian \fontsize{20pt}{10pt} \textbf{melorethan} }\\
\noindent /melor{\textprimstress}e{\texttheta}an/\\
\noindent lit. plant+tool\\


\noindent History:

\vspace{-0pt}
\hspace{40pt}
\begin{tabular}{ccc}
\textit{2000} & /{\textschwa}mellore{\dh}an/&$\rightarrow$ & \textit{2300} & /mellore{\dh}an/&$\rightarrow$ & \textit{6500} & /melore{\dh}an/&$\rightarrow$ & \textit{12002} & /melore{\texttheta}an/& \\
\end{tabular}

\vspace{20pt}\hline

\end{nopagebreak}
\filbreak



\vspace{15pt}
\begin{nopagebreak}
\noindent{\fontsize{20pt}{10pt}\textbf{drielathil} } \textit{house/home} (N)\\
\noindent {\tovian \fontsize{20pt}{10pt} \textbf{drielathil} }\\
\noindent /driel{\textprimstress}a{\texttheta}il/\\
\noindent lit. person+place\\


\noindent History:

\vspace{-0pt}
\hspace{40pt}
\begin{tabular}{ccc}
\textit{10} & /drihela{\texttheta}ile/&$\rightarrow$ & \textit{5000} & /driela{\texttheta}ile/&$\rightarrow$ & \textit{9500} & /driela{\texttheta}il/& \\
\end{tabular}

\vspace{20pt}\hline

\end{nopagebreak}
\filbreak



\vspace{15pt}
\begin{nopagebreak}
\noindent{\fontsize{20pt}{10pt}\textbf{lhan} } \textit{how} (INT)\\
\noindent {\tovian \fontsize{20pt}{10pt} \textbf{lhan} }\\
\noindent /{\textbeltl}{\textprimstress}an/\\


\noindent History:

\vspace{-0pt}
\hspace{40pt}
\begin{tabular}{ccc}
\textit{0} & /{\textbeltl}an/& \\
\end{tabular}

\vspace{20pt}\hline

\end{nopagebreak}
\filbreak



\vspace{15pt}
\begin{nopagebreak}
\noindent{\fontsize{20pt}{10pt}\textbf{shush} } \textit{ice/frost} (N)\\
\noindent {\tovian \fontsize{20pt}{10pt} \textbf{shush} }\\
\noindent /{\textesh}{\textprimstress}u{\textesh}/\\


\noindent History:

\vspace{-0pt}
\hspace{40pt}
\begin{tabular}{ccc}
\textit{0} & /{\textesh}u{\textesh}a/&$\rightarrow$ & \textit{9500} & /{\textesh}u{\textesh}/& \\
\end{tabular}

\vspace{20pt}\hline

\end{nopagebreak}
\filbreak



\vspace{15pt}
\begin{nopagebreak}
\noindent{\fontsize{20pt}{10pt}\textbf{se-} } \textit{IMP/imperative mood marker} (MOOD)\\
\noindent {\tovian \fontsize{20pt}{10pt} \textbf{se-} }\\
\noindent /s{\textprimstress}e/\\


\noindent History:

\vspace{-0pt}
\hspace{40pt}
\begin{tabular}{ccc}
\textit{0} & /se-/& \\
\end{tabular}

\vspace{20pt}\hline

\end{nopagebreak}
\filbreak



\vspace{15pt}
\begin{nopagebreak}
\noindent{\fontsize{20pt}{10pt}\textbf{gad} } \textit{importance/grandness/greatness} (N)\\
\noindent {\tovian \fontsize{20pt}{10pt} \textbf{gad} }\\
\noindent /g{\textprimstress}ad/\\


\noindent History:

\vspace{-0pt}
\hspace{40pt}
\begin{tabular}{ccc}
\textit{0} & /aga{\textsubbridge{t}}a/&$\rightarrow$ & \textit{2000} & /agada/&$\rightarrow$ & \textit{7500} & /gada/&$\rightarrow$ & \textit{9500} & /gad/& \\
\end{tabular}

\vspace{20pt}\hline

\end{nopagebreak}
\filbreak



\vspace{15pt}
\begin{nopagebreak}
\noindent{\fontsize{20pt}{10pt}\textbf{tid} } \textit{in (obsolete)} (became locative "ti-")\\
\noindent {\tovian \fontsize{20pt}{10pt} \textbf{tid} }\\
\noindent /{\textsubbridge{t}}{\textprimstress}id/\\


\noindent History:

\vspace{-0pt}
\hspace{40pt}
\begin{tabular}{ccc}
\textit{0} & /{\textsubbridge{t}}id/& \\
\end{tabular}

\vspace{20pt}\hline

\end{nopagebreak}
\filbreak



\vspace{15pt}
\begin{nopagebreak}
\noindent{\fontsize{20pt}{10pt}\textbf{lo-} } \textit{inanimate noun class} (CLASS)\\
\noindent {\tovian \fontsize{20pt}{10pt} \textbf{lo-} }\\
\noindent /l{\textprimstress}o/\\


\noindent History:

\vspace{-0pt}
\hspace{40pt}
\begin{tabular}{ccc}
\textit{0} & /lo-/& \\
\end{tabular}

\vspace{20pt}\hline

\end{nopagebreak}
\filbreak



\vspace{15pt}
\begin{nopagebreak}
\noindent{\fontsize{20pt}{10pt}\textbf{si-} } \textit{INS/instrumental case marker} (CASE)\\
\noindent {\tovian \fontsize{20pt}{10pt} \textbf{si-} }\\
\noindent /s{\textprimstress}i/\\


\noindent History:

\vspace{-0pt}
\hspace{40pt}
\begin{tabular}{ccc}
\textit{0} & /si-/& \\
\end{tabular}

\vspace{20pt}\hline

\end{nopagebreak}
\filbreak



\vspace{15pt}
\begin{nopagebreak}
\noindent{\fontsize{20pt}{10pt}\textbf{teshan} } \textit{instrument} (N)\\
\noindent {\tovian \fontsize{20pt}{10pt} \textbf{teshan} }\\
\noindent /{\textsubbridge{t}}{\textprimstress}e{\textesh}an/\\
\noindent lit. sound+tool\\


\noindent History:

\vspace{-0pt}
\hspace{40pt}
\begin{tabular}{ccc}
\textit{1500} & /{\textsubbridge{t}}{\textsubbridge{t}}e{\textyogh}ah{\dh}an/&$\rightarrow$ & \textit{3000} & /{\textsubbridge{t}}e{\textyogh}ah{\dh}an/&$\rightarrow$ & \textit{4502} & /{\textsubbridge{t}}e{\textyogh}ahan/&$\rightarrow$ & \textit{5000} & /{\textsubbridge{t}}e{\textyogh}aan/&$\rightarrow$ & \textit{5500} & /{\textsubbridge{t}}e{\textyogh}an/&$\rightarrow$ & \textit{12001} & /{\textsubbridge{t}}e{\textesh}an/& \\
\end{tabular}

\vspace{20pt}\hline

\end{nopagebreak}
\filbreak



\vspace{15pt}
\begin{nopagebreak}
\noindent{\fontsize{20pt}{10pt}\textbf{lod} } \textit{it/it (inanimate)} (P)\\
\noindent {\tovian \fontsize{20pt}{10pt} \textbf{lod} }\\
\noindent /l{\textprimstress}od/\\


\noindent History:

\vspace{-0pt}
\hspace{40pt}
\begin{tabular}{ccc}
\textit{0} & /lo{\textsubbridge{t}}a/&$\rightarrow$ & \textit{2000} & /loda/&$\rightarrow$ & \textit{9500} & /lod/& \\
\end{tabular}

\vspace{20pt}\hline

\end{nopagebreak}
\filbreak



\vspace{15pt}
\begin{nopagebreak}
\noindent{\fontsize{20pt}{10pt}\textbf{yalroladil} } \textit{jealousy} (N)\\
\noindent {\tovian \fontsize{20pt}{10pt} \textbf{yalroladil} }\\
\noindent /yalrol{\textprimstress}adil/\\
\noindent lit. wickedness+comparison\\


\noindent History:

\vspace{-0pt}
\hspace{40pt}
\begin{tabular}{ccc}
\textit{402} & /oyalroladil/&$\rightarrow$ & \textit{2100} & /yalroladil/& \\
\end{tabular}

\vspace{20pt}\hline

\end{nopagebreak}
\filbreak



\vspace{15pt}
\begin{nopagebreak}
\noindent{\fontsize{20pt}{10pt}\textbf{tlun} } \textit{joining/sharing} (N)\\
\noindent {\tovian \fontsize{20pt}{10pt} \textbf{tlun} }\\
\noindent /{\texttoptiebar{t\textbeltl}}{\textprimstress}un/\\


\noindent History:

\vspace{-0pt}
\hspace{40pt}
\begin{tabular}{ccc}
\textit{0} & /{\texttoptiebar{t\textbeltl}}una/&$\rightarrow$ & \textit{9500} & /{\texttoptiebar{t\textbeltl}}un/& \\
\end{tabular}

\vspace{20pt}\hline

\end{nopagebreak}
\filbreak



\vspace{15pt}
\begin{nopagebreak}
\noindent{\fontsize{20pt}{10pt}\textbf{vek} } \textit{killing} (N)\\
\noindent {\tovian \fontsize{20pt}{10pt} \textbf{vek} }\\
\noindent /v{\textprimstress}ek/\\


\noindent History:

\vspace{-0pt}
\hspace{40pt}
\begin{tabular}{ccc}
\textit{0} & /vekh/&$\rightarrow$ & \textit{5000} & /vek/& \\
\end{tabular}

\vspace{20pt}\hline

\end{nopagebreak}
\filbreak



\vspace{15pt}
\begin{nopagebreak}
\noindent{\fontsize{20pt}{10pt}\textbf{rishel} } \textit{kindness} (N)\\
\noindent {\tovian \fontsize{20pt}{10pt} \textbf{rishel} }\\
\noindent /r{\textprimstress}i{\textesh}el/\\


\noindent History:

\vspace{-0pt}
\hspace{40pt}
\begin{tabular}{ccc}
\textit{0} & /rai{\textesh}el/&$\rightarrow$ & \textit{5500} & /ri{\textesh}el/& \\
\end{tabular}

\vspace{20pt}\hline

\end{nopagebreak}
\filbreak



\vspace{15pt}
\begin{nopagebreak}
\noindent{\fontsize{20pt}{10pt}\textbf{nashithan} } \textit{knife} (N)\\
\noindent {\tovian \fontsize{20pt}{10pt} \textbf{nashithan} }\\
\noindent /na{\textesh}{\textprimstress}i{\texttheta}an/\\
\noindent lit. small+sharpness+tool\\


\noindent History:

\vspace{-0pt}
\hspace{40pt}
\begin{tabular}{ccc}
\textit{50} & /nina{\textyogh}i{\texttheta}{\dh}an/&$\rightarrow$ & \textit{4502} & /nina{\textyogh}i{\texttheta}an/&$\rightarrow$ & \textit{9200} & /n{\textschwa}na{\textyogh}i{\texttheta}an/&$\rightarrow$ & \textit{11990} & /nna{\textyogh}i{\texttheta}an/&$\rightarrow$ & \textit{12001} & /nna{\textesh}i{\texttheta}an/&$\rightarrow$ & \textit{12005} & /na{\textesh}i{\texttheta}an/& \\
\end{tabular}

\vspace{20pt}\hline

\end{nopagebreak}
\filbreak



\vspace{15pt}
\begin{nopagebreak}
\noindent{\fontsize{20pt}{10pt}\textbf{lhar} } \textit{knowledge} (N)\\
\noindent {\tovian \fontsize{20pt}{10pt} \textbf{lhar} }\\
\noindent /{\textbeltl}{\textprimstress}ar/\\


\noindent History:

\vspace{-0pt}
\hspace{40pt}
\begin{tabular}{ccc}
\textit{0} & /{\textbeltl}ara/&$\rightarrow$ & \textit{9500} & /{\textbeltl}ar/& \\
\end{tabular}

\vspace{20pt}\hline

\end{nopagebreak}
\filbreak



\vspace{15pt}
\begin{nopagebreak}
\noindent{\fontsize{20pt}{10pt}\textbf{tel} } \textit{lake} (N)\\
\noindent {\tovian \fontsize{20pt}{10pt} \textbf{tel} }\\
\noindent /{\textsubbridge{t}}{\textprimstress}el/\\


\noindent History:

\vspace{-0pt}
\hspace{40pt}
\begin{tabular}{ccc}
\textit{0} & /o{\textsubbridge{t}}kele/&$\rightarrow$ & \textit{3000} & /o{\textsubbridge{t}}ele/&$\rightarrow$ & \textit{7500} & /{\textsubbridge{t}}ele/&$\rightarrow$ & \textit{9500} & /{\textsubbridge{t}}el/& \\
\end{tabular}

\vspace{20pt}\hline

\end{nopagebreak}
\filbreak



\vspace{15pt}
\begin{nopagebreak}
\noindent{\fontsize{20pt}{10pt}\textbf{falor} } \textit{leader} (N)\\
\noindent {\tovian \fontsize{20pt}{10pt} \textbf{falor} }\\
\noindent /f{\textprimstress}alor/\\


\noindent History:

\vspace{-0pt}
\hspace{40pt}
\begin{tabular}{ccc}
\textit{0} & /falor/& \\
\end{tabular}

\vspace{20pt}\hline

\end{nopagebreak}
\filbreak



\vspace{15pt}
\begin{nopagebreak}
\noindent{\fontsize{20pt}{10pt}\textbf{hed} } \textit{left} (N)\\
\noindent {\tovian \fontsize{20pt}{10pt} \textbf{hed} }\\
\noindent /h{\textprimstress}ed/\\


\noindent History:

\vspace{-0pt}
\hspace{40pt}
\begin{tabular}{ccc}
\textit{0} & /he{\textsubbridge{t}}a/&$\rightarrow$ & \textit{2000} & /heda/&$\rightarrow$ & \textit{9500} & /hed/& \\
\end{tabular}

\vspace{20pt}\hline

\end{nopagebreak}
\filbreak



\vspace{15pt}
\begin{nopagebreak}
\noindent{\fontsize{20pt}{10pt}\textbf{kel} } \textit{leg/limb/branch} (N)\\
\noindent {\tovian \fontsize{20pt}{10pt} \textbf{kel} }\\
\noindent /k{\textprimstress}el/\\


\noindent History:

\vspace{-0pt}
\hspace{40pt}
\begin{tabular}{ccc}
\textit{0} & /kelu/&$\rightarrow$ & \textit{9500} & /kel/& \\
\end{tabular}

\vspace{20pt}\hline

\end{nopagebreak}
\filbreak



\vspace{15pt}
\begin{nopagebreak}
\noindent{\fontsize{20pt}{10pt}\textbf{mel} } \textit{life} (N)\\
\noindent {\tovian \fontsize{20pt}{10pt} \textbf{mel} }\\
\noindent /m{\textprimstress}el/\\


\noindent History:

\vspace{-0pt}
\hspace{40pt}
\begin{tabular}{ccc}
\textit{0} & /{\textschwa}mel/&$\rightarrow$ & \textit{2300} & /mel/& \\
\end{tabular}

\vspace{20pt}\hline

\end{nopagebreak}
\filbreak



\vspace{15pt}
\begin{nopagebreak}
\noindent{\fontsize{20pt}{10pt}\textbf{an} } \textit{light} (N)\\
\noindent {\tovian \fontsize{20pt}{10pt} \textbf{an} }\\
\noindent /{\textprimstress}an/\\


\noindent History:

\vspace{-0pt}
\hspace{40pt}
\begin{tabular}{ccc}
\textit{0} & /ahana/&$\rightarrow$ & \textit{5000} & /aana/&$\rightarrow$ & \textit{5500} & /ana/&$\rightarrow$ & \textit{9500} & /an/& \\
\end{tabular}

\vspace{20pt}\hline

\end{nopagebreak}
\filbreak



\vspace{15pt}
\begin{nopagebreak}
\noindent{\fontsize{20pt}{10pt}\textbf{ti-} } \textit{LOC/locative case marker} (CASE)\\
\noindent {\tovian \fontsize{20pt}{10pt} \textbf{ti-} }\\
\noindent /{\textsubbridge{t}}{\textprimstress}i/\\


\noindent History:

\vspace{-0pt}
\hspace{40pt}
\begin{tabular}{ccc}
\textit{0} & /{\textsubbridge{t}}i-/& \\
\end{tabular}

\vspace{20pt}\hline

\end{nopagebreak}
\filbreak



\vspace{15pt}
\begin{nopagebreak}
\noindent{\fontsize{20pt}{10pt}\textbf{sil} } \textit{make/create} (N)\\
\noindent {\tovian \fontsize{20pt}{10pt} \textbf{sil} }\\
\noindent /s{\textprimstress}il/\\


\noindent History:

\vspace{-0pt}
\hspace{40pt}
\begin{tabular}{ccc}
\textit{0} & /silahe{\textsubbridge{t}}/&$\rightarrow$ & \textit{1000} & /silah{\textsubbridge{t}}/&$\rightarrow$ & \textit{3000} & /silah/&$\rightarrow$ & \textit{5000} & /sila/&$\rightarrow$ & \textit{9500} & /sil/& \\
\end{tabular}

\vspace{20pt}\hline

\end{nopagebreak}
\filbreak



\vspace{15pt}
\begin{nopagebreak}
\noindent{\fontsize{20pt}{10pt}\textbf{na} } \textit{me/I} (P)\\
\noindent {\tovian \fontsize{20pt}{10pt} \textbf{na} }\\
\noindent /n{\textprimstress}a/\\


\noindent History:

\vspace{-0pt}
\hspace{40pt}
\begin{tabular}{ccc}
\textit{0} & /na/& \\
\end{tabular}

\vspace{20pt}\hline

\end{nopagebreak}
\filbreak



\vspace{15pt}
\begin{nopagebreak}
\noindent{\fontsize{20pt}{10pt}\textbf{lethidriks} } \textit{meat} (N)\\
\noindent {\tovian \fontsize{20pt}{10pt} \textbf{lethidriks} }\\
\noindent /le{\texttheta}{\textprimstress}idriks/\\
\noindent lit. tasting+GEN+blood\\


\noindent History:

\vspace{-0pt}
\hspace{40pt}
\begin{tabular}{ccc}
\textit{1000} & /ale{\dh}ahi-driks/&$\rightarrow$ & \textit{1000} & /ale{\dh}ahidriks/&$\rightarrow$ & \textit{5000} & /ale{\dh}aidriks/&$\rightarrow$ & \textit{5500} & /ale{\dh}idriks/&$\rightarrow$ & \textit{7500} & /le{\dh}idriks/&$\rightarrow$ & \textit{12002} & /le{\texttheta}idriks/& \\
\end{tabular}

\vspace{20pt}\hline

\end{nopagebreak}
\filbreak



\vspace{15pt}
\begin{nopagebreak}
\noindent{\fontsize{20pt}{10pt}\textbf{lagorim} } \textit{memory} (N)\\
\noindent {\tovian \fontsize{20pt}{10pt} \textbf{lagorim} }\\
\noindent /lag{\textprimstress}orim/\\


\noindent History:

\vspace{-0pt}
\hspace{40pt}
\begin{tabular}{ccc}
\textit{0} & /elakorim/&$\rightarrow$ & \textit{2000} & /elagorim/&$\rightarrow$ & \textit{7500} & /lagorim/& \\
\end{tabular}

\vspace{20pt}\hline

\end{nopagebreak}
\filbreak



\vspace{15pt}
\begin{nopagebreak}
\noindent{\fontsize{20pt}{10pt}\textbf{mo-} } \textit{MID/middle voice marker} (VOICE)\\
\noindent {\tovian \fontsize{20pt}{10pt} \textbf{mo-} }\\
\noindent /m{\textprimstress}o/\\


\noindent History:

\vspace{-0pt}
\hspace{40pt}
\begin{tabular}{ccc}
\textit{0} & /mo-/& \\
\end{tabular}

\vspace{20pt}\hline

\end{nopagebreak}
\filbreak



\vspace{15pt}
\begin{nopagebreak}
\noindent{\fontsize{20pt}{10pt}\textbf{naninat} } \textit{month} (N)\\
\noindent {\tovian \fontsize{20pt}{10pt} \textbf{naninat} }\\
\noindent /nan{\textprimstress}ina{\textsubbridge{t}}/\\
\noindent lit. small+small+cycle\\


\noindent History:

\vspace{-0pt}
\hspace{40pt}
\begin{tabular}{ccc}
\textit{0} & /ninanina{\textsubbridge{t}}o{\textsubbridge{t}}/&$\rightarrow$ & \textit{1000} & /ninanina{\textsubbridge{t}}{\textsubbridge{t}}/&$\rightarrow$ & \textit{3000} & /ninanina{\textsubbridge{t}}/&$\rightarrow$ & \textit{9200} & /n{\textschwa}nanina{\textsubbridge{t}}/&$\rightarrow$ & \textit{11990} & /nnanina{\textsubbridge{t}}/&$\rightarrow$ & \textit{12005} & /nanina{\textsubbridge{t}}/& \\
\end{tabular}

\vspace{20pt}\hline

\end{nopagebreak}
\filbreak



\vspace{15pt}
\begin{nopagebreak}
\noindent{\fontsize{20pt}{10pt}\textbf{lius} } \textit{moon} (N)\\
\noindent {\tovian \fontsize{20pt}{10pt} \textbf{lius} }\\
\noindent /l{\textprimstress}ius/\\


\noindent History:

\vspace{-0pt}
\hspace{40pt}
\begin{tabular}{ccc}
\textit{0} & /alius/&$\rightarrow$ & \textit{7500} & /lius/& \\
\end{tabular}

\vspace{20pt}\hline

\end{nopagebreak}
\filbreak



\vspace{15pt}
\begin{nopagebreak}
\noindent{\fontsize{20pt}{10pt}\textbf{mar} } \textit{mother} (N)\\
\noindent {\tovian \fontsize{20pt}{10pt} \textbf{mar} }\\
\noindent /m{\textprimstress}ar/\\


\noindent History:

\vspace{-0pt}
\hspace{40pt}
\begin{tabular}{ccc}
\textit{0} & /mari/&$\rightarrow$ & \textit{9500} & /mar/& \\
\end{tabular}

\vspace{20pt}\hline

\end{nopagebreak}
\filbreak



\vspace{15pt}
\begin{nopagebreak}
\noindent{\fontsize{20pt}{10pt}\textbf{olan} } \textit{mouth} (N)\\
\noindent {\tovian \fontsize{20pt}{10pt} \textbf{olan} }\\
\noindent /{\textprimstress}olan/\\


\noindent History:

\vspace{-0pt}
\hspace{40pt}
\begin{tabular}{ccc}
\textit{0} & /eolan/&$\rightarrow$ & \textit{7500} & /olan/& \\
\end{tabular}

\vspace{20pt}\hline

\end{nopagebreak}
\filbreak



\vspace{15pt}
\begin{nopagebreak}
\noindent{\fontsize{20pt}{10pt}\textbf{kwil} } \textit{movement} (N)\\
\noindent {\tovian \fontsize{20pt}{10pt} \textbf{kwil} }\\
\noindent /kw{\textprimstress}il/\\


\noindent History:

\vspace{-0pt}
\hspace{40pt}
\begin{tabular}{ccc}
\textit{0} & /kwil/& \\
\end{tabular}

\vspace{20pt}\hline

\end{nopagebreak}
\filbreak



\vspace{15pt}
\begin{nopagebreak}
\noindent{\fontsize{20pt}{10pt}\textbf{ki} } \textit{nine/9} (N)\\
\noindent {\tovian \fontsize{20pt}{10pt} \textbf{ki} }\\
\noindent /k{\textprimstress}i/\\


\noindent History:

\vspace{-0pt}
\hspace{40pt}
\begin{tabular}{ccc}
\textit{0} & /ki/& \\
\end{tabular}

\vspace{20pt}\hline

\end{nopagebreak}
\filbreak



\vspace{15pt}
\begin{nopagebreak}
\noindent{\fontsize{20pt}{10pt}\textbf{i-} } \textit{NOM/nomative case marker} (CASE)\\
\noindent {\tovian \fontsize{20pt}{10pt} \textbf{i-} }\\
\noindent /{\textprimstress}i/\\


\noindent History:

\vspace{-0pt}
\hspace{40pt}
\begin{tabular}{ccc}
\textit{0} & /i-/& \\
\end{tabular}

\vspace{20pt}\hline

\end{nopagebreak}
\filbreak



\vspace{15pt}
\begin{nopagebreak}
\noindent{\fontsize{20pt}{10pt}\textbf{nadur} } \textit{non-person hair/fur/hide/skin/scales} (N)\\
\noindent {\tovian \fontsize{20pt}{10pt} \textbf{nadur} }\\
\noindent /n{\textprimstress}adur/\\


\noindent History:

\vspace{-0pt}
\hspace{40pt}
\begin{tabular}{ccc}
\textit{0} & /ana{\textsubbridge{t}}ur/&$\rightarrow$ & \textit{2000} & /anadur/&$\rightarrow$ & \textit{7500} & /nadur/& \\
\end{tabular}

\vspace{20pt}\hline

\end{nopagebreak}
\filbreak



\vspace{15pt}
\begin{nopagebreak}
\noindent{\fontsize{20pt}{10pt}\textbf{lar} } \textit{north} (N)\\
\noindent {\tovian \fontsize{20pt}{10pt} \textbf{lar} }\\
\noindent /l{\textprimstress}ar/\\


\noindent History:

\vspace{-0pt}
\hspace{40pt}
\begin{tabular}{ccc}
\textit{0} & /lare/&$\rightarrow$ & \textit{9500} & /lar/& \\
\end{tabular}

\vspace{20pt}\hline

\end{nopagebreak}
\filbreak



\vspace{15pt}
\begin{nopagebreak}
\noindent{\fontsize{20pt}{10pt}\textbf{verun} } \textit{nose} (N)\\
\noindent {\tovian \fontsize{20pt}{10pt} \textbf{verun} }\\
\noindent /v{\textprimstress}erun/\\


\noindent History:

\vspace{-0pt}
\hspace{40pt}
\begin{tabular}{ccc}
\textit{0} & /verun/& \\
\end{tabular}

\vspace{20pt}\hline

\end{nopagebreak}
\filbreak



\vspace{15pt}
\begin{nopagebreak}
\noindent{\fontsize{20pt}{10pt}\textbf{tl0-} } \textit{OBL/obligative mood marker} (MOOD)\\
\noindent {\tovian \fontsize{20pt}{10pt} \textbf{tl0-} }\\
\noindent /{\texttoptiebar{t\textbeltl}}0-/\\


\noindent History:

\vspace{-0pt}
\hspace{40pt}
\begin{tabular}{ccc}
\textit{0} & /{\texttoptiebar{t\textbeltl}}0-/& \\
\end{tabular}

\vspace{20pt}\hline

\end{nopagebreak}
\filbreak



\vspace{15pt}
\begin{nopagebreak}
\noindent{\fontsize{20pt}{10pt}\textbf{lhialhi} } \textit{one hundred forty four/144} (N)\\
\noindent {\tovian \fontsize{20pt}{10pt} \textbf{lhialhi} }\\
\noindent /{\textbeltl}i{\textprimstress}a{\textbeltl}i/\\
\noindent lit. twelve+twelve\\


\noindent History:

\vspace{-0pt}
\hspace{40pt}
\begin{tabular}{ccc}
\textit{4000} & /{\textbeltl}ia{\textbeltl}ia/&$\rightarrow$ & \textit{9500} & /{\textbeltl}ia{\textbeltl}i/& \\
\end{tabular}

\vspace{20pt}\hline

\end{nopagebreak}
\filbreak



\vspace{15pt}
\begin{nopagebreak}
\noindent{\fontsize{20pt}{10pt}\textbf{lo} } \textit{one/1} (N)\\
\noindent {\tovian \fontsize{20pt}{10pt} \textbf{lo} }\\
\noindent /l{\textprimstress}o/\\


\noindent History:

\vspace{-0pt}
\hspace{40pt}
\begin{tabular}{ccc}
\textit{0} & /lo/& \\
\end{tabular}

\vspace{20pt}\hline

\end{nopagebreak}
\filbreak



\vspace{15pt}
\begin{nopagebreak}
\noindent{\fontsize{20pt}{10pt}\textbf{do} } \textit{opposite/not/un-} (N)\\
\noindent {\tovian \fontsize{20pt}{10pt} \textbf{do} }\\
\noindent /d{\textprimstress}o/\\


\noindent History:

\vspace{-0pt}
\hspace{40pt}
\begin{tabular}{ccc}
\textit{0} & /do/& \\
\end{tabular}

\vspace{20pt}\hline

\end{nopagebreak}
\filbreak



\vspace{15pt}
\begin{nopagebreak}
\noindent{\fontsize{20pt}{10pt}\textbf{lhe-} } \textit{OPT/optative mood marker} (MOOD)\\
\noindent {\tovian \fontsize{20pt}{10pt} \textbf{lhe-} }\\
\noindent /{\textbeltl}{\textprimstress}e/\\


\noindent History:

\vspace{-0pt}
\hspace{40pt}
\begin{tabular}{ccc}
\textit{0} & /{\textbeltl}e-/& \\
\end{tabular}

\vspace{20pt}\hline

\end{nopagebreak}
\filbreak



\vspace{15pt}
\begin{nopagebreak}
\noindent{\fontsize{20pt}{10pt}\textbf{har} } \textit{pain} (N)\\
\noindent {\tovian \fontsize{20pt}{10pt} \textbf{har} }\\
\noindent /h{\textprimstress}ar/\\


\noindent History:

\vspace{-0pt}
\hspace{40pt}
\begin{tabular}{ccc}
\textit{0} & /hara/&$\rightarrow$ & \textit{9500} & /har/& \\
\end{tabular}

\vspace{20pt}\hline

\end{nopagebreak}
\filbreak



\vspace{15pt}
\begin{nopagebreak}
\noindent{\fontsize{20pt}{10pt}\textbf{pa-} } \textit{PASS/passive voice marker} (VOICE)\\
\noindent {\tovian \fontsize{20pt}{10pt} \textbf{pa-} }\\
\noindent /p{\textprimstress}a/\\


\noindent History:

\vspace{-0pt}
\hspace{40pt}
\begin{tabular}{ccc}
\textit{0} & /pa-/& \\
\end{tabular}

\vspace{20pt}\hline

\end{nopagebreak}
\filbreak



\vspace{15pt}
\begin{nopagebreak}
\noindent{\fontsize{20pt}{10pt}\textbf{hulnum} } \textit{past} (Ns)\\
\noindent {\tovian \fontsize{20pt}{10pt} \textbf{hulnum} }\\
\noindent /h{\textprimstress}ulnum/\\
\noindent lit. behind+path\\


\noindent History:

\vspace{-0pt}
\hspace{40pt}
\begin{tabular}{ccc}
\textit{10} & /hulnum/& \\
\end{tabular}

\vspace{20pt}\hline

\end{nopagebreak}
\filbreak



\vspace{15pt}
\begin{nopagebreak}
\noindent{\fontsize{20pt}{10pt}\textbf{num} } \textit{path} (N)\\
\noindent {\tovian \fontsize{20pt}{10pt} \textbf{num} }\\
\noindent /n{\textprimstress}um/\\


\noindent History:

\vspace{-0pt}
\hspace{40pt}
\begin{tabular}{ccc}
\textit{0} & /num/& \\
\end{tabular}

\vspace{20pt}\hline

\end{nopagebreak}
\filbreak



\vspace{15pt}
\begin{nopagebreak}
\noindent{\fontsize{20pt}{10pt}\textbf{tlaf} } \textit{peace/peacefulness/calmness/calm} (N)\\
\noindent {\tovian \fontsize{20pt}{10pt} \textbf{tlaf} }\\
\noindent /{\texttoptiebar{t\textbeltl}}{\textprimstress}a/\\


\noindent History:

\vspace{-0pt}
\hspace{40pt}
\begin{tabular}{ccc}
\textit{0} & /{\texttoptiebar{t\textbeltl}}af/&$\rightarrow$ & \textit{7501} & /{\texttoptiebar{t\textbeltl}}a{\textphi}/& \\
\end{tabular}

\vspace{20pt}\hline

\end{nopagebreak}
\filbreak



\vspace{15pt}
\begin{nopagebreak}
\noindent{\fontsize{20pt}{10pt}\textbf{driel} } \textit{person} (N)\\
\noindent {\tovian \fontsize{20pt}{10pt} \textbf{driel} }\\
\noindent /dr{\textprimstress}iel/\\


\noindent History:

\vspace{-0pt}
\hspace{40pt}
\begin{tabular}{ccc}
\textit{0} & /drihela/&$\rightarrow$ & \textit{5000} & /driela/&$\rightarrow$ & \textit{9500} & /driel/& \\
\end{tabular}

\vspace{20pt}\hline

\end{nopagebreak}
\filbreak



\vspace{15pt}
\begin{nopagebreak}
\noindent{\fontsize{20pt}{10pt}\textbf{trel} } \textit{person skin} (N)\\
\noindent {\tovian \fontsize{20pt}{10pt} \textbf{trel} }\\
\noindent /{\textsubbridge{t}}r{\textprimstress}el/\\


\noindent History:

\vspace{-0pt}
\hspace{40pt}
\begin{tabular}{ccc}
\textit{0} & /{\textsubbridge{t}}relahe{\textsubbridge{t}}/&$\rightarrow$ & \textit{1000} & /{\textsubbridge{t}}relah{\textsubbridge{t}}/&$\rightarrow$ & \textit{3000} & /{\textsubbridge{t}}relah/&$\rightarrow$ & \textit{5000} & /{\textsubbridge{t}}rela/&$\rightarrow$ & \textit{9500} & /{\textsubbridge{t}}rel/& \\
\end{tabular}

\vspace{20pt}\hline

\end{nopagebreak}
\filbreak



\vspace{15pt}
\begin{nopagebreak}
\noindent{\fontsize{20pt}{10pt}\textbf{thil} } \textit{place} (N)\\
\noindent {\tovian \fontsize{20pt}{10pt} \textbf{thil} }\\
\noindent /{\texttheta}{\textprimstress}il/\\


\noindent History:

\vspace{-0pt}
\hspace{40pt}
\begin{tabular}{ccc}
\textit{0} & /{\texttheta}ile/&$\rightarrow$ & \textit{9500} & /{\texttheta}il/& \\
\end{tabular}

\vspace{20pt}\hline

\end{nopagebreak}
\filbreak



\vspace{15pt}
\begin{nopagebreak}
\noindent{\fontsize{20pt}{10pt}\textbf{tumthil} } \textit{plain} (N)\\
\noindent {\tovian \fontsize{20pt}{10pt} \textbf{tumthil} }\\
\noindent /{\textsubbridge{t}}{\textprimstress}um{\texttheta}il/\\
\noindent lit. flatness+place\\


\noindent History:

\vspace{-0pt}
\hspace{40pt}
\begin{tabular}{ccc}
\textit{50} & /{\textsubbridge{t}}um{\texttheta}ile/&$\rightarrow$ & \textit{9500} & /{\textsubbridge{t}}um{\texttheta}il/& \\
\end{tabular}

\vspace{20pt}\hline

\end{nopagebreak}
\filbreak



\vspace{15pt}
\begin{nopagebreak}
\noindent{\fontsize{20pt}{10pt}\textbf{melor} } \textit{plant} (N)\\
\noindent {\tovian \fontsize{20pt}{10pt} \textbf{melor} }\\
\noindent /m{\textprimstress}elor/\\
\noindent lit. life+action\\


\noindent History:

\vspace{-0pt}
\hspace{40pt}
\begin{tabular}{ccc}
\textit{50} & /{\textschwa}mellore/&$\rightarrow$ & \textit{2300} & /mellore/&$\rightarrow$ & \textit{6500} & /melore/&$\rightarrow$ & \textit{9500} & /melor/& \\
\end{tabular}

\vspace{20pt}\hline

\end{nopagebreak}
\filbreak



\vspace{15pt}
\begin{nopagebreak}
\noindent{\fontsize{20pt}{10pt}\textbf{-e} } \textit{plural marker} (PLURAL)\\
\noindent {\tovian \fontsize{20pt}{10pt} \textbf{-e} }\\
\noindent /{\textprimstress}e/\\


\noindent History:

\vspace{-0pt}
\hspace{40pt}
\begin{tabular}{ccc}
\textit{0} & /-e/& \\
\end{tabular}

\vspace{20pt}\hline

\end{nopagebreak}
\filbreak



\vspace{15pt}
\begin{nopagebreak}
\noindent{\fontsize{20pt}{10pt}\textbf{elis} } \textit{possession} (N)\\
\noindent {\tovian \fontsize{20pt}{10pt} \textbf{elis} }\\
\noindent /{\textprimstress}elis/\\


\noindent History:

\vspace{-0pt}
\hspace{40pt}
\begin{tabular}{ccc}
\textit{0} & /elis/& \\
\end{tabular}

\vspace{20pt}\hline

\end{nopagebreak}
\filbreak



\vspace{15pt}
\begin{nopagebreak}
\noindent{\fontsize{20pt}{10pt}\textbf{sho} } \textit{possibility} (N)\\
\noindent {\tovian \fontsize{20pt}{10pt} \textbf{sho} }\\
\noindent /{\textesh}{\textprimstress}o/\\


\noindent History:

\vspace{-0pt}
\hspace{40pt}
\begin{tabular}{ccc}
\textit{0} & /{\textesh}ohe/&$\rightarrow$ & \textit{5000} & /{\textesh}oe/&$\rightarrow$ & \textit{9500} & /{\textesh}o/& \\
\end{tabular}

\vspace{20pt}\hline

\end{nopagebreak}
\filbreak



\vspace{15pt}
\begin{nopagebreak}
\noindent{\fontsize{20pt}{10pt}\textbf{sho-} } \textit{POT/potential mood marker} (MOOD)\\
\noindent {\tovian \fontsize{20pt}{10pt} \textbf{sho-} }\\
\noindent /{\textesh}{\textprimstress}o/\\


\noindent History:

\vspace{-0pt}
\hspace{40pt}
\begin{tabular}{ccc}
\textit{0} & /{\textesh}o-/& \\
\end{tabular}

\vspace{20pt}\hline

\end{nopagebreak}
\filbreak



\vspace{15pt}
\begin{nopagebreak}
\noindent{\fontsize{20pt}{10pt}\textbf{lhelnum} } \textit{present} (Ns)\\
\noindent {\tovian \fontsize{20pt}{10pt} \textbf{lhelnum} }\\
\noindent /{\textbeltl}{\textprimstress}elnum/\\
\noindent lit. center+path\\


\noindent History:

\vspace{-0pt}
\hspace{40pt}
\begin{tabular}{ccc}
\textit{10} & /{\textbeltl}elnum/& \\
\end{tabular}

\vspace{20pt}\hline

\end{nopagebreak}
\filbreak



\vspace{15pt}
\begin{nopagebreak}
\noindent{\fontsize{20pt}{10pt}\textbf{lorefed} } \textit{present/demonstrate} (N)\\
\noindent {\tovian \fontsize{20pt}{10pt} \textbf{lorefed} }\\
\noindent /lor{\textprimstress}efed/\\
\noindent lit. action+show\\


\noindent History:

\vspace{-0pt}
\hspace{40pt}
\begin{tabular}{ccc}
\textit{0} & /lorefe{\textsubbridge{t}}e/&$\rightarrow$ & \textit{2000} & /lorefede/&$\rightarrow$ & \textit{9500} & /lorefed/& \\
\end{tabular}

\vspace{20pt}\hline

\end{nopagebreak}
\filbreak



\vspace{15pt}
\begin{nopagebreak}
\noindent{\fontsize{20pt}{10pt}\textbf{sor} } \textit{proximity/closeness} (N)\\
\noindent {\tovian \fontsize{20pt}{10pt} \textbf{sor} }\\
\noindent /s{\textprimstress}or/\\


\noindent History:

\vspace{-0pt}
\hspace{40pt}
\begin{tabular}{ccc}
\textit{0} & /sor/& \\
\end{tabular}

\vspace{20pt}\hline

\end{nopagebreak}
\filbreak



\vspace{15pt}
\begin{nopagebreak}
\noindent{\fontsize{20pt}{10pt}\textbf{lhu-} } \textit{PUR/purpose case marker} (CASE)\\
\noindent {\tovian \fontsize{20pt}{10pt} \textbf{lhu-} }\\
\noindent /{\textbeltl}{\textprimstress}u/\\


\noindent History:

\vspace{-0pt}
\hspace{40pt}
\begin{tabular}{ccc}
\textit{0} & /{\textbeltl}u-/& \\
\end{tabular}

\vspace{20pt}\hline

\end{nopagebreak}
\filbreak



\vspace{15pt}
\begin{nopagebreak}
\noindent{\fontsize{20pt}{10pt}\textbf{kith} } \textit{reaching/arriving/traveling} (N)\\
\noindent {\tovian \fontsize{20pt}{10pt} \textbf{kith} }\\
\noindent /k{\textprimstress}i{\texttheta}/\\


\noindent History:

\vspace{-0pt}
\hspace{40pt}
\begin{tabular}{ccc}
\textit{0} & /ki{\texttheta}sa/&$\rightarrow$ & \textit{4502} & /ki{\texttheta}a/&$\rightarrow$ & \textit{9500} & /ki{\texttheta}/& \\
\end{tabular}

\vspace{20pt}\hline

\end{nopagebreak}
\filbreak



\vspace{15pt}
\begin{nopagebreak}
\noindent{\fontsize{20pt}{10pt}\textbf{ra-} } \textit{REC/reciprocal voice marker} (VOICE)\\
\noindent {\tovian \fontsize{20pt}{10pt} \textbf{ra-} }\\
\noindent /r{\textprimstress}a/\\


\noindent History:

\vspace{-0pt}
\hspace{40pt}
\begin{tabular}{ccc}
\textit{0} & /ra-/& \\
\end{tabular}

\vspace{20pt}\hline

\end{nopagebreak}
\filbreak



\vspace{15pt}
\begin{nopagebreak}
\noindent{\fontsize{20pt}{10pt}\textbf{lhoden} } \textit{redness/red} (N)\\
\noindent {\tovian \fontsize{20pt}{10pt} \textbf{lhoden} }\\
\noindent /{\textbeltl}{\textprimstress}oden/\\


\noindent History:

\vspace{-0pt}
\hspace{40pt}
\begin{tabular}{ccc}
\textit{0} & /{\textbeltl}eho{\textsubbridge{t}}en/&$\rightarrow$ & \textit{1000} & /{\textbeltl}ho{\textsubbridge{t}}en/&$\rightarrow$ & \textit{2000} & /{\textbeltl}hoden/&$\rightarrow$ & \textit{4502} & /{\textbeltl}oden/& \\
\end{tabular}

\vspace{20pt}\hline

\end{nopagebreak}
\filbreak



\vspace{15pt}
\begin{nopagebreak}
\noindent{\fontsize{20pt}{10pt}\textbf{te-} } \textit{REFL/reflexive voice marker} (VOICE)\\
\noindent {\tovian \fontsize{20pt}{10pt} \textbf{te-} }\\
\noindent /{\textsubbridge{t}}{\textprimstress}e/\\


\noindent History:

\vspace{-0pt}
\hspace{40pt}
\begin{tabular}{ccc}
\textit{0} & /{\textsubbridge{t}}e-/& \\
\end{tabular}

\vspace{20pt}\hline

\end{nopagebreak}
\filbreak



\vspace{15pt}
\begin{nopagebreak}
\noindent{\fontsize{20pt}{10pt}\textbf{ryfethr} } \textit{report/confession} (N)\\
\noindent {\tovian \fontsize{20pt}{10pt} \textbf{ryfethr} }\\
\noindent /ryf{\textprimstress}e{\texttheta}r/\\
\noindent lit. truth+speech\\


\noindent History:

\vspace{-0pt}
\hspace{40pt}
\begin{tabular}{ccc}
\textit{7200} & /aryfe{\dh}ra/&$\rightarrow$ & \textit{7500} & /ryfe{\dh}ra/&$\rightarrow$ & \textit{9500} & /ryfe{\dh}r/&$\rightarrow$ & \textit{12002} & /ryfe{\texttheta}r/& \\
\end{tabular}

\vspace{20pt}\hline

\end{nopagebreak}
\filbreak



\vspace{15pt}
\begin{nopagebreak}
\noindent{\fontsize{20pt}{10pt}\textbf{lhy} } \textit{right} (N)\\
\noindent {\tovian \fontsize{20pt}{10pt} \textbf{lhy} }\\
\noindent /{\textbeltl}y/\\


\noindent History:

\vspace{-0pt}
\hspace{40pt}
\begin{tabular}{ccc}
\textit{0} & /{\textbeltl}oy/&$\rightarrow$ & \textit{2100} & /{\textbeltl}y/& \\
\end{tabular}

\vspace{20pt}\hline

\end{nopagebreak}
\filbreak



\vspace{15pt}
\begin{nopagebreak}
\noindent{\fontsize{20pt}{10pt}\textbf{shotlel} } \textit{river} (N)\\
\noindent {\tovian \fontsize{20pt}{10pt} \textbf{shotlel} }\\
\noindent /{\textesh}{\textprimstress}o{\texttoptiebar{t\textbeltl}}el/\\
\noindent lit. flow+water\\


\noindent History:

\vspace{-0pt}
\hspace{40pt}
\begin{tabular}{ccc}
\textit{500} & /{\textesh}o{\texttoptiebar{t\textbeltl}}z{\textsubbridge{t}}el/&$\rightarrow$ & \textit{2000} & /{\textesh}o{\texttoptiebar{t\textbeltl}}zdel/&$\rightarrow$ & \textit{4500} & /{\textesh}o{\texttoptiebar{t\textbeltl}}zel/&$\rightarrow$ & \textit{12000} & /{\textesh}o{\texttoptiebar{t\textbeltl}}sel/&$\rightarrow$ & \textit{14000} & /{\textesh}o{\texttoptiebar{t\textbeltl}}el/& \\
\end{tabular}

\vspace{20pt}\hline

\end{nopagebreak}
\filbreak



\vspace{15pt}
\begin{nopagebreak}
\noindent{\fontsize{20pt}{10pt}\textbf{sethr} } \textit{running/moving quickly/rushing} (N)\\
\noindent {\tovian \fontsize{20pt}{10pt} \textbf{sethr} }\\
\noindent /s{\textprimstress}e{\texttheta}r/\\


\noindent History:

\vspace{-0pt}
\hspace{40pt}
\begin{tabular}{ccc}
\textit{0} & /se{\texttheta}ir/&$\rightarrow$ & \textit{8000} & /se{\texttheta}r/& \\
\end{tabular}

\vspace{20pt}\hline

\end{nopagebreak}
\filbreak



\vspace{15pt}
\begin{nopagebreak}
\noindent{\fontsize{20pt}{10pt}\textbf{lur} } \textit{sad} (N)\\
\noindent {\tovian \fontsize{20pt}{10pt} \textbf{lur} }\\
\noindent /l{\textprimstress}ur/\\


\noindent History:

\vspace{-0pt}
\hspace{40pt}
\begin{tabular}{ccc}
\textit{0} & /lura/&$\rightarrow$ & \textit{9500} & /lur/& \\
\end{tabular}

\vspace{20pt}\hline

\end{nopagebreak}
\filbreak



\vspace{15pt}
\begin{nopagebreak}
\noindent{\fontsize{20pt}{10pt}\textbf{shotl} } \textit{sand} (N)\\
\noindent {\tovian \fontsize{20pt}{10pt} \textbf{shotl} }\\
\noindent /{\textesh}{\textprimstress}o{\texttoptiebar{t\textbeltl}}/\\
\noindent lit. flow+ground\\
\noindent \textit{ Same as "flowing"}\\


\noindent History:

\vspace{-0pt}
\hspace{40pt}
\begin{tabular}{ccc}
\textit{1000} & /{\textesh}o{\texttoptiebar{t\textbeltl}}{\texttheta}ok{\textesh}{\textsubbridge{t}}/&$\rightarrow$ & \textit{1000} & /{\textesh}o{\texttoptiebar{t\textbeltl}}{\texttheta}k{\textesh}{\textsubbridge{t}}/&$\rightarrow$ & \textit{3000} & /{\textesh}o{\texttoptiebar{t\textbeltl}}{\texttheta}k{\textesh}/&$\rightarrow$ & \textit{4500} & /{\textesh}o{\texttoptiebar{t\textbeltl}}{\texttheta}{\textesh}/&$\rightarrow$ & \textit{4502} & /{\textesh}o{\texttoptiebar{t\textbeltl}}{\texttheta}/&$\rightarrow$ & \textit{14000} & /{\textesh}o{\texttoptiebar{t\textbeltl}}/& \\
\end{tabular}

\vspace{20pt}\hline

\end{nopagebreak}
\filbreak



\vspace{15pt}
\begin{nopagebreak}
\noindent{\fontsize{20pt}{10pt}\textbf{nor} } \textit{sea} (N)\\
\noindent {\tovian \fontsize{20pt}{10pt} \textbf{nor} }\\
\noindent /n{\textprimstress}or/\\


\noindent History:

\vspace{-0pt}
\hspace{40pt}
\begin{tabular}{ccc}
\textit{0} & /nor/& \\
\end{tabular}

\vspace{20pt}\hline

\end{nopagebreak}
\filbreak



\vspace{15pt}
\begin{nopagebreak}
\noindent{\fontsize{20pt}{10pt}\textbf{pulo} } \textit{second} (Ns)\\
\noindent {\tovian \fontsize{20pt}{10pt} \textbf{pulo} }\\
\noindent /p{\textprimstress}ulo/\\
\noindent lit. ESS+two\\


\noindent History:

\vspace{-0pt}
\hspace{40pt}
\begin{tabular}{ccc}
\textit{0} & /pu-loe/&$\rightarrow$ & \textit{1000} & /puloe/&$\rightarrow$ & \textit{9500} & /pulo/& \\
\end{tabular}

\vspace{20pt}\hline

\end{nopagebreak}
\filbreak



\vspace{15pt}
\begin{nopagebreak}
\noindent{\fontsize{20pt}{10pt}\textbf{pol} } \textit{seven/7} (N)\\
\noindent {\tovian \fontsize{20pt}{10pt} \textbf{pol} }\\
\noindent /p{\textprimstress}ol/\\


\noindent History:

\vspace{-0pt}
\hspace{40pt}
\begin{tabular}{ccc}
\textit{0} & /pol/& \\
\end{tabular}

\vspace{20pt}\hline

\end{nopagebreak}
\filbreak



\vspace{15pt}
\begin{nopagebreak}
\noindent{\fontsize{20pt}{10pt}\textbf{tlunalis} } \textit{sharing} (N)\\
\noindent {\tovian \fontsize{20pt}{10pt} \textbf{tlunalis} }\\
\noindent /{\texttoptiebar{t\textbeltl}}un{\textprimstress}alis/\\
\noindent lit. joining+possession\\


\noindent History:

\vspace{-0pt}
\hspace{40pt}
\begin{tabular}{ccc}
\textit{500} & /{\texttoptiebar{t\textbeltl}}unaelis/&$\rightarrow$ & \textit{10000} & /{\texttoptiebar{t\textbeltl}}unalis/& \\
\end{tabular}

\vspace{20pt}\hline

\end{nopagebreak}
\filbreak



\vspace{15pt}
\begin{nopagebreak}
\noindent{\fontsize{20pt}{10pt}\textbf{shith} } \textit{sharpness/alertness/wakefulness} (N)\\
\noindent {\tovian \fontsize{20pt}{10pt} \textbf{shith} }\\
\noindent /{\textesh}{\textprimstress}i{\texttheta}/\\


\noindent History:

\vspace{-0pt}
\hspace{40pt}
\begin{tabular}{ccc}
\textit{0} & /{\textyogh}i{\texttheta}/&$\rightarrow$ & \textit{12001} & /{\textesh}i{\texttheta}/& \\
\end{tabular}

\vspace{20pt}\hline

\end{nopagebreak}
\filbreak



\vspace{15pt}
\begin{nopagebreak}
\noindent{\fontsize{20pt}{10pt}\textbf{benan} } \textit{shelter/protection/covering} (N)\\
\noindent {\tovian \fontsize{20pt}{10pt} \textbf{benan} }\\
\noindent /b{\textprimstress}enan/\\


\noindent History:

\vspace{-0pt}
\hspace{40pt}
\begin{tabular}{ccc}
\textit{0} & /benani/&$\rightarrow$ & \textit{9500} & /benan/& \\
\end{tabular}

\vspace{20pt}\hline

\end{nopagebreak}
\filbreak



\vspace{15pt}
\begin{nopagebreak}
\noindent{\fontsize{20pt}{10pt}\textbf{fed} } \textit{show} (N)\\
\noindent {\tovian \fontsize{20pt}{10pt} \textbf{fed} }\\
\noindent /f{\textprimstress}ed/\\


\noindent History:

\vspace{-0pt}
\hspace{40pt}
\begin{tabular}{ccc}
\textit{0} & /fe{\textsubbridge{t}}e/&$\rightarrow$ & \textit{2000} & /fede/&$\rightarrow$ & \textit{9500} & /fed/& \\
\end{tabular}

\vspace{20pt}\hline

\end{nopagebreak}
\filbreak



\vspace{15pt}
\begin{nopagebreak}
\noindent{\fontsize{20pt}{10pt}\textbf{tem} } \textit{six/6} (N)\\
\noindent {\tovian \fontsize{20pt}{10pt} \textbf{tem} }\\
\noindent /{\textsubbridge{t}}{\textprimstress}em/\\


\noindent History:

\vspace{-0pt}
\hspace{40pt}
\begin{tabular}{ccc}
\textit{0} & /{\textsubbridge{t}}ep/&$\rightarrow$ & \textit{11000} & /{\textsubbridge{t}}em/& \\
\end{tabular}

\vspace{20pt}\hline

\end{nopagebreak}
\filbreak



\vspace{15pt}
\begin{nopagebreak}
\noindent{\fontsize{20pt}{10pt}\textbf{lhiatev} } \textit{sixteen/16} (N)\\
\noindent {\tovian \fontsize{20pt}{10pt} \textbf{lhiatev} }\\
\noindent /{\textbeltl}i{\textprimstress}a{\textsubbridge{t}}ev/\\
\noindent lit. twelve+four\\


\noindent History:

\vspace{-0pt}
\hspace{40pt}
\begin{tabular}{ccc}
\textit{4000} & /{\textbeltl}ia{\textsubbridge{t}}ev/& \\
\end{tabular}

\vspace{20pt}\hline

\end{nopagebreak}
\filbreak



\vspace{15pt}
\begin{nopagebreak}
\noindent{\fontsize{20pt}{10pt}\textbf{rithil} } \textit{sky} (N)\\
\noindent {\tovian \fontsize{20pt}{10pt} \textbf{rithil} }\\
\noindent /r{\textprimstress}i{\texttheta}il/\\
\noindent lit. air+place\\


\noindent History:

\vspace{-0pt}
\hspace{40pt}
\begin{tabular}{ccc}
\textit{10} & /rai{\texttheta}{\texttheta}ile/&$\rightarrow$ & \textit{4502} & /rai{\texttheta}ile/&$\rightarrow$ & \textit{5500} & /ri{\texttheta}ile/&$\rightarrow$ & \textit{9500} & /ri{\texttheta}il/& \\
\end{tabular}

\vspace{20pt}\hline

\end{nopagebreak}
\filbreak



\vspace{15pt}
\begin{nopagebreak}
\noindent{\fontsize{20pt}{10pt}\textbf{nin} } \textit{small} (N)\\
\noindent {\tovian \fontsize{20pt}{10pt} \textbf{nin} }\\
\noindent /n{\textprimstress}in/\\


\noindent History:

\vspace{-0pt}
\hspace{40pt}
\begin{tabular}{ccc}
\textit{0} & /nina/&$\rightarrow$ & \textit{9500} & /nin/& \\
\end{tabular}

\vspace{20pt}\hline

\end{nopagebreak}
\filbreak



\vspace{15pt}
\begin{nopagebreak}
\noindent{\fontsize{20pt}{10pt}\textbf{mataolan} } \textit{smile} (N)\\
\noindent {\tovian \fontsize{20pt}{10pt} \textbf{mataolan} }\\
\noindent /ma{\textsubbridge{t}}a{\textprimstress}olan/\\
\noindent lit. happy+mouth\\


\noindent History:

\vspace{-0pt}
\hspace{40pt}
\begin{tabular}{ccc}
\textit{400} & /ma{\textsubbridge{t}}{\textsubbridge{t}}aeolan/&$\rightarrow$ & \textit{3000} & /ma{\textsubbridge{t}}aeolan/&$\rightarrow$ & \textit{10000} & /ma{\textsubbridge{t}}aolan/& \\
\end{tabular}

\vspace{20pt}\hline

\end{nopagebreak}
\filbreak



\vspace{15pt}
\begin{nopagebreak}
\noindent{\fontsize{20pt}{10pt}\textbf{tesh} } \textit{sound} (N)\\
\noindent {\tovian \fontsize{20pt}{10pt} \textbf{tesh} }\\
\noindent /{\textsubbridge{t}}{\textprimstress}e{\textesh}/\\


\noindent History:

\vspace{-0pt}
\hspace{40pt}
\begin{tabular}{ccc}
\textit{0} & /{\textsubbridge{t}}e{\textsubbridge{t}}e{\textyogh}ah/&$\rightarrow$ & \textit{1000} & /{\textsubbridge{t}}{\textsubbridge{t}}e{\textyogh}ah/&$\rightarrow$ & \textit{3000} & /{\textsubbridge{t}}e{\textyogh}ah/&$\rightarrow$ & \textit{5000} & /{\textsubbridge{t}}e{\textyogh}a/&$\rightarrow$ & \textit{9500} & /{\textsubbridge{t}}e{\textyogh}/&$\rightarrow$ & \textit{12001} & /{\textsubbridge{t}}e{\textesh}/& \\
\end{tabular}

\vspace{20pt}\hline

\end{nopagebreak}
\filbreak



\vspace{15pt}
\begin{nopagebreak}
\noindent{\fontsize{20pt}{10pt}\textbf{rar} } \textit{south} (N)\\
\noindent {\tovian \fontsize{20pt}{10pt} \textbf{rar} }\\
\noindent /r{\textprimstress}ar/\\


\noindent History:

\vspace{-0pt}
\hspace{40pt}
\begin{tabular}{ccc}
\textit{0} & /rar/& \\
\end{tabular}

\vspace{20pt}\hline

\end{nopagebreak}
\filbreak



\vspace{15pt}
\begin{nopagebreak}
\noindent{\fontsize{20pt}{10pt}\textbf{fethr} } \textit{speaking} (N)\\
\noindent {\tovian \fontsize{20pt}{10pt} \textbf{fethr} }\\
\noindent /f{\textprimstress}e{\texttheta}r/\\
\noindent lit. show+thought\\


\noindent History:

\vspace{-0pt}
\hspace{40pt}
\begin{tabular}{ccc}
\textit{100} & /fe{\textsubbridge{t}}ei{\texttheta}ir/&$\rightarrow$ & \textit{1000} & /f{\textsubbridge{t}}ei{\texttheta}ir/&$\rightarrow$ & \textit{3000} & /fei{\texttheta}ir/&$\rightarrow$ & \textit{5500} & /fe{\texttheta}ir/&$\rightarrow$ & \textit{8000} & /fe{\texttheta}r/& \\
\end{tabular}

\vspace{20pt}\hline

\end{nopagebreak}
\filbreak



\vspace{15pt}
\begin{nopagebreak}
\noindent{\fontsize{20pt}{10pt}\textbf{fethr} } \textit{speech} (N)\\
\noindent {\tovian \fontsize{20pt}{10pt} \textbf{fethr} }\\
\noindent /f{\textprimstress}e{\texttheta}r/\\


\noindent History:

\vspace{-0pt}
\hspace{40pt}
\begin{tabular}{ccc}
\textit{0} & /fe{\dh}ra/&$\rightarrow$ & \textit{9500} & /fe{\dh}r/&$\rightarrow$ & \textit{12002} & /fe{\texttheta}r/& \\
\end{tabular}

\vspace{20pt}\hline

\end{nopagebreak}
\filbreak



\vspace{15pt}
\begin{nopagebreak}
\noindent{\fontsize{20pt}{10pt}\textbf{tlesh} } \textit{speed} (N)\\
\noindent {\tovian \fontsize{20pt}{10pt} \textbf{tlesh} }\\
\noindent /{\texttoptiebar{t\textbeltl}}{\textprimstress}e{\textesh}/\\


\noindent History:

\vspace{-0pt}
\hspace{40pt}
\begin{tabular}{ccc}
\textit{0} & /{\texttoptiebar{t\textbeltl}}e{\textyogh}/&$\rightarrow$ & \textit{12001} & /{\texttoptiebar{t\textbeltl}}e{\textesh}/& \\
\end{tabular}

\vspace{20pt}\hline

\end{nopagebreak}
\filbreak



\vspace{15pt}
\begin{nopagebreak}
\noindent{\fontsize{20pt}{10pt}\textbf{ing} } \textit{stickiness} (N)\\
\noindent {\tovian \fontsize{20pt}{10pt} \textbf{ing} }\\
\noindent /{\textprimstress}i{\ng}/\\


\noindent History:

\vspace{-0pt}
\hspace{40pt}
\begin{tabular}{ccc}
\textit{0} & /inga/&$\rightarrow$ & \textit{6000} & /i{\ng}ga/&$\rightarrow$ & \textit{8500} & /i{\ng}a/&$\rightarrow$ & \textit{9500} & /i{\ng}/& \\
\end{tabular}

\vspace{20pt}\hline

\end{nopagebreak}
\filbreak



\vspace{15pt}
\begin{nopagebreak}
\noindent{\fontsize{20pt}{10pt}\textbf{silabor} } \textit{stomach} (N)\\
\noindent {\tovian \fontsize{20pt}{10pt} \textbf{silabor} }\\
\noindent /sil{\textprimstress}abor/\\


\noindent History:

\vspace{-0pt}
\hspace{40pt}
\begin{tabular}{ccc}
\textit{0} & /sildabore/&$\rightarrow$ & \textit{4501} & /silabore/&$\rightarrow$ & \textit{9500} & /silabor/& \\
\end{tabular}

\vspace{20pt}\hline

\end{nopagebreak}
\filbreak



\vspace{15pt}
\begin{nopagebreak}
\noindent{\fontsize{20pt}{10pt}\textbf{meron} } \textit{stone/rock} (N)\\
\noindent {\tovian \fontsize{20pt}{10pt} \textbf{meron} }\\
\noindent /m{\textprimstress}eron/\\


\noindent History:

\vspace{-0pt}
\hspace{40pt}
\begin{tabular}{ccc}
\textit{0} & /meron/& \\
\end{tabular}

\vspace{20pt}\hline

\end{nopagebreak}
\filbreak



\vspace{15pt}
\begin{nopagebreak}
\noindent{\fontsize{20pt}{10pt}\textbf{palor} } \textit{stopping/ceasing} (N)\\
\noindent {\tovian \fontsize{20pt}{10pt} \textbf{palor} }\\
\noindent /p{\textprimstress}alor/\\


\noindent History:

\vspace{-0pt}
\hspace{40pt}
\begin{tabular}{ccc}
\textit{0} & /palor/& \\
\end{tabular}

\vspace{20pt}\hline

\end{nopagebreak}
\filbreak



\vspace{15pt}
\begin{nopagebreak}
\noindent{\fontsize{20pt}{10pt}\textbf{wi-} } \textit{SUB/subjunctive mood marker} (MOOD)\\
\noindent {\tovian \fontsize{20pt}{10pt} \textbf{wi-} }\\
\noindent /w{\textprimstress}i/\\


\noindent History:

\vspace{-0pt}
\hspace{40pt}
\begin{tabular}{ccc}
\textit{0} & /wi-/& \\
\end{tabular}

\vspace{20pt}\hline

\end{nopagebreak}
\filbreak



\vspace{15pt}
\begin{nopagebreak}
\noindent{\fontsize{20pt}{10pt}\textbf{efol} } \textit{sun} (N)\\
\noindent {\tovian \fontsize{20pt}{10pt} \textbf{efol} }\\
\noindent /{\textprimstress}efol/\\


\noindent History:

\vspace{-0pt}
\hspace{40pt}
\begin{tabular}{ccc}
\textit{0} & /eifol/&$\rightarrow$ & \textit{5500} & /efol/& \\
\end{tabular}

\vspace{20pt}\hline

\end{nopagebreak}
\filbreak



\vspace{15pt}
\begin{nopagebreak}
\noindent{\fontsize{20pt}{10pt}\textbf{ngam} } \textit{surpassing/exceeding/overwhelming} (N)\\
\noindent {\tovian \fontsize{20pt}{10pt} \textbf{ngam} }\\
\noindent /{\ng}{\textprimstress}am/\\


\noindent History:

\vspace{-0pt}
\hspace{40pt}
\begin{tabular}{ccc}
\textit{0} & /{\ng}o{\ng}abi/&$\rightarrow$ & \textit{9200} & /{\ng}{\textschwa}{\ng}abi/&$\rightarrow$ & \textit{9500} & /{\ng}{\textschwa}{\ng}ab/&$\rightarrow$ & \textit{11000} & /{\ng}{\textschwa}{\ng}am/&$\rightarrow$ & \textit{11990} & /{\ng}{\ng}am/&$\rightarrow$ & \textit{12005} & /{\ng}am/& \\
\end{tabular}

\vspace{20pt}\hline

\end{nopagebreak}
\filbreak



\vspace{15pt}
\begin{nopagebreak}
\noindent{\fontsize{20pt}{10pt}\textbf{vekhan} } \textit{sword} (N)\\
\noindent {\tovian \fontsize{20pt}{10pt} \textbf{vekhan} }\\
\noindent /v{\textprimstress}ekhan/\\
\noindent lit. killing+tool\\


\noindent History:

\vspace{-0pt}
\hspace{40pt}
\begin{tabular}{ccc}
\textit{50} & /vekh{\dh}an/&$\rightarrow$ & \textit{4502} & /vekhan/& \\
\end{tabular}

\vspace{20pt}\hline

\end{nopagebreak}
\filbreak



\vspace{15pt}
\begin{nopagebreak}
\noindent{\fontsize{20pt}{10pt}\textbf{ngatlun} } \textit{tar} (N)\\
\noindent {\tovian \fontsize{20pt}{10pt} \textbf{ngatlun} }\\
\noindent /{\ng}{\textprimstress}a{\texttoptiebar{t\textbeltl}}un/\\
\noindent lit. stickiness+joining\\


\noindent History:

\vspace{-0pt}
\hspace{40pt}
\begin{tabular}{ccc}
\textit{50} & /inga{\texttoptiebar{t\textbeltl}}una/&$\rightarrow$ & \textit{6000} & /i{\ng}ga{\texttoptiebar{t\textbeltl}}una/&$\rightarrow$ & \textit{7500} & /{\ng}ga{\texttoptiebar{t\textbeltl}}una/&$\rightarrow$ & \textit{8500} & /{\ng}a{\texttoptiebar{t\textbeltl}}una/&$\rightarrow$ & \textit{9500} & /{\ng}a{\texttoptiebar{t\textbeltl}}un/& \\
\end{tabular}

\vspace{20pt}\hline

\end{nopagebreak}
\filbreak



\vspace{15pt}
\begin{nopagebreak}
\noindent{\fontsize{20pt}{10pt}\textbf{lharafed} } \textit{teaching} (N)\\
\noindent {\tovian \fontsize{20pt}{10pt} \textbf{lharafed} }\\
\noindent /{\textbeltl}ar{\textprimstress}afed/\\
\noindent lit. knowledge+show\\


\noindent History:

\vspace{-0pt}
\hspace{40pt}
\begin{tabular}{ccc}
\textit{0} & /{\textbeltl}arafe{\textsubbridge{t}}e/&$\rightarrow$ & \textit{2000} & /{\textbeltl}arafede/&$\rightarrow$ & \textit{9500} & /{\textbeltl}arafed/& \\
\end{tabular}

\vspace{20pt}\hline

\end{nopagebreak}
\filbreak



\vspace{15pt}
\begin{nopagebreak}
\noindent{\fontsize{20pt}{10pt}\textbf{tu-} } \textit{TEMP/temporal case marker} (CASE)\\
\noindent {\tovian \fontsize{20pt}{10pt} \textbf{tu-} }\\
\noindent /{\textsubbridge{t}}{\textprimstress}u/\\


\noindent History:

\vspace{-0pt}
\hspace{40pt}
\begin{tabular}{ccc}
\textit{0} & /{\textsubbridge{t}}u-/& \\
\end{tabular}

\vspace{20pt}\hline

\end{nopagebreak}
\filbreak



\vspace{15pt}
\begin{nopagebreak}
\noindent{\fontsize{20pt}{10pt}\textbf{tir} } \textit{ten/10} (N)\\
\noindent {\tovian \fontsize{20pt}{10pt} \textbf{tir} }\\
\noindent /{\textsubbridge{t}}{\textprimstress}ir/\\


\noindent History:

\vspace{-0pt}
\hspace{40pt}
\begin{tabular}{ccc}
\textit{0} & /{\textsubbridge{t}}ir/& \\
\end{tabular}

\vspace{20pt}\hline

\end{nopagebreak}
\filbreak



\vspace{15pt}
\begin{nopagebreak}
\noindent{\fontsize{20pt}{10pt}\textbf{lod} } \textit{them (inanimate)} (Ps)\\
\noindent {\tovian \fontsize{20pt}{10pt} \textbf{lod} }\\
\noindent /l{\textprimstress}od/\\


\noindent History:

\vspace{-0pt}
\hspace{40pt}
\begin{tabular}{ccc}
\textit{0} & /lo{\textsubbridge{t}}e/&$\rightarrow$ & \textit{2000} & /lode/&$\rightarrow$ & \textit{9500} & /lod/& \\
\end{tabular}

\vspace{20pt}\hline

\end{nopagebreak}
\filbreak



\vspace{15pt}
\begin{nopagebreak}
\noindent{\fontsize{20pt}{10pt}\textbf{t} } \textit{they (plural)} (Ps)\\
\noindent {\tovian \fontsize{20pt}{10pt} \textbf{t} }\\
\noindent /{\textsubbridge{t}}/\\


\noindent History:

\vspace{-0pt}
\hspace{40pt}
\begin{tabular}{ccc}
\textit{0} & /{\textsubbridge{t}}e/&$\rightarrow$ & \textit{12004} & /{\textsubbridge{t}}/& \\
\end{tabular}

\vspace{20pt}\hline

\end{nopagebreak}
\filbreak



\vspace{15pt}
\begin{nopagebreak}
\noindent{\fontsize{20pt}{10pt}\textbf{tiel} } \textit{thing} (N)\\
\noindent {\tovian \fontsize{20pt}{10pt} \textbf{tiel} }\\
\noindent /{\textsubbridge{t}}{\textprimstress}iel/\\


\noindent History:

\vspace{-0pt}
\hspace{40pt}
\begin{tabular}{ccc}
\textit{0} & /{\textsubbridge{t}}ihel/&$\rightarrow$ & \textit{5000} & /{\textsubbridge{t}}iel/& \\
\end{tabular}

\vspace{20pt}\hline

\end{nopagebreak}
\filbreak



\vspace{15pt}
\begin{nopagebreak}
\noindent{\fontsize{20pt}{10pt}\textbf{nalh} } \textit{thinking/believing/pondering} (N)\\
\noindent {\tovian \fontsize{20pt}{10pt} \textbf{nalh} }\\
\noindent /n{\textprimstress}a{\textbeltl}/\\


\noindent History:

\vspace{-0pt}
\hspace{40pt}
\begin{tabular}{ccc}
\textit{0} & /na{\textbeltl}/& \\
\end{tabular}

\vspace{20pt}\hline

\end{nopagebreak}
\filbreak



\vspace{15pt}
\begin{nopagebreak}
\noindent{\fontsize{20pt}{10pt}\textbf{lhial} } \textit{thirteen/13} (N)\\
\noindent {\tovian \fontsize{20pt}{10pt} \textbf{lhial} }\\
\noindent /{\textbeltl}{\textprimstress}ial/\\
\noindent lit. twelve+one\\


\noindent History:

\vspace{-0pt}
\hspace{40pt}
\begin{tabular}{ccc}
\textit{4000} & /{\textbeltl}ialo/&$\rightarrow$ & \textit{9500} & /{\textbeltl}ial/& \\
\end{tabular}

\vspace{20pt}\hline

\end{nopagebreak}
\filbreak



\vspace{15pt}
\begin{nopagebreak}
\noindent{\fontsize{20pt}{10pt}\textbf{femlhi} } \textit{thirty six/36} (N)\\
\noindent {\tovian \fontsize{20pt}{10pt} \textbf{femlhi} }\\
\noindent /f{\textprimstress}em{\textbeltl}i/\\
\noindent lit. three+twelve\\


\noindent History:

\vspace{-0pt}
\hspace{40pt}
\begin{tabular}{ccc}
\textit{4000} & /fem{\textbeltl}ia/&$\rightarrow$ & \textit{9500} & /fem{\textbeltl}i/& \\
\end{tabular}

\vspace{20pt}\hline

\end{nopagebreak}
\filbreak



\vspace{15pt}
\begin{nopagebreak}
\noindent{\fontsize{20pt}{10pt}\textbf{loelhiatem} } \textit{thirty/30} (N)\\
\noindent {\tovian \fontsize{20pt}{10pt} \textbf{loelhiatem} }\\
\noindent /loe{\textbeltl}i{\textprimstress}a{\textsubbridge{t}}em/\\
\noindent lit. twenty four+six\\


\noindent History:

\vspace{-0pt}
\hspace{40pt}
\begin{tabular}{ccc}
\textit{4000} & /loe{\textbeltl}ia{\textsubbridge{t}}ep/&$\rightarrow$ & \textit{11000} & /loe{\textbeltl}ia{\textsubbridge{t}}em/& \\
\end{tabular}

\vspace{20pt}\hline

\end{nopagebreak}
\filbreak



\vspace{15pt}
\begin{nopagebreak}
\noindent{\fontsize{20pt}{10pt}\textbf{ithr} } \textit{thought} (N)\\
\noindent {\tovian \fontsize{20pt}{10pt} \textbf{ithr} }\\
\noindent /{\textprimstress}i{\texttheta}r/\\


\noindent History:

\vspace{-0pt}
\hspace{40pt}
\begin{tabular}{ccc}
\textit{0} & /i{\texttheta}ir/&$\rightarrow$ & \textit{8000} & /i{\texttheta}r/& \\
\end{tabular}

\vspace{20pt}\hline

\end{nopagebreak}
\filbreak



\vspace{15pt}
\begin{nopagebreak}
\noindent{\fontsize{20pt}{10pt}\textbf{fem} } \textit{three/3} (N)\\
\noindent {\tovian \fontsize{20pt}{10pt} \textbf{fem} }\\
\noindent /f{\textprimstress}em/\\


\noindent History:

\vspace{-0pt}
\hspace{40pt}
\begin{tabular}{ccc}
\textit{0} & /fem/& \\
\end{tabular}

\vspace{20pt}\hline

\end{nopagebreak}
\filbreak



\vspace{15pt}
\begin{nopagebreak}
\noindent{\fontsize{20pt}{10pt}\textbf{hysha} } \textit{throwing} (N)\\
\noindent {\tovian \fontsize{20pt}{10pt} \textbf{hysha} }\\
\noindent /hy{\textesh}{\textprimstress}a/\\


\noindent History:

\vspace{-0pt}
\hspace{40pt}
\begin{tabular}{ccc}
\textit{0} & /huy{\textyogh}a/&$\rightarrow$ & \textit{2100} & /hy{\textyogh}a/&$\rightarrow$ & \textit{12001} & /hy{\textesh}a/& \\
\end{tabular}

\vspace{20pt}\hline

\end{nopagebreak}
\filbreak



\vspace{15pt}
\begin{nopagebreak}
\noindent{\fontsize{20pt}{10pt}\textbf{tlunalor} } \textit{tie/bind/knot} (N)\\
\noindent {\tovian \fontsize{20pt}{10pt} \textbf{tlunalor} }\\
\noindent /{\texttoptiebar{t\textbeltl}}un{\textprimstress}alor/\\
\noindent lit. joining+action\\


\noindent History:

\vspace{-0pt}
\hspace{40pt}
\begin{tabular}{ccc}
\textit{10} & /{\texttoptiebar{t\textbeltl}}unalore/&$\rightarrow$ & \textit{9500} & /{\texttoptiebar{t\textbeltl}}unalor/& \\
\end{tabular}

\vspace{20pt}\hline

\end{nopagebreak}
\filbreak



\vspace{15pt}
\begin{nopagebreak}
\noindent{\fontsize{20pt}{10pt}\textbf{ngothr} } \textit{time} (N)\\
\noindent {\tovian \fontsize{20pt}{10pt} \textbf{ngothr} }\\
\noindent /{\ng}{\textprimstress}o{\texttheta}r/\\


\noindent History:

\vspace{-0pt}
\hspace{40pt}
\begin{tabular}{ccc}
\textit{0} & /nko{\texttheta}ir/&$\rightarrow$ & \textit{2000} & /ngo{\texttheta}ir/&$\rightarrow$ & \textit{6000} & /{\ng}go{\texttheta}ir/&$\rightarrow$ & \textit{8000} & /{\ng}go{\texttheta}r/&$\rightarrow$ & \textit{8500} & /{\ng}o{\texttheta}r/& \\
\end{tabular}

\vspace{20pt}\hline

\end{nopagebreak}
\filbreak



\vspace{15pt}
\begin{nopagebreak}
\noindent{\fontsize{20pt}{10pt}\textbf{than} } \textit{tool} (N)\\
\noindent {\tovian \fontsize{20pt}{10pt} \textbf{than} }\\
\noindent /{\texttheta}{\textprimstress}an/\\


\noindent History:

\vspace{-0pt}
\hspace{40pt}
\begin{tabular}{ccc}
\textit{0} & /{\dh}an/&$\rightarrow$ & \textit{12002} & /{\texttheta}an/& \\
\end{tabular}

\vspace{20pt}\hline

\end{nopagebreak}
\filbreak



\vspace{15pt}
\begin{nopagebreak}
\noindent{\fontsize{20pt}{10pt}\textbf{thul} } \textit{top} (N)\\
\noindent {\tovian \fontsize{20pt}{10pt} \textbf{thul} }\\
\noindent /{\texttheta}{\textprimstress}ul/\\


\noindent History:

\vspace{-0pt}
\hspace{40pt}
\begin{tabular}{ccc}
\textit{0} & /{\dh}ul/&$\rightarrow$ & \textit{12002} & /{\texttheta}ul/& \\
\end{tabular}

\vspace{20pt}\hline

\end{nopagebreak}
\filbreak



\vspace{15pt}
\begin{nopagebreak}
\noindent{\fontsize{20pt}{10pt}\textbf{pelor} } \textit{town/village} (N)\\
\noindent {\tovian \fontsize{20pt}{10pt} \textbf{pelor} }\\
\noindent /p{\textprimstress}elor/\\


\noindent History:

\vspace{-0pt}
\hspace{40pt}
\begin{tabular}{ccc}
\textit{0} & /pelor/& \\
\end{tabular}

\vspace{20pt}\hline

\end{nopagebreak}
\filbreak



\vspace{15pt}
\begin{nopagebreak}
\noindent{\fontsize{20pt}{10pt}\textbf{lhethel} } \textit{trance/sleep} (N)\\
\noindent {\tovian \fontsize{20pt}{10pt} \textbf{lhethel} }\\
\noindent /{\textbeltl}{\textprimstress}e{\texttheta}el/\\


\noindent History:

\vspace{-0pt}
\hspace{40pt}
\begin{tabular}{ccc}
\textit{0} & /{\textbeltl}e{\dh}el/&$\rightarrow$ & \textit{12002} & /{\textbeltl}e{\texttheta}el/& \\
\end{tabular}

\vspace{20pt}\hline

\end{nopagebreak}
\filbreak



\vspace{15pt}
\begin{nopagebreak}
\noindent{\fontsize{20pt}{10pt}\textbf{glar} } \textit{tree} (N)\\
\noindent {\tovian \fontsize{20pt}{10pt} \textbf{glar} }\\
\noindent /gl{\textprimstress}ar/\\


\noindent History:

\vspace{-0pt}
\hspace{40pt}
\begin{tabular}{ccc}
\textit{0} & /aklare/&$\rightarrow$ & \textit{2000} & /aglare/&$\rightarrow$ & \textit{7500} & /glare/&$\rightarrow$ & \textit{9500} & /glar/& \\
\end{tabular}

\vspace{20pt}\hline

\end{nopagebreak}
\filbreak



\vspace{15pt}
\begin{nopagebreak}
\noindent{\fontsize{20pt}{10pt}\textbf{lhi} } \textit{twelve/12} (N)\\
\noindent {\tovian \fontsize{20pt}{10pt} \textbf{lhi} }\\
\noindent /{\textbeltl}{\textprimstress}i/\\


\noindent History:

\vspace{-0pt}
\hspace{40pt}
\begin{tabular}{ccc}
\textit{0} & /{\textbeltl}iya/&$\rightarrow$ & \textit{1000} & /{\textbeltl}ia/&$\rightarrow$ & \textit{9500} & /{\textbeltl}i/& \\
\end{tabular}

\vspace{20pt}\hline

\end{nopagebreak}
\filbreak



\vspace{15pt}
\begin{nopagebreak}
\noindent{\fontsize{20pt}{10pt}\textbf{loelhial} } \textit{twenty five/25} (N)\\
\noindent {\tovian \fontsize{20pt}{10pt} \textbf{loelhial} }\\
\noindent /loe{\textbeltl}{\textprimstress}ial/\\
\noindent lit. twenty four+one\\


\noindent History:

\vspace{-0pt}
\hspace{40pt}
\begin{tabular}{ccc}
\textit{4000} & /loe{\textbeltl}ialo/&$\rightarrow$ & \textit{9500} & /loe{\textbeltl}ial/& \\
\end{tabular}

\vspace{20pt}\hline

\end{nopagebreak}
\filbreak



\vspace{15pt}
\begin{nopagebreak}
\noindent{\fontsize{20pt}{10pt}\textbf{loelhi} } \textit{twenty four/24} (N)\\
\noindent {\tovian \fontsize{20pt}{10pt} \textbf{loelhi} }\\
\noindent /lo{\textprimstress}e{\textbeltl}i/\\
\noindent lit. two+twelve\\


\noindent History:

\vspace{-0pt}
\hspace{40pt}
\begin{tabular}{ccc}
\textit{4000} & /loe{\textbeltl}ia/&$\rightarrow$ & \textit{9500} & /loe{\textbeltl}i/& \\
\end{tabular}

\vspace{20pt}\hline

\end{nopagebreak}
\filbreak



\vspace{15pt}
\begin{nopagebreak}
\noindent{\fontsize{20pt}{10pt}\textbf{loelhialo} } \textit{twenty six/26} (N)\\
\noindent {\tovian \fontsize{20pt}{10pt} \textbf{loelhialo} }\\
\noindent /loe{\textbeltl}i{\textprimstress}alo/\\
\noindent lit. twenty four+two\\


\noindent History:

\vspace{-0pt}
\hspace{40pt}
\begin{tabular}{ccc}
\textit{4000} & /loe{\textbeltl}ialoe/&$\rightarrow$ & \textit{9500} & /loe{\textbeltl}ialo/& \\
\end{tabular}

\vspace{20pt}\hline

\end{nopagebreak}
\filbreak



\vspace{15pt}
\begin{nopagebreak}
\noindent{\fontsize{20pt}{10pt}\textbf{lhiath} } \textit{twenty/20} (N)\\
\noindent {\tovian \fontsize{20pt}{10pt} \textbf{lhiath} }\\
\noindent /{\textbeltl}{\textprimstress}ia{\texttheta}/\\
\noindent lit. twelve+eight\\


\noindent History:

\vspace{-0pt}
\hspace{40pt}
\begin{tabular}{ccc}
\textit{4000} & /{\textbeltl}iaa{\texttheta}/&$\rightarrow$ & \textit{5500} & /{\textbeltl}ia{\texttheta}/& \\
\end{tabular}

\vspace{20pt}\hline

\end{nopagebreak}
\filbreak



\vspace{15pt}
\begin{nopagebreak}
\noindent{\fontsize{20pt}{10pt}\textbf{lo} } \textit{two/2} (Ns)\\
\noindent {\tovian \fontsize{20pt}{10pt} \textbf{lo} }\\
\noindent /l{\textprimstress}o/\\


\noindent History:

\vspace{-0pt}
\hspace{40pt}
\begin{tabular}{ccc}
\textit{0} & /lo/& \\
\end{tabular}

\vspace{20pt}\hline

\end{nopagebreak}
\filbreak



\vspace{15pt}
\begin{nopagebreak}
\noindent{\fontsize{20pt}{10pt}\textbf{ther} } \textit{up} (N)\\
\noindent {\tovian \fontsize{20pt}{10pt} \textbf{ther} }\\
\noindent /{\texttheta}{\textprimstress}er/\\


\noindent History:

\vspace{-0pt}
\hspace{40pt}
\begin{tabular}{ccc}
\textit{0} & /{\texttheta}era/&$\rightarrow$ & \textit{9500} & /{\texttheta}er/& \\
\end{tabular}

\vspace{20pt}\hline

\end{nopagebreak}
\filbreak



\vspace{15pt}
\begin{nopagebreak}
\noindent{\fontsize{20pt}{10pt}\textbf{n} } \textit{us/we} (Ps)\\
\noindent {\tovian \fontsize{20pt}{10pt} \textbf{n} }\\
\noindent /n/\\


\noindent History:

\vspace{-0pt}
\hspace{40pt}
\begin{tabular}{ccc}
\textit{0} & /ne/&$\rightarrow$ & \textit{12004} & /n/& \\
\end{tabular}

\vspace{20pt}\hline

\end{nopagebreak}
\filbreak



\vspace{15pt}
\begin{nopagebreak}
\noindent{\fontsize{20pt}{10pt}\textbf{velar} } \textit{value} (N)\\
\noindent {\tovian \fontsize{20pt}{10pt} \textbf{velar} }\\
\noindent /v{\textprimstress}elar/\\


\noindent History:

\vspace{-0pt}
\hspace{40pt}
\begin{tabular}{ccc}
\textit{0} & /velare/&$\rightarrow$ & \textit{9500} & /velar/& \\
\end{tabular}

\vspace{20pt}\hline

\end{nopagebreak}
\filbreak



\vspace{15pt}
\begin{nopagebreak}
\noindent{\fontsize{20pt}{10pt}\textbf{nolhen} } \textit{waking dreamer} (N)\\
\noindent {\tovian \fontsize{20pt}{10pt} \textbf{nolhen} }\\
\noindent /n{\textprimstress}o{\textbeltl}en/\\
\noindent lit. walking+dream\\


\noindent History:

\vspace{-0pt}
\hspace{40pt}
\begin{tabular}{ccc}
\textit{10} & /no{\textbeltl}{\textbeltl}ena/&$\rightarrow$ & \textit{4502} & /no{\textbeltl}ena/&$\rightarrow$ & \textit{9500} & /no{\textbeltl}en/& \\
\end{tabular}

\vspace{20pt}\hline

\end{nopagebreak}
\filbreak



\vspace{15pt}
\begin{nopagebreak}
\noindent{\fontsize{20pt}{10pt}\textbf{nolh} } \textit{walking} (N)\\
\noindent {\tovian \fontsize{20pt}{10pt} \textbf{nolh} }\\
\noindent /n{\textprimstress}o{\textbeltl}/\\


\noindent History:

\vspace{-0pt}
\hspace{40pt}
\begin{tabular}{ccc}
\textit{0} & /no{\textbeltl}/& \\
\end{tabular}

\vspace{20pt}\hline

\end{nopagebreak}
\filbreak



\vspace{15pt}
\begin{nopagebreak}
\noindent{\fontsize{20pt}{10pt}\textbf{selor} } \textit{washing} (N)\\
\noindent {\tovian \fontsize{20pt}{10pt} \textbf{selor} }\\
\noindent /s{\textprimstress}elor/\\
\noindent lit. water+action\\


\noindent History:

\vspace{-0pt}
\hspace{40pt}
\begin{tabular}{ccc}
\textit{0} & /z{\textsubbridge{t}}ellore/&$\rightarrow$ & \textit{2000} & /zdellore/&$\rightarrow$ & \textit{4500} & /zellore/&$\rightarrow$ & \textit{6500} & /zelore/&$\rightarrow$ & \textit{9500} & /zelor/&$\rightarrow$ & \textit{12000} & /selor/& \\
\end{tabular}

\vspace{20pt}\hline

\end{nopagebreak}
\filbreak



\vspace{15pt}
\begin{nopagebreak}
\noindent{\fontsize{20pt}{10pt}\textbf{sel} } \textit{water} (N)\\
\noindent {\tovian \fontsize{20pt}{10pt} \textbf{sel} }\\
\noindent /s{\textprimstress}el/\\


\noindent History:

\vspace{-0pt}
\hspace{40pt}
\begin{tabular}{ccc}
\textit{0} & /z{\textsubbridge{t}}el/&$\rightarrow$ & \textit{2000} & /zdel/&$\rightarrow$ & \textit{4500} & /zel/&$\rightarrow$ & \textit{12000} & /sel/& \\
\end{tabular}

\vspace{20pt}\hline

\end{nopagebreak}
\filbreak



\vspace{15pt}
\begin{nopagebreak}
\noindent{\fontsize{20pt}{10pt}\textbf{tirlhonum} } \textit{week} (N)\\
\noindent {\tovian \fontsize{20pt}{10pt} \textbf{tirlhonum} }\\
\noindent /{\textsubbridge{t}}ir{\textbeltl}{\textprimstress}onum/\\
\noindent lit. ten+day\\


\noindent History:

\vspace{-0pt}
\hspace{40pt}
\begin{tabular}{ccc}
\textit{4000} & /{\textsubbridge{t}}ir{\textbeltl}onum/& \\
\end{tabular}

\vspace{20pt}\hline

\end{nopagebreak}
\filbreak



\vspace{15pt}
\begin{nopagebreak}
\noindent{\fontsize{20pt}{10pt}\textbf{tlem} } \textit{weight} (N)\\
\noindent {\tovian \fontsize{20pt}{10pt} \textbf{tlem} }\\
\noindent /{\texttoptiebar{t\textbeltl}}{\textprimstress}em/\\


\noindent History:

\vspace{-0pt}
\hspace{40pt}
\begin{tabular}{ccc}
\textit{0} & /a{\texttoptiebar{t\textbeltl}}emo/&$\rightarrow$ & \textit{7500} & /{\texttoptiebar{t\textbeltl}}emo/&$\rightarrow$ & \textit{9500} & /{\texttoptiebar{t\textbeltl}}em/& \\
\end{tabular}

\vspace{20pt}\hline

\end{nopagebreak}
\filbreak



\vspace{15pt}
\begin{nopagebreak}
\noindent{\fontsize{20pt}{10pt}\textbf{fil} } \textit{west} (N)\\
\noindent {\tovian \fontsize{20pt}{10pt} \textbf{fil} }\\
\noindent /f{\textprimstress}il/\\


\noindent History:

\vspace{-0pt}
\hspace{40pt}
\begin{tabular}{ccc}
\textit{0} & /fil/& \\
\end{tabular}

\vspace{20pt}\hline

\end{nopagebreak}
\filbreak



\vspace{15pt}
\begin{nopagebreak}
\noindent{\fontsize{20pt}{10pt}\textbf{lhan} } \textit{what} (INT)\\
\noindent {\tovian \fontsize{20pt}{10pt} \textbf{lhan} }\\
\noindent /{\textbeltl}{\textprimstress}an/\\


\noindent History:

\vspace{-0pt}
\hspace{40pt}
\begin{tabular}{ccc}
\textit{0} & /{\textbeltl}an/& \\
\end{tabular}

\vspace{20pt}\hline

\end{nopagebreak}
\filbreak



\vspace{15pt}
\begin{nopagebreak}
\noindent{\fontsize{20pt}{10pt}\textbf{lhan} } \textit{when} (INT)\\
\noindent {\tovian \fontsize{20pt}{10pt} \textbf{lhan} }\\
\noindent /{\textbeltl}{\textprimstress}an/\\


\noindent History:

\vspace{-0pt}
\hspace{40pt}
\begin{tabular}{ccc}
\textit{0} & /{\textbeltl}an/& \\
\end{tabular}

\vspace{20pt}\hline

\end{nopagebreak}
\filbreak



\vspace{15pt}
\begin{nopagebreak}
\noindent{\fontsize{20pt}{10pt}\textbf{lhan} } \textit{where} (INT)\\
\noindent {\tovian \fontsize{20pt}{10pt} \textbf{lhan} }\\
\noindent /{\textbeltl}{\textprimstress}an/\\


\noindent History:

\vspace{-0pt}
\hspace{40pt}
\begin{tabular}{ccc}
\textit{0} & /{\textbeltl}an/& \\
\end{tabular}

\vspace{20pt}\hline

\end{nopagebreak}
\filbreak



\vspace{15pt}
\begin{nopagebreak}
\noindent{\fontsize{20pt}{10pt}\textbf{lhan} } \textit{who} (INT)\\
\noindent {\tovian \fontsize{20pt}{10pt} \textbf{lhan} }\\
\noindent /{\textbeltl}{\textprimstress}an/\\


\noindent History:

\vspace{-0pt}
\hspace{40pt}
\begin{tabular}{ccc}
\textit{0} & /{\textbeltl}an/& \\
\end{tabular}

\vspace{20pt}\hline

\end{nopagebreak}
\filbreak



\vspace{15pt}
\begin{nopagebreak}
\noindent{\fontsize{20pt}{10pt}\textbf{lhan} } \textit{why} (INT)\\
\noindent {\tovian \fontsize{20pt}{10pt} \textbf{lhan} }\\
\noindent /{\textbeltl}{\textprimstress}an/\\


\noindent History:

\vspace{-0pt}
\hspace{40pt}
\begin{tabular}{ccc}
\textit{0} & /{\textbeltl}an/& \\
\end{tabular}

\vspace{20pt}\hline

\end{nopagebreak}
\filbreak



\vspace{15pt}
\begin{nopagebreak}
\noindent{\fontsize{20pt}{10pt}\textbf{shilad} } \textit{width} (N)\\
\noindent {\tovian \fontsize{20pt}{10pt} \textbf{shilad} }\\
\noindent /{\textesh}{\textprimstress}ilad/\\


\noindent History:

\vspace{-0pt}
\hspace{40pt}
\begin{tabular}{ccc}
\textit{0} & /{\textyogh}ila{\textsubbridge{t}}e/&$\rightarrow$ & \textit{2000} & /{\textyogh}ilade/&$\rightarrow$ & \textit{9500} & /{\textyogh}ilad/&$\rightarrow$ & \textit{12001} & /{\textesh}ilad/& \\
\end{tabular}

\vspace{20pt}\hline

\end{nopagebreak}
\filbreak



\vspace{15pt}
\begin{nopagebreak}
\noindent{\fontsize{20pt}{10pt}\textbf{rithwil} } \textit{wind/breath} (N)\\
\noindent {\tovian \fontsize{20pt}{10pt} \textbf{rithwil} }\\
\noindent /r{\textprimstress}i{\texttheta}wil/\\
\noindent lit. air+movement\\


\noindent History:

\vspace{-0pt}
\hspace{40pt}
\begin{tabular}{ccc}
\textit{10} & /rai{\texttheta}kwil/&$\rightarrow$ & \textit{3000} & /rai{\texttheta}wil/&$\rightarrow$ & \textit{5500} & /ri{\texttheta}wil/& \\
\end{tabular}

\vspace{20pt}\hline

\end{nopagebreak}
\filbreak



\vspace{15pt}
\begin{nopagebreak}
\noindent{\fontsize{20pt}{10pt}\textbf{ary} } \textit{wisdom/truth} (N)\\
\noindent {\tovian \fontsize{20pt}{10pt} \textbf{ary} }\\
\noindent /{\textprimstress}ary/\\


\noindent History:

\vspace{-0pt}
\hspace{40pt}
\begin{tabular}{ccc}
\textit{0} & /aray/&$\rightarrow$ & \textit{2100} & /ary/& \\
\end{tabular}

\vspace{20pt}\hline

\end{nopagebreak}
\filbreak



\vspace{15pt}
\begin{nopagebreak}
\noindent{\fontsize{20pt}{10pt}\textbf{thraf} } \textit{wondering/curiosity} (N)\\
\noindent {\tovian \fontsize{20pt}{10pt} \textbf{thraf} }\\
\noindent /{\texttheta}r{\textprimstress}af/\\


\noindent History:

\vspace{-0pt}
\hspace{40pt}
\begin{tabular}{ccc}
\textit{0} & /{\texttheta}irafa/&$\rightarrow$ & \textit{8000} & /{\texttheta}rafa/&$\rightarrow$ & \textit{9500} & /{\texttheta}raf/& \\
\end{tabular}

\vspace{20pt}\hline

\end{nopagebreak}
\filbreak



\vspace{15pt}
\begin{nopagebreak}
\noindent{\fontsize{20pt}{10pt}\textbf{glar} } \textit{woods} (N)\\
\noindent {\tovian \fontsize{20pt}{10pt} \textbf{glar} }\\
\noindent /gl{\textprimstress}ar/\\


\noindent History:

\vspace{-0pt}
\hspace{40pt}
\begin{tabular}{ccc}
\textit{0} & /glar/& \\
\end{tabular}

\vspace{20pt}\hline

\end{nopagebreak}
\filbreak



\vspace{15pt}
\begin{nopagebreak}
\noindent{\fontsize{20pt}{10pt}\textbf{felor} } \textit{writing/painting} (N)\\
\noindent {\tovian \fontsize{20pt}{10pt} \textbf{felor} }\\
\noindent /f{\textprimstress}elor/\\
\noindent lit. show+action\\


\noindent History:

\vspace{-0pt}
\hspace{40pt}
\begin{tabular}{ccc}
\textit{10} & /fe{\textsubbridge{t}}elore/&$\rightarrow$ & \textit{1000} & /f{\textsubbridge{t}}elore/&$\rightarrow$ & \textit{3000} & /felore/&$\rightarrow$ & \textit{9500} & /felor/& \\
\end{tabular}

\vspace{20pt}\hline

\end{nopagebreak}
\filbreak



\vspace{15pt}
\begin{nopagebreak}
\noindent{\fontsize{20pt}{10pt}\textbf{ninat} } \textit{year} (N)\\
\noindent {\tovian \fontsize{20pt}{10pt} \textbf{ninat} }\\
\noindent /n{\textprimstress}ina{\textsubbridge{t}}/\\
\noindent lit. small+cycle\\


\noindent History:

\vspace{-0pt}
\hspace{40pt}
\begin{tabular}{ccc}
\textit{10} & /nina{\textsubbridge{t}}o{\textsubbridge{t}}/&$\rightarrow$ & \textit{1000} & /nina{\textsubbridge{t}}{\textsubbridge{t}}/&$\rightarrow$ & \textit{3000} & /nina{\textsubbridge{t}}/& \\
\end{tabular}

\vspace{20pt}\hline

\end{nopagebreak}
\filbreak



\vspace{15pt}
\begin{nopagebreak}
\noindent{\fontsize{20pt}{10pt}\textbf{wa} } \textit{you} (P)\\
\noindent {\tovian \fontsize{20pt}{10pt} \textbf{wa} }\\
\noindent /w{\textprimstress}a/\\


\noindent History:

\vspace{-0pt}
\hspace{40pt}
\begin{tabular}{ccc}
\textit{0} & /wa/& \\
\end{tabular}

\vspace{20pt}\hline

\end{nopagebreak}
\filbreak



\vspace{15pt}
\begin{nopagebreak}
\noindent{\fontsize{20pt}{10pt}\textbf{w} } \textit{you (plural)} (Ps)\\
\noindent {\tovian \fontsize{20pt}{10pt} \textbf{w} }\\
\noindent /w/\\


\noindent History:

\vspace{-0pt}
\hspace{40pt}
\begin{tabular}{ccc}
\textit{0} & /we/&$\rightarrow$ & \textit{12004} & /w/& \\
\end{tabular}

\vspace{20pt}\hline

\end{nopagebreak}
\filbreak



\vspace{15pt}
\begin{nopagebreak}
\noindent{\fontsize{20pt}{10pt}\textbf{donan} } \textit{young} (N)\\
\noindent {\tovian \fontsize{20pt}{10pt} \textbf{donan} }\\
\noindent /d{\textprimstress}onan/\\
\noindent lit. not+old\\


\noindent History:

\vspace{-0pt}
\hspace{40pt}
\begin{tabular}{ccc}
\textit{0} & /donan/& \\
\end{tabular}

\vspace{20pt}\hline

\end{nopagebreak}
\filbreak



\onecolumn


\end{document}

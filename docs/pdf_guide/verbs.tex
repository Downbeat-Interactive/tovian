\section*{Verbs}

Verbs are constructed around a consonant root that denotes its aspect. This root is combined with a suffix to indicate tense. The tense suffixes are \textit{-e} for past,\textit{ -a} for present, and \textit{-o} for future. Optional prefixes can be added to denote mood and voice. This final conjugated auxiliary verb is placed in front of the infinitive. The verb that indicates the action is never conjugated, only the auxiliary verb changes form to show aspect, tense, mood, and voice.

\begin{table}[H]
    \centering
    \begin{tabular}{|c|c|}
        \hline
        \textbf{Tense} & \textbf{Auxiliary Verb Suffix} \\
        \hline
        Past & -e \\
        Present & -a \\
        Future & -o \\
        \hline
    \end{tabular}
    % \caption{Person Suffixes in Tovian}
    \label{tab:verb-tense}
\end{table}


For example, to say "the person is speaking", start with the verb ``to speak": \textit{fethr}. Use the continuous aspect consonant\textit{sh} (from the verb \textit{shotl}, ``to flow") to form the auxiliary verb \textit{sha}. The auxilary verb goes before the infinitive, resulting in \textit{sha fethr}, ``is speaking". \textit{Driel} is ``the person", resulting in the final sentence: \textit{driel sha fethr}, ``the person is speaking". A very literal translation could be ``The person flows speaking".

\begin{table}[h!]
    \centering
    \subsubsection{Aspects with present tense examples}
    \begin{tabular}{c|c|c|c}
    \hline
         \textbf{Aspect}& \makecell{\textbf{Root}\\\textbf{Verb}}& \makecell{\textbf{Aux.}\\\textbf{root}} & \makecell{\textbf{Pres. tense}\\\textbf{Example}} \\
\hline
         Simple&\makecell{\textit{fethr} \\ speak}  & f- & \makecell{\textit{fa mrila}\\``sees"} \\ \hline
         \red{Imperfective}&\makecell{\textit{thraf} \\ wonder}  & th- & \makecell{\textit{tha mrila}\\ ``is seeing"} \\ \hline
         Perfect&\makecell{\textit{lhara} \\ know}  & lh- & \makecell{\textit{lha mrila} \\ ``has seen"} \\ \hline
         Near&\makecell{\textit{tlun} \\ share}& tl- & \makecell{\textit{tla mrila} \\ ``is about to see" }\\ \hline
         Immediate&\makecell{\textit{kesh} \\ burn}& k- & \makecell{\textit{ka mrila} \\ ``is seeing right now" }\\ \hline
         Habitual&\makecell{\textit{mel} \\ live}  & m- & \makecell{\textit{ma mrila} \\``sees regularly" }\\ \hline
         Progressive&\makecell{\textit{sethr} \\ run}& s- & \makecell{\textit{sa mrila} \\``is seeing now" }\\ \hline
         Continuous&\makecell{\textit{shotl} \\ flow}& sh- & \makecell{\textit{sha mrila} \\ ``is seeing" }\\ \hline
         Iterative&\makecell{\textit{nalh} \\ think}& n- & \makecell{\textit{na mrila} \\ ``sees repeatedly" }\\ \hline
         Inceptive&\makecell{\textit{yalor} \\ begin}& y- & \makecell{\textit{ya mrila} \\ ``begins  seeing" }\\ \hline
         Cessative&\makecell{\textit{panor} \\ end}& p- & \makecell{\textit{pa mrila} \\ ``stops seeing" }\\ \hline
         Remote&\makecell{\textit{huysh} \\ throw}  & h- & \makecell{\textit{ha mrila} \\``sees (distant)" \footnotemark[1]} \\
     \end{tabular}
     \\
     \medskip
     \raggedright
     {\footnotesize \footnotemark[1] \raggedleft In past and future tenses, the remote aspect refers to actions that took place long ago or far in the future, but the meaning in present tense is more nuanced. Elves tend to use the rare present remote aspect as an insult, for example \textit{wele ha fethr}: ``they speak but it is irrelevant"}
    \label{tab:aspects}
\end{table}

\medskip


\subsection*{Mood and Voice}

% \subsection*{All verb forms}
\begin{tabular}{rl}
&\textbf{Mood}\\
\textbf{unmarked}& $\rightarrow$ indicative \\
\textbf{wi-}& $\rightarrow$ subjunctive \\
\textbf{se-}& $\rightarrow$ imperative \\
\textbf{ko-} & $\rightarrow$ conditional \\
\textbf{xo-}& $\rightarrow$ counterfactual \\
\textbf{hle-}& $\rightarrow$ optative \\
\textbf{tlo-}& $\rightarrow$ obligative \\
\textbf{ef-}& $\rightarrow$ necessative \\
\textbf{sho-}& $\rightarrow$ potential \\



&\textbf{Voices}\\
\textbf{unmarked}& $\rightarrow$ active \\
\textbf{te-}& $\rightarrow$ reflexive \\
\textbf{pa-}& $\rightarrow$ passive \\
\textbf{mo-}& $\rightarrow$ middle \\
\textbf{ke-}& $\rightarrow$ causative \\
\textbf{ra-}& $\rightarrow$ reciprocal \\

% &\textbf{Other (not combined)}\\
% \textbf{li-}& $\rightarrow$ gerund \\

\end{tabular}

\medskip

Mood and voice can be combined by adding the mood prefix before the voice prefix. For example, adding the obligative mood marker \textit{tlo-} and the reflexive voice marker \textit{te-} to the habitual aspect auxilary root stem \textit{m-} with the present tense conjugation \textit{-a}, the final auxilary verb is \textit{tlotema}. 

\ex
\begingl
\gla idriel tlotema lorsel //
\glb NOM-person OBLIG-REFL-HAB.PRES wash-INF //
\glft "She should habitually wash herself." //
\endgl
\xe



\subsection{Unused Combinations}
Tovian is very flexible when it comes to combining different tenses, aspects, moods, and voices. However, certain verb conjugations result in combinations that are unlikely or never used due to their impractical or nonsensical nature. For example, the Past Imperative is very rare because commands most often refer to actions in the present or future.



\section{Copulas}
Copulas are formed by combining verbs and nouns to describe states and qualities. For example the word "mel" (life/living) can be used as a copula: "i-la-driel fa mela" means "The person is alive," or equally "The person lives" This method applies to various attributes by changing the verb root and noun. For instance, "driel fe araya" translates to "The person was wise." 

 \section{Comparative Constructions}

    Tovian uses comparative constructions to express excessiveness, for example: "bigger than appropriate." The ablative case is often used for this comparison: "the tree is too big" could be expressed as \textit{ilo-glar \CAabl-le-velar fa mota} ("The tree is big away-from-worthiness" or simply "The tree is bigger than what is worthy/right").

\section{Adjectives}

Tovian does not use a separate class of adjectives. Instead, all descriptive qualities are expressed through nouns. These attributive nouns follow the noun they modify in a fixed noun–noun structure. For example, the phrase \textit{driel aray} translates literally as “person wisdom,” conveying the meaning “the wise person.” To convey "people wisdom" or "the wisdom of people," one would switch the order to: \textit{aray driel}. This structure does not need inflection or agreement between adjectives and nouns, for example "the wise people" would simply be \textit{driele aray}. In cases requiring emphasis or clarification, a copular construction can be used instead: \textit{i-la-driel fa araya} (“The person is wise”). Many commonly used descriptive terms—such as colors, emotional states, or physical traits—exist as standalone nouns and follow the same syntactic pattern. This noun-only system reinforces the conceptual unity of properties and entities, treating attributes not as modifiers but as qualities with independent substance.


\section{Agreement}

Suffixes are added to verbs to indicate the person of the subject.

\begin{table}[H]
    \centering
    \begin{tabular}{|c|c|}
        \hline
        \textbf{Person} & \textbf{Suffix} \\
        \hline
        1st person & i \\
        2nd person & o \\
        3rd person & a \\
        \hline
    \end{tabular}
    % \caption{Person Suffixes in Tovian}
    \label{tab:suffixes}
\end{table}


\begin{table}[H]
    \centering
    \begin{tabular}{|c|c|c|}
        \hline
        \textbf{Phrase} & \textbf{Translation} & \textbf{Person} \\
        \hline
        ilane sa nalh\textbf{i} & "We think" & 1st \\
        ilana sa nalh\textbf{i} & "I think" & 1st \\
        ilawa sa nalh\textbf{o} & "You think" & 2nd \\
        ilawe sa nalh\textbf{o} & "You (plural) think" & 2nd \\
        ilata sa nalh\textbf{a} & "He/she/they think" & 3rd \\
        ilate sa nalh\textbf{a} & "They (plural) think" & 3rd \\
        \hline
    \end{tabular}
    % \caption{Agreement in Tovian}
    \label{tab:agreement}
\end{table}

%  Verb stems are added to nouns and always end in ``lore". Different tenses, aspects, moods, and voices can be applied by adding prefixes. The prefixes in the order they should be applied are as follows:\footnote{\noindent Complex verb forms are saved for formal communication.}

% \subsection{Simple Verb Forms}

% \begin{tabular}{rl}
% &\textbf{Tense}\\
% \textbf{} &$\rightarrow$$ present tense \\
% \textbf{vi-} &$\rightarrow$$ future tense \\
% \textbf{in-} &$\rightarrow$$ past tense \\

% &\textbf{Aspect}\\
% \textbf{de-}& $\rightarrow$ perfect \\
% \textbf{me-} & $\rightarrow$ participle \\
% \textbf{shi-}& $\rightarrow$ continuous \\

% &\textbf{Mood}\\
% \textbf{ko-} & $\rightarrow$ conditional \\
% \textbf{se-}& $\rightarrow$ imperative \\
% \textbf{wi-}& $\rightarrow$ subjunctive \\

% &\textbf{Voices}\\
% \textbf{te-}& $\rightarrow$ reflexive \\
% \textbf{pe-}& $\rightarrow$ passive \\

% &\textbf{Other (not combined)}\\
% \textbf{li-}& $\rightarrow$ gerund \\


% \end{tabular}
% % 55 combinations


% \subsection*{All verb forms}
% \begin{tabular}{rl}
% &\textbf{Tense}\\
% \textbf{} &$\rightarrow$$ present tense \\
% \textbf{vi-} &$\rightarrow$$ future tense \\
% \textbf{in-} &$\rightarrow$$ past tense \\
% &\textbf{Aspect}\\
% \textbf{de-}& $\rightarrow$ perfect \\
% \textbf{me-} & $\rightarrow$ participle \\
% \textbf{re-} & $\rightarrow$ habitual \\
% \textbf{ro-}& $\rightarrow$ iterative \\
% \textbf{ve-} & $\rightarrow$ immediate \\
% \textbf{shi-}& $\rightarrow$ continuous \\
% \textbf{wo-}& $\rightarrow$ remote \\
% \textbf{to-}& $\rightarrow$ near \\
% \textbf{yi-}& $\rightarrow$ inceptive \\
% \textbf{hlo-}& $\rightarrow$ cessative \\

% &\textbf{Mood}\\
% \textbf{ko-} & $\rightarrow$ conditional \\
% \textbf{se-}& $\rightarrow$ imperative \\
% \textbf{wi-}& $\rightarrow$ subjunctive \\
% \textbf{xo-}& $\rightarrow$ counterfactual \\

% &\textbf{Voices}\\
% \textbf{te-}& $\rightarrow$ reflexive \\
% \textbf{pa-}& $\rightarrow$ passive \\
% \textbf{mo-}& $\rightarrow$ middle \\
% \textbf{ke-}& $\rightarrow$ causative \\
% \textbf{ra-}& $\rightarrow$ reciprocal \\

% &\textbf{Other (not combined)}\\
% \textbf{li-}& $\rightarrow$ gerund \\


% \end{tabular}
% % 601 combinations

% \medskip

% \subsection*{Verb Conjugations}
% Using "speak" as an example: \textit{fethr}

% \subsubsection*{Tense}
% \begin{itemize}
%     \item \textbf{Present Tense:} \textit{} \\
%           \hanging Speak: \textit{lore-fethr}
          
%     \item \textbf{Future Tense:} \textit{vi-} \\
%           \hanging Will speak: \textit{vilore-fethr}
          
%     \item \textbf{Past Tense:} \textit{in-} \\
%           \hanging Spoke: \textit{inlore-fethr}
% \end{itemize}

% \subsubsection*{Aspect}
% \begin{itemize}
%     \item \textbf{Perfect:} \textit{de-} \\
%           \hanging Has spoken: \textit{delore-fethr}
          
%     \item \textbf{Continuous:} \textit{shi-} \\
%           \hanging Is speaking: \textit{shilore-fethr}
          
%     \item \textbf{Habitual:} \textit{re-} \\
%           \hanging Speaks regularly: \textit{relore-fethr}
          
%     \item \textbf{Immediate:} \textit{ve-} \\
%           \hanging Is about to speak: \textit{velore-fethr}
          
%     \item \textbf{Participle:} \textit{ma-} \\
%           \hanging Speaking (as a participle): \textit{malore-fethr}
% \end{itemize}

% \subsubsection*{Mood}
% \begin{itemize}
%     \item \textbf{Conditional:} \textit{ko-} \\
%           \hanging Would speak: \textit{kolore-fethr}
          
%     \item \textbf{Imperative:} \textit{se-} \\
%           \hanging Speak!: \textit{selore-fethr}
          
%     \item \textbf{Subjunctive:} \textit{wi-} \\
%           \hanging May speak: \textit{wilore-fethr}
          
%     \item \textbf{Counterfactual:} \textit{xo-} \\
%           \hanging Would have spoken: \textit{xolore-fethr}
% \end{itemize}

% \subsubsection*{Voice}
% \begin{itemize}
%     \item \textbf{Reflexive:} \textit{te-} \\
%           \hanging Speaks to oneself: \textit{telore-fethr}
          
%     \item \textbf{Passive:} \textit{pa-} \\
%           \hanging Is spoken: \textit{palore-fethr}
          
%     \item \textbf{Middle:} \textit{mo-} \\
%           \hanging Speaks for oneself: \textit{molore-fethr}
          
%     \item \textbf{Causative:} \textit{ke-} \\
%           \hanging Makes someone speak: \textit{kelore-fethr}
          
%     \item \textbf{Reciprocal:} \textit{ra-} \\
%           \hanging They speak to each other: \textit{ralore-fethr}
% \end{itemize}

% \subsubsection*{Other}
% \begin{itemize}
%     \item \textbf{Gerund:} \textit{li-} \\
%           \hanging Speaking (as a noun): \textit{lilore-fethr}
% \end{itemize}

% \subsubsection*{Combined Forms}
% \begin{itemize}
%     \item \textbf{Past Perfect:} \textit{de- + in-} \\
%           \hanging Had spoken: \textit{deinlore-fethr}
          
%     \item \textbf{Future Continuous:} \textit{shi- + vi-} \\
%           \hanging Will be speaking: \textit{shivilore-fethr}
          
%     \item \textbf{Conditional Habitual:} \textit{re- + ko-} \\
%           \hanging Would speak regularly: \textit{rekolore-fethr}
          
%     \item \textbf{Immediate Imperative:} \textit{ve- + se-} \\
%           \hanging Speak immediately!: \textit{veselore-fethr}
          
%     \item \textbf{Subjunctive Participle:} \textit{wi- + ma-} \\
%           \hanging Speaking in a hypothetical or wished-for sense: \textit{wimalore-fethr}
% \end{itemize}


% \subsection*{To be and to be like}
% All verbs with the exception of two follow the above structure. To express that the subject of the sentence is equivalent to the object, use ``'as" as a postfix on the object. To express similarity, use ``'asl" instead. For example:

% \begin{itemize}
%   \item \textbf{To be:} \\
%         \textit{Tel \CAacc-\CLin-helor'as.} (The lake is glass.) \\
%         - \textit{Tel} (subject: the lake) \\
%         - \textit{\CAacc-\CLin-helor'as} (object: glass with ``as" indicating "is")

%  \item \textbf{To be like:} \\
%         \textit{Tel \CAacc-\CLin-helor'asl.} (The lake is like glass.) \\
%         - \textit{Tel} (subject: the lake) \\
%         - \textit{helor'as} (object: glass with ``asl'" indicating "is like")

% \end{itemize}

